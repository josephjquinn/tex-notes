\documentclass[11pt]{article}
\usepackage{amsmath, amssymb, amsthm}
\usepackage{geometry}
\usepackage{graphicx}
\geometry{margin=1in}
\setlength\parindent{0pt}
\title{Foundations of Machine Learning -- Neural Networks}
\author{}
\date{}

\begin{document}
\maketitle


\section*{Neural Networks}

Lets revisit our regression optimization

\medskip

\textbf{Loss:}

\[
\arg\min_{w} \frac{1}{N} \sum_{n=1}^{N} \left( w^T \phi(x^{(i)}) - y^{(i)} \right)^2 
= \arg\min_{w} \| \Phi w - y \|^2
\]

\[
\text{Where } \Phi = 
\begin{bmatrix}
\phi(x^{(1)}), \ldots, \phi(x^{(N)})
\end{bmatrix}^T 
\in \mathbb{R}^{N \times M}, \quad
w \in \mathbb{R}^M
\]

\textbf{Optimization:}

\begin{enumerate}
    \item Closed form solution:
    \[
    w^* = (\Phi^T \Phi)^{-1} \Phi^T y
    \]

    \item Gradient descent:
    \[
    w^{(t)} = w^{(t-1)} - \epsilon \nabla_w \text{Loss}(w^{(t-1)})
    \]
\end{enumerate}

Before, it was more or less a guess an check system to determine which bias function would be best to use for a given
dataset, but with neural networks we can learn the bias function during training!

\[
\min_{w, \{\phi_j\}_{j=1}^M} 
\sum_i 
\left(
    \left(
        \sum_j w_j \phi_j(x^{(i)})
    \right)
    - y^{(i)}
\right)^2
\]

The building block of neural networks is the perceptron

\begin{figure}[h]
	\centering
	\includegraphics[width=0.5\textwidth]{../imgs/perceptron.png} % 
\end{figure}

\pagebreak

\paragraph{Multilayer Perceptron (MLP)}
We can take this building block and create a system with multiple hidden layers.

\medskip

The Universal Approximation Theorem states that an MLP An MLP with varying complexity can approximate any continuous function to any desired degree of accuracy.
\begin{figure}[h]
	\centering
	\includegraphics[width=0.5\textwidth]{../imgs/mlp-ex.png} % 
\end{figure}

\subsection*{Problems with MLP}
A 256x256 (RGB) image $\Rightarrow$ roughly 200k dimensional input x

A fully connected network would need a very large number of parameters, and very likely would overfit the data

Also generic deep networks also do not capture hte "natural" invariances we expect in images (translation, scale)
\begin{figure}[h]
	\centering
	\includegraphics[width=0.9\textwidth]{../imgs/img-transofrmations.png} 
\end{figure}

\pagebreak

\subsection*{Convolution}

MLPs treat every pixel as unique, so they must relearn the same feature (like a beak) at each position, 
wasting parameters and computation. CNNs solve this by sharing weights through filters that slide across the image, 
allowing one detector to recognize a pattern anywhere—making learning more efficient and translation-invariant.


\begin{figure}[h]
	\centering
	\includegraphics[width=0.9\textwidth]{../imgs/convulution-ex.png}  
\end{figure}



\end{document}


