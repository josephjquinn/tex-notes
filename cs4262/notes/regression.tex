

\documentclass[11pt]{article}
\usepackage{amsmath, amssymb, amsthm}
\usepackage{geometry}
\geometry{margin=1in}
\title{Foundations of Machine Learning -- Lecture 2 Notes}
\author{}
\date{}

\begin{document}
\maketitle

\section*{Regression}
Model: $y=f(x)+ n$,
Where $n$ is a random variable that represents measurement errors, unobserved influences, etc.


\subsection*{Mean Squared Error}
We aim to choose a model $f$ that minimizes the expected prediction error, i.e., we want $f(x)$ to be as close as possible to $y$ for all $x$.

\medskip

\textbf{Error for a single data point:}
$n^{(i)} = f\big(x^{(i)}\big) - y^{(i)}$

\textbf{MSE:}
$\min_{f} \; \mathbb{E}_{x,y}\!\left[ \big(f(x) - y\big)^2 \right]$

This requires knowledge of the full probability distribution of $(x,y)$, which is not available in practice (e.g., knowing the price of every house in an entire city).
So, instead we will use the empirical version.

\[
	\min_{f} \; \frac{1}{N} \sum_{i=1}^{N} \Big(f\big(x^{(i)}\big) - y^{(i)}\Big)^2
\]




\subsection*{Associating Living Area with the Price of a House}

We assume the house price $y$ can be modeled as $y = f(x) + n$,
where $x$ is the living area and $n$ is noise.

\medskip

Suppose $n$ follows a Gaussian distribution:
\[
	n \sim \mathcal{N}(0, \sigma^2).
\]

Therefore, the conditional distribution of $y$ is
\[
	y \sim \mathcal{N}\!\big(f(x), \sigma^2\big).
\]

Equivalently, the conditional density is given by
\[
	p(y \mid x, f, \sigma) \;=\; \frac{1}{\sqrt{2\pi\sigma^2}}
	\exp\!\left( -\frac{\big(y - f(x)\big)^2}{2\sigma^2} \right).
\]

This tells you how probable $y$ is, given the living area
$x$, the function $f$, and the noise level $\sigma$

\pagebreak

Assume $(x_i, y_i)$ are independent and identically distributed (i.i.d). Then:

\[
	p(y \mid x, f, \sigma)
	\;=\;
	\prod_{i=1}^N p\!\big(y^{(i)} \mid x^{(i)}, f, \sigma\big).
\]

This is called the \emph{likelihood} of our model $f$, with respect to the observed data
\(\{(x^{(i)}, y^{(i)})\}_{i=1}^N\).

This tells us, how well our function represents the data observed?

\medskip

Working with this product over $N$ is not easy. Let’s take
the $\log(\cdot)$:

\begin{align*}
	\arg\max_{f} \; \prod_{i=1}^N p\!\big(y^{(i)} \mid x^{(i)}, f, \sigma\big)
	 & = \arg\max_{f} \; \sum_{i=1}^N \log p\!\big(y^{(i)} \mid x^{(i)}, f, \sigma\big)                \\
	 & = \arg\min_{f} \; \Bigg(- \sum_{i=1}^N \log p\!\big(y^{(i)} \mid x^{(i)}, f, \sigma\big)\Bigg).
\end{align*}


\section*{Linear Regression}
In linear regression, our regressor, $f$, is linear (affine).

\[
	f(x) = w_1x+w_0
\]

Therefore,

\[
	\min_{w_0, w_1} \; \frac{1}{N} \sum_{i=1}^N \Big(y^{(i)} - w_1 x^{(i)} - w_0 \Big)^2
\]


In matrix notation we start with
\[
	\min_{w} \; \frac{1}{N} \sum_{i=1}^N \Big(y^{(i)} - w^T \hat{x}^{(i)}\Big)^2,
	\quad \text{where } \hat{x}^{(i)} = \begin{bmatrix} 1 \\ x^{(i)} \end{bmatrix}.
\]

Now, collect all data into matrices:


\begin{itemize}
	\item $y = \big[y^{(1)}, \ldots, y^{(N)}\big]^T \in \mathbb{R}^{N \times 1}$
	\item $X = \big[\hat{x}^{(1)}, \ldots, \hat{x}^{(N)}\big]^T \in \mathbb{R}^{N \times (d+1)}
		      \quad \text{(here $d = 1$)}$
	\item $w = \big[w_0, w_1, \ldots, w_d\big]^T \in \mathbb{R}^{d+1}$
\end{itemize}


\[
	w =
	\begin{bmatrix}
		w_0 \\ w_1
	\end{bmatrix},
	\qquad
	y =
	\begin{bmatrix}
		y^{(1)} \\ y^{(2)} \\ \vdots \\ y^{(N)}
	\end{bmatrix},
	\qquad
	X =
	\begin{bmatrix}
		1      & x^{(1)} \\
		\vdots & \vdots  \\
		1      & x^{(N)}
	\end{bmatrix}.
\]

In compact notation, the objective becomes
\[
	\min_{w} \; \frac{1}{N} \|y - Xw\|^2.
\]

Now lets solve for w in order to find our optimal weights.

\begin{align*}
	\mathcal{L} & = \|Xw - y\|^2                      \\
	            & = (Xw - y)^T (Xw - y)               \\
	            & = (w^T X^T - y^T)(Xw - y)           \\
	            & = w^T X^T X w - 2 w^T X^T y + y^T y
\end{align*}

\[
	\nabla_w \mathcal{L} = 2X^T X w - 2X^T y
\]

At the optimal point the gradient should be the zero vector:

\[
	\nabla_w \mathcal{L} = 0 \;\;\;\;\Rightarrow\;\;\;\; X^T X w - X^T y = 0
	\;\;\;\;\Rightarrow\;\;\;\; w^* = (X^T X)^{-1} X^T y
\]


\subsection*{What can go wrong with this solution?}
\begin{enumerate}
	\item $X^T X$ might not be full-rank / invertible! For example, if $N < d+1$.
	\item $X^T X \in \mathbb{R}^{(d+1) \times (d+1)}$, where $d$ is the dimensionality of the input $x$. \\
	      Matrix inversion is, generally speaking, cubic in matrix size, i.e.\ $O((d+1)^3)$! \\
	      This can be very slow for high-dimensional data.
\end{enumerate}


When $X^T X$ is not full-rank/invertible, we can use the Moore--Penrose Pseudoinverse.
\[
	w^* = \underbrace{(X^T X + \lambda I)^{-1} X^T y}_{\text{Pseudoinverse}}
\]

Here, $I$ is the identity matrix and $\lambda$ is the regularization parameter.

A higher $\lambda$ means a higher penalty on large coefficients.

\bigskip

\textbf{Ridge Regression:}
\[
	\arg\min_w \; \|Xw - y\|^2 + \lambda \|w\|^2
\]

Expanding:
\[
	\mathcal{L} = w^T X^T X w - 2 w^T X^T y + y^T y + \lambda w^T w
\]

Gradient:
\[
	\nabla_w \mathcal{L} = 2X^T X w - 2X^T y + 2\lambda w
\]

Setting the gradient to zero:
\[
	\nabla_w \mathcal{L} = 0 \;\;\;\;\Rightarrow\;\;\;\; w^* = (X^T X + \lambda I)^{-1} X^T y
\]

\section*{Gradient Descent}


For large $d$, to avoid matrix inversion we can use the Gradient Descent algorithm.
The idea is to iteratively update the parameters to reach the optimal point.

\medskip

\noindent Gradient always points in the direction of increasing the cost, so we follow the negative gradient:

\[
	w^{(t)} = w^{(t-1)} - \epsilon \nabla_w \mathcal{L}
\]

$\epsilon$ is the learning rate, a positive scalar that determines the size of the update step.

\pagebreak

For our linear regression problem:
\[
	\nabla_w \mathcal{L} = 2X^T X w - 2X^T y
\]

Therefore, the update equations will be as follows:
\[
	w^{(t)} = w^{(t-1)} - 2\epsilon \big(X^T X w^{(t-1)} - X^T y\big)
\]
\[
	= (I - \epsilon' X^T X) w^{(t-1)} + \epsilon' X^T y
\]

Gradient Descent is sensitive to the choice of learning rate:
\begin{itemize}
	\item Too small = Too slow
	\item Too large = Messy oscillatory Behavior
\end{itemize}


\subsection*{More on gradient descent}

Let $g = \nabla_w \mathcal{L}(w^{(0)})$ and let $H$ be the Hessian matrix such that
\[
	[H]_{ij} = \frac{\partial^2 \mathcal{L}(w^{0})}{\partial w_i \partial w_j}
\]

Recall the Taylor series of a function (second order):
\[
	\mathcal{L}(w) \approx \mathcal{L}(w^{(0)}) + (w - w^{(0)})^T g
	+ \frac{1}{2}(w - w^{(0)})^T H (w - w^{(0)})
\]

Now substitute the gradient descent update $w = w^{(0)} - \epsilon g$
into the Taylor series:
\[
	\mathcal{L}(w^{(0)} - \epsilon g)
	= \mathcal{L}(w^{(0)}) - \epsilon g^T g
	+ \tfrac{1}{2}\epsilon^2 g^T H g.
\]

We want the loss to decrease after the update:
\[
	\mathcal{L}(w^{(0)} - \epsilon g) < \mathcal{L}(w^{(0)}).
\]

\begin{itemize}
	\item When \( g^T H g \leq 0 \), we have no issue (but no convergence either), e.g. if \(\mathcal{L}(w)\) is concave.
	\item When \( g^T H g > 0 \), to satisfy the inequality we must choose \(\epsilon\) carefully:
\end{itemize}

\[
	\epsilon g^T g > \tfrac{1}{2}\epsilon^2 g^T H g
	\quad \Rightarrow \quad
	\epsilon < \frac{2 g^T g}{g^T H g}.
\]


\subsection*{Line Search}
This is a method used for selecting a learning rate.

\medskip

First set a fixed set of learning rates:
\[
	E := [\epsilon_1, \ldots, \epsilon_L]
\]

Then calculate
\[
	\mathcal{L}\!\Big(w^{(t-1)} - \epsilon_l \nabla_w \big(w^{(t-1)}\big)\Big)
\]

for all $l \in [1, \ldots, L]$ and pick the $\epsilon_l$ that minimizes the loss.

\medskip

This method works great when loss evaluation is cheap!




\end{document}

