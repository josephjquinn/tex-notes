
\documentclass[11pt]{article}
\usepackage{amsmath, amssymb, amsthm}
\usepackage{geometry}
\geometry{margin=1in}
\title{Foundations of Machine Learning -- Lecture 1 Notes}
\author{Based on slides by Thomas Beckers}
\date{}

\begin{document}
\maketitle

\section*{Bias-Variance Decomposition}

We assume the data follows:
\[
	y = f^*(x) + n, \quad \text{where } n \sim \mathcal{N}(0, \sigma^2).
\]
Here, $f^*(x)$ is the true function and $n$ is Gaussian noise.
The dataset is:
\[
	\mathcal{D} = \{ (x^{(i)}, y^{(i)}) \}.
\]

\subsection*{Model Error}
Given a learned model $f(x)$, we want to evaluate its expected error at a test input $x$ across all possible training sets:
\[
	\text{err}(x) = \mathbb{E}_{\mathcal{D}} \big[ (f(x) - y)^2 \big].
\]

\subsection*{Decomposition}
This error can be decomposed into three terms:
\[
	\text{err}(x) = \underbrace{(\mathbb{E}[f(x)] - f^*(x))^2}_{\text{Bias}^2}
	+ \underbrace{\mathbb{E}\big[(f(x) - \mathbb{E}[f(x)])^2\big]}_{\text{Variance}}
	+ \underbrace{\sigma^2}_{\text{Irreducible Error}}.
\]

\begin{itemize}
	\item \textbf{Bias$^2$}: Measures how far the average model prediction is from the true function.
	\item \textbf{Variance}: Captures how much the model’s predictions fluctuate across different training sets.
	\item \textbf{Irreducible Error}: Comes from inherent noise in the data ($\sigma^2$).
\end{itemize}

Thus, the total expected error is a combination of bias, variance, and irreducible noise.

\paragraph{Recap}
\begin{itemize}
	\item Bias penalizes being systematically wrong.
	\item Variance penalizes being unstable across datasets.
	\item Noise is always there, no matter what.
\end{itemize}

\begin{itemize}
  \item Underfitting $\rightarrow$ High Bias $\rightarrow$ Low Variance
  \item Overfitting $\rightarrow$ Low Bias $\rightarrow$ High Variance
\end{itemize}

\end{document}


