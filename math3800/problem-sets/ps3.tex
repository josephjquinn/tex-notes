\documentclass{article}

\usepackage{fancyhdr}
\usepackage{extramarks}
\usepackage{amsmath}
\usepackage{amsthm}
\usepackage{amsfonts}
\usepackage{tikz}
\usepackage[plain]{algorithm}
\usepackage{algpseudocode}
\usepackage{amsmath,amssymb}

\usetikzlibrary{automata,positioning}

%
% Basic Document Settings
%

\topmargin=-0.45in
\evensidemargin=0in
\oddsidemargin=0in
\textwidth=6.5in
\textheight=9.0in
\headsep=0.25in

\linespread{1.1}

\pagestyle{fancy}
\lhead{\hmwkAuthorName}
\chead{\hmwkClass: \hmwkTitle}
\rhead{}
\lfoot{\lastxmark}
\cfoot{\thepage}

\renewcommand\headrulewidth{0.4pt}
\renewcommand\footrulewidth{0.4pt}

\setlength\parindent{0pt}

%
% Create Problem Sections
%

\newcommand{\enterProblemHeader}[1]{
    \nobreak\extramarks{}{Problem \arabic{#1} continued on next page\ldots}\nobreak{}
    \nobreak\extramarks{Problem \arabic{#1} (continued)}{Problem \arabic{#1} continued on next page\ldots}\nobreak{}
}

\newcommand{\exitProblemHeader}[1]{
    \nobreak\extramarks{Problem \arabic{#1} (continued)}{Problem \arabic{#1} continued on next page\ldots}\nobreak{}
    \stepcounter{#1}
    \nobreak\extramarks{Problem \arabic{#1}}{}\nobreak{}
}

\setcounter{secnumdepth}{0}
\newcounter{partCounter}
\newcounter{homeworkProblemCounter}
\setcounter{homeworkProblemCounter}{1}
\nobreak\extramarks{Problem \arabic{homeworkProblemCounter}}{}\nobreak{}



% Define theorem environment
\newtheorem*{theorem}{Theorem}

%
% Homework Problem Environment
%
% This environment takes an optional argument. When given, it will adjust the
% problem counter. This is useful for when the problems given for your
% assignment aren't sequential. See the last 3 problems of this template for an
% example.
%
\newenvironment{homeworkProblem}[1][-1]{
    \ifnum#1>0
        \setcounter{homeworkProblemCounter}{#1}
    \fi
    \section{Problem \arabic{homeworkProblemCounter}}
    \setcounter{partCounter}{1}
    \enterProblemHeader{homeworkProblemCounter}
}{
    \exitProblemHeader{homeworkProblemCounter}
}

%
% Homework Details
%   - Title
%   - Due date
%   - Class
%   - Section/Time
%   - Instructor
%   - Author
%


\newcommand{\hmwkTitle}{Problem set\ 3}
\newcommand{\hmwkDueDate}{September 7, 2025}
\newcommand{\hmwkClass}{Number Theory}
\newcommand{\hmwkClassTime}{Section 2}
\newcommand{\hmwkClassInstructor}{Dr. Eleanor McSpirit}
\newcommand{\hmwkAuthorName}{\textbf{Joseph Quinn}}

%
% Title Page
%

\title{
    \vspace{2in}
    \textmd{\textbf{\hmwkClass:\ \hmwkTitle}}\\
    \normalsize\vspace{0.1in}\small{\hmwkDueDate}\\
    \vspace{0.1in}\large{\textit{\hmwkClassInstructor\ \hmwkClassTime}}
    \vspace{3in}
}

\author{\hmwkAuthorName}
\date{}

\renewcommand{\part}[1]{\textbf{\large Part \Alph{partCounter}}\stepcounter{partCounter}\\}

%
% Various Helper Commands
%

% Useful for algorithms
\newcommand{\alg}[1]{\textsc{\bfseries \footnotesize #1}}

% For derivatives
\newcommand{\deriv}[1]{\frac{\mathrm{d}}{\mathrm{d}x} (#1)}

% For partial derivatives
\newcommand{\pderiv}[2]{\frac{\partial}{\partial #1} (#2)}

% Integral dx
\newcommand{\dx}{\mathrm{d}x}

% Alias for the Solution section header
\newcommand{\solution}{\textbf{\large Solution}}

% Probability commands: Expectation, Variance, Covariance, Bias
\newcommand{\E}{\mathrm{E}}
\newcommand{\Var}{\mathrm{Var}}
\newcommand{\Cov}{\mathrm{Cov}}
\newcommand{\Bias}{\mathrm{Bias}}

%  proof-step macro:
\newcommand{\step}[2]{& #1 & & \text{#2} \\}

\begin{document}

\maketitle
\pagebreak

\begin{homeworkProblem}
	$7u + 31v = 1$.
	\paragraph{(i)}
	Using the Euclidean algorithm we see,
	\[
		(31,7) \Rightarrow 31 = (4 \cdot 7) + 3
	\]
	\[
		(7,3) \Rightarrow 7 = (2 \cdot 3) + 1
	\]
	\[
		(3,1) \Rightarrow 3 = (3 \cdot 1) + 0
	\]

	Therefore,
	\[
		gcd(31,7) = 1
	\]

	\paragraph{(ii)}
	Using the extended Euclidean algorithm we see,
	$1 = 7 - (2 \cdot 3)$ and $3 = 31 - (4 \cdot 7)$
	Using substitution we see,
	\[
		1 = 7-(2 \cdot (31 - 4 \cdot 7)) = 7-(2 \cdot (31 - 28)) = 9 \cdot 7 - 2 \cdot 31
	\]

	Now, looking at our equation we see we have a solution being $u = 9$ and $v=-2$
	\paragraph{(iii)}
	Reducing $7x \equiv 1 \pmod{31}$ gives
	\[
		31 \mid (7x - 1).
	\]

	And by definition of divisibility:
	\[
		\exists\, k \in \mathbb{Z} \;\; \text{such that} \;\; 7x - 1 = 31k
	\]

	From part (ii), we have
	\[
		7 \cdot 9 - 2 \cdot 31 = 1
	\]

	And by rearranging
	\[
		7 \cdot 9 - 1 = 31 \cdot 2
	\]

	This shows $7 \cdot 9 - 1$ is divisible by 31, proving $x$ = 9 is a solution.

	\paragraph{(iv)}

	The general divisibility condition is:
	\[
		7x - 1 = 31k, \quad k \in \mathbb{Z}.
	\]

	Since one solution is \(x = 9\), the complete set of solutions is:
	\[
		x = 9 + 31m, \quad m \in \mathbb{Z}.
	\]

	Check:
	\[
		7(9 + 31m) - 1 = 63 + 217m - 1,
	\]
	\[
		= 62 + 217m,
	\]
	\[
		= 31 \cdot 2 + 31 \cdot 7m,
	\]
	\[
		= 31(2 + 7m).
	\]

	Thus \(7x - 1\) is divisible by \(31\) for all \(x = 9 + 31m\).

\end{homeworkProblem}

\pagebreak

\begin{homeworkProblem}
	Find
	\[
		\bigl( 3^{14} \cdot 7^{22} \cdot 11^{5} \cdot 17^{3}, \; 5^{2} \cdot 11^{4} \cdot 13^{8} \cdot 17 \bigr).
	\]

	By the Fundamental Theorem of Arithmetic, we compute the gcd by looking at each prime and taking the minimum exponent that appears in both factorizations.

	First, the left side has $14$ factors of $3$, while the right side has none.
	So the gcd will contain no powers of $3$.

	Next, the left side has $22$ factors of $7$, while the right side has none.
	So the gcd will contain no powers of $7$.

	For the prime $11$, the left side has exponent $5$ and the right side has exponent $4$.
	So the gcd contains $11^{\min(5,4)} = 11^{4}$.

	For the prime $17$, the left side has exponent $3$ and the right side has exponent $1$.
	So the gcd contains $17^{\min(3,1)} = 17$.

	Finally, the primes $5$ and $13$ appear only on the right side and not on the left,
	so they do not appear in the gcd.

	Putting this together, the gcd is
	\[
		11^{4} \cdot 17.
	\]
\end{homeworkProblem}


\begin{homeworkProblem}
	\begin{proof}
		Since $\gcd(a,b) = d$, we may write
		\[
			a = da', \quad b = db'
		\]
		for some integers $a', b'$.

		Suppose $g = \gcd(a', b')$. Then $g \mid a'$ and $g \mid b'$, so
		\[
			dg \mid da' = a \quad \text{and} \quad dg \mid db' = b.
		\]

		Hence $dg$ is a common divisor of $a$ and $b$. But since $d = \gcd(a,b)$
		is already the greatest common divisor, no larger common divisor can exist.
		We can then conclude that $g = 1$.

		Thus
		\[
			\gcd\!\left(\tfrac{a}{d}, \tfrac{b}{d}\right) = 1.
		\]
	\end{proof}
\end{homeworkProblem}


\begin{homeworkProblem}

\end{homeworkProblem}

\begin{homeworkProblem}
	\paragraph{(i)}

	Case 1: $a<b$
	\begin{align*}
		m+n & = p^a u + p^b v                 \\
		    & = p^a\Big(u + p^{\,b-a} v\Big).
	\end{align*}
	Now $b-a\ge 1$, so $p\mid p^{\,b-a}v$. Also $p\nmid u$. Hence
	\[
		u+p^{\,b-a}v \equiv u \not\equiv 0 \pmod p,
	\]
	so $p\nmid\big(u+p^{\,b-a}v\big)$. Therefore no extra factor of $p$ sits in the bracket, and
	\[
		p^a \parallel (m+n).
	\]

	Case 2: $b<a$. By symmetry (swap the roles of $m,n$), the same reasoning gives
	\[
		p^b \parallel (m+n).
	\]

	Based on this we can conclude,
	\[
		p^{\min(a,b)} \parallel (m+n).
	\]

	\medskip

	\paragraph{(ii)}
	Compute:
	\begin{align*}
		mn & = (p^a u)(p^b v) \\
		   & = p^{a+b}\,(uv).
	\end{align*}
	Since $p\nmid u$ and $p\nmid v$, we have $p\nmid uv$ (if $p$ divided $uv$, it would divide at least one of $u$ or $v$). Thus no further $p$-power divides $uv$, and
	\[
		p^{a+b} \parallel (mn).
	\]

	\medskip

	\paragraph{(iii)}

	With $m=p^a u$ and $p\nmid u$,
	\begin{align*}
		m^n & = (p^a u)^n    \\
		    & = p^{an}\,u^n.
	\end{align*}
	If $p\nmid u$ then $p\nmid u^n$ (a prime dividing a power would divide the base). Hence no further $p$-power divides $u^n$, and
	\[
		p^{an} \parallel (m^n).
	\]
\end{homeworkProblem}

\begin{homeworkProblem}

	\begin{proof}
		Let $d = \gcd(u,v)$. We will show that $d=1$.

		Since $d \mid u$ and $u \mid a$, we can write that $d \mid a$.
		And, since $d \mid v$ and $v \mid b$, we can write that $d \mid b$.
		Therefore, $d$ divides both $a$ and $b$, so $d \mid \gcd(a,b)$.

		By hypothesis, $\gcd(a,b) = 1$. Hence $d \mid 1$, which forces $d=1$.

		Thus $\gcd(u,v) = 1$, as required.
	\end{proof}



\end{homeworkProblem}

\end{document}
