
\documentclass{article}

\usepackage{fancyhdr}
\usepackage{extramarks}
\usepackage{amsmath}
\usepackage{amsthm}
\usepackage{amsfonts}
\usepackage{tikz}
\usepackage[plain]{algorithm}
\usepackage{algpseudocode}
\usepackage{amsmath,amssymb}

\usetikzlibrary{automata,positioning}

%
% Basic Document Settings
%

\topmargin=-0.45in
\evensidemargin=0in
\oddsidemargin=0in
\textwidth=6.5in
\textheight=9.0in
\headsep=0.25in

\linespread{1.1}

\pagestyle{fancy}
\lhead{\hmwkAuthorName}
\chead{\hmwkClass: \hmwkTitle}
\rhead{}
\lfoot{\lastxmark}
\cfoot{\thepage}

\renewcommand\headrulewidth{0.4pt}
\renewcommand\footrulewidth{0.4pt}

\setlength\parindent{0pt}

%
% Create Problem Sections
%

\newcommand{\enterProblemHeader}[1]{
    \nobreak\extramarks{}{Problem \arabic{#1} continued on next page\ldots}\nobreak{}
    \nobreak\extramarks{Problem \arabic{#1} (continued)}{Problem \arabic{#1} continued on next page\ldots}\nobreak{}
}

\newcommand{\exitProblemHeader}[1]{
    \nobreak\extramarks{Problem \arabic{#1} (continued)}{Problem \arabic{#1} continued on next page\ldots}\nobreak{}
    \stepcounter{#1}
    \nobreak\extramarks{Problem \arabic{#1}}{}\nobreak{}
}

\setcounter{secnumdepth}{0}
\newcounter{partCounter}
\newcounter{homeworkProblemCounter}
\setcounter{homeworkProblemCounter}{1}
\nobreak\extramarks{Problem \arabic{homeworkProblemCounter}}{}\nobreak{}



% Define theorem environment
\newtheorem*{theorem}{Theorem}

%
% Homework Problem Environment
%
% This environment takes an optional argument. When given, it will adjust the
% problem counter. This is useful for when the problems given for your
% assignment aren't sequential. See the last 3 problems of this template for an
% example.
%
\newenvironment{homeworkProblem}[1][-1]{
    \ifnum#1>0
        \setcounter{homeworkProblemCounter}{#1}
    \fi
    \section{Problem \arabic{homeworkProblemCounter}}
    \setcounter{partCounter}{1}
    \enterProblemHeader{homeworkProblemCounter}
}{
    \exitProblemHeader{homeworkProblemCounter}
}

%
% Homework Details
%   - Title
%   - Due date
%   - Class
%   - Section/Time
%   - Instructor
%   - Author
%


\newcommand{\hmwkTitle}{Problem set\ 4}
\newcommand{\hmwkDueDate}{September 16, 2025}
\newcommand{\hmwkClass}{Number Theory}
\newcommand{\hmwkClassTime}{Section 2}
\newcommand{\hmwkClassInstructor}{Dr. Eleanor McSpirit}
\newcommand{\hmwkAuthorName}{\textbf{Joseph Quinn}}

%
% Title Page
%

\title{
    \vspace{2in}
    \textmd{\textbf{\hmwkClass:\ \hmwkTitle}}\\
    \normalsize\vspace{0.1in}\small{\hmwkDueDate}\\
    \vspace{0.1in}\large{\textit{\hmwkClassInstructor\ \hmwkClassTime}}
    \vspace{3in}
}

\author{\hmwkAuthorName}
\date{}

\renewcommand{\part}[1]{\textbf{\large Part \Alph{partCounter}}\stepcounter{partCounter}\\}

%
% Various Helper Commands
%

% Useful for algorithms
\newcommand{\alg}[1]{\textsc{\bfseries \footnotesize #1}}

% For derivatives
\newcommand{\deriv}[1]{\frac{\mathrm{d}}{\mathrm{d}x} (#1)}

% For partial derivatives
\newcommand{\pderiv}[2]{\frac{\partial}{\partial #1} (#2)}

% Integral dx
\newcommand{\dx}{\mathrm{d}x}

% Alias for the Solution section header
\newcommand{\solution}{\textbf{\large Solution}}

% Probability commands: Expectation, Variance, Covariance, Bias
\newcommand{\E}{\mathrm{E}}
\newcommand{\Var}{\mathrm{Var}}
\newcommand{\Cov}{\mathrm{Cov}}
\newcommand{\Bias}{\mathrm{Bias}}

%  proof-step macro:
\newcommand{\step}[2]{& #1 & & \text{#2} \\}

\begin{document}

\maketitle
\pagebreak

\begin{homeworkProblem}
	\paragraph{(i)}
	\begin{proof}

		By proof of contraction lets assume there does exist numbers x and y such that $x^2n = y^2$.

		By the Fundamental Theorem of Arithmetic, there are distinct primes $q_1,\dots,q_m$ and exponents $r_1,\dots,r_m\in\mathbb{N}$ with
		\[
			n = q_1^{r_1} q_2^{r_2} \cdots q_m^{r_m}.
		\]
		Since $n$ is not a square, at least one exponent $r_j$ is odd.


		Now, based on the same theorem, we can write,
		\[
			x = q_1^{a_1} q_2^{a_2} \cdots q_k^{a_k}, \qquad
			y = q_1^{b_1} q_2^{b_2} \cdots q_l^{b_l},
		\]

		Then, based on the rules of exponents we can write.
		\[
			x^2 = q_1^{2a_1} q_2^{2a_2} \cdots q_k^{2a_k}, \qquad
			y^2 = q_1^{2b_1} q_2^{2b_2} \cdots q_l^{2b_l}.
		\]


		We see that in both $x^2$ and $y^2$, all exponents are even.
		Multiplying $x^2 \cdot n$ gives
		\[
			x^{2}n = \left(q_1^{2a_1} q_2^{2a_2} \cdots q_k^{2a_k}\right)\left(p_1^{r_1} p_2^{r_2} \cdots p_m^{r_m}\right).
		\]

		The left-hand side has at least one odd exponent, while the right-hand side has only even exponents.
		And since we assumed

		\[
			x^{2}n = y^{2}.
		\]
		Contradiction and this cannot exist.
	\end{proof}

	\paragraph{(ii)}
	\begin{proof}
		By proof of contraction lets assume $\sqrt{n}$ is rational

		We can then write
		\[
			\sqrt{n} = \frac{y}{x},
		\]
		Where x and y are integers

		Squaring both sides results in,
		\[
			n = \frac{y^2}{x^2}
		\]
		And multiplying both sides by $x^2$
		\[
			x^2n = y^2
		\]

		This results in the same equation we had from (i), from before, we know: if $n$ is not a square, then no such natural numbers $x,y$ can exist.

		But our assumption gave exactly such a pair $(x,y)$. This is a contradiction.

		Therefore, $\sqrt{n}$ cannot be rational. Hence, if $n$ is not a square,
	\end{proof}

\end{homeworkProblem}

\begin{homeworkProblem}
	\begin{proof}
		Suppose, for contradiction, that there are only finitely many primes congruent to $3 \pmod{4}$.
		List them as $p_1, p_2, \dots, p_n$ and set
		\[
			P = p_1p_2 \cdots p_n
		\]

		Now define
		\[
			N = 4P - 1.
		\]
		Since we know that 4P is divisible by 4, we can write
		\[
			4P \equiv 0 \pmod{4}
		\]
		We also know that.
		\[
			-1 \equiv 3\pmod{4}
		\]
		So by Theorem 1.3 we can write

		\[
			N \equiv -1 \equiv 3 \pmod{4}.
		\]

		We claim that none of the primes $p_i$ divide $N$. Indeed, for each $i$,
		\[
			N = 4P - 1 \equiv -1 \pmod{p_i},
		\]
		because $p_i \mid P$. Thus $p_i \nmid N$.

		Let
		\[
			N = q_1^{e_1} q_2^{e_2} \cdots q_t^{e_t}
		\]
		be the prime factorization of $N$. Each $q_j$ is an odd prime, so $q_j \equiv 1 \pmod{4}$ or $q_j \equiv 3 \pmod{4}$.

		By Theorem~2.37, the product of numbers each congruent to $1 \pmod{4}$ is itself congruent to $1 \pmod{4}$.
		Since $N \equiv 3 \pmod{4}$, not all $q_j$ can be $\equiv 1 \pmod{4}$. Therefore, at least one $q_j \equiv 3 \pmod{4}$.

		This prime $q_j$ is distinct from the listed primes $p_1, \dots, p_n$, contradicting the assumption that we had found all such primes.

		Hence there must be infinitely many primes congruent to $3 \pmod{4}$.
	\end{proof}

\end{homeworkProblem}

\begin{homeworkProblem}
	\begin{proof}
		We want to solve the system of congruences
		\[
			3x \equiv 2 \pmod{4}, \quad 4x \equiv 1 \pmod{5}.
		\]

		First, by Theorem 3.19, the congruence $3x \equiv 2 \pmod{4}$ has a solution iff there exist integers $x_1,y_1$ such that
		\[
			3x + 4y = 2.
		\]

		We compute $\gcd(3,4)$ using the Euclidean algorithm:
		\begin{align}
			4 & = 1\cdot 3 + 1, \\
			3 & = 3\cdot 1 + 0.
		\end{align}
		Tracing back,
		\[
			1 = 4 - 1\cdot 3 = (-1)\cdot 3 + (1)\cdot 4.
		\]
		Multiplying both sides by $2$ gives
		\[
			2 = (-2)\cdot 3 + (2)\cdot 4.
		\]
		Thus $(x_1,y_1)=(-2,2)$ is a particular solution.
		Modulo $4$, this means
		\[
			x \equiv -2 \equiv 2 \pmod{4}.
		\]
		Now moving to the second congruence $4x \equiv 1 \pmod{5}$. By Theorem 3.19, this has a solution iff there exist integers $x_2,y_2$ such that
		\[
			4x + 5y= 1.
		\]

		Using the Euclidean algorithm:
		\begin{align}
			5 & = 1\cdot 4 + 1, \\
			4 & = 4\cdot 1 + 0.
		\end{align}
		So
		\[
			1 = 5 - 1\cdot 4 = (1)\cdot 5 + (-1)\cdot 4.
		\]
		Thus $(x_2,y_2)=(-1,1)$ is a particular solution, meaning
		\[
			x \equiv -1 \equiv 4 \pmod{5}.
		\]

		Now we have reduced the system to
		\[
			x \equiv 2 \pmod{4}, \quad x \equiv 4 \pmod{5}.
		\]

		Write $x = 2 + 4k$ and substitute into the second congruence:
		\[
			2 + 4k \equiv 4 \pmod{5} \quad \Longrightarrow \quad 4k \equiv 2 \pmod{5}.
		\]
		Since $4 \equiv -1 \pmod{5}$, this is
		\[
			-k \equiv 2 \pmod{5} \quad \Longrightarrow \quad k \equiv 3 \pmod{5}.
		\]
		So $k=3+5t$ for $t\in \mathbb{Z}$, and hence
		\[
			x = 2 + 4(3+5t) = 14 + 20t.
		\]

		Therefore, the complete solution set is
		\[
			x \equiv 14 \pmod{20}.
		\]
	\end{proof}
\end{homeworkProblem}

\end{document}
