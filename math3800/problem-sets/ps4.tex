
\documentclass{article}

\usepackage{fancyhdr}
\usepackage{extramarks}
\usepackage{amsmath}
\usepackage{amsthm}
\usepackage{amsfonts}
\usepackage{tikz}
\usepackage[plain]{algorithm}
\usepackage{algpseudocode}
\usepackage{amsmath,amssymb}

\usetikzlibrary{automata,positioning}

%
% Basic Document Settings
%

\topmargin=-0.45in
\evensidemargin=0in
\oddsidemargin=0in
\textwidth=6.5in
\textheight=9.0in
\headsep=0.25in

\linespread{1.1}

\pagestyle{fancy}
\lhead{\hmwkAuthorName}
\chead{\hmwkClass: \hmwkTitle}
\rhead{}
\lfoot{\lastxmark}
\cfoot{\thepage}

\renewcommand\headrulewidth{0.4pt}
\renewcommand\footrulewidth{0.4pt}

\setlength\parindent{0pt}

%
% Create Problem Sections
%

\newcommand{\enterProblemHeader}[1]{
    \nobreak\extramarks{}{Problem \arabic{#1} continued on next page\ldots}\nobreak{}
    \nobreak\extramarks{Problem \arabic{#1} (continued)}{Problem \arabic{#1} continued on next page\ldots}\nobreak{}
}

\newcommand{\exitProblemHeader}[1]{
    \nobreak\extramarks{Problem \arabic{#1} (continued)}{Problem \arabic{#1} continued on next page\ldots}\nobreak{}
    \stepcounter{#1}
    \nobreak\extramarks{Problem \arabic{#1}}{}\nobreak{}
}

\setcounter{secnumdepth}{0}
\newcounter{partCounter}
\newcounter{homeworkProblemCounter}
\setcounter{homeworkProblemCounter}{1}
\nobreak\extramarks{Problem \arabic{homeworkProblemCounter}}{}\nobreak{}



% Define theorem environment
\newtheorem*{theorem}{Theorem}

%
% Homework Problem Environment
%
% This environment takes an optional argument. When given, it will adjust the
% problem counter. This is useful for when the problems given for your
% assignment aren't sequential. See the last 3 problems of this template for an
% example.
%
\newenvironment{homeworkProblem}[1][-1]{
    \ifnum#1>0
        \setcounter{homeworkProblemCounter}{#1}
    \fi
    \section{Problem \arabic{homeworkProblemCounter}}
    \setcounter{partCounter}{1}
    \enterProblemHeader{homeworkProblemCounter}
}{
    \exitProblemHeader{homeworkProblemCounter}
}

%
% Homework Details
%   - Title
%   - Due date
%   - Class
%   - Section/Time
%   - Instructor
%   - Author
%


\newcommand{\hmwkTitle}{Problem set\ 4}
\newcommand{\hmwkDueDate}{September 16, 2025}
\newcommand{\hmwkClass}{Number Theory}
\newcommand{\hmwkClassTime}{Section 2}
\newcommand{\hmwkClassInstructor}{Dr. Eleanor McSpirit}
\newcommand{\hmwkAuthorName}{\textbf{Joseph Quinn}}

%
% Title Page
%

\title{
    \vspace{2in}
    \textmd{\textbf{\hmwkClass:\ \hmwkTitle}}\\
    \normalsize\vspace{0.1in}\small{\hmwkDueDate}\\
    \vspace{0.1in}\large{\textit{\hmwkClassInstructor\ \hmwkClassTime}}
    \vspace{3in}
}

\author{\hmwkAuthorName}
\date{}

\renewcommand{\part}[1]{\textbf{\large Part \Alph{partCounter}}\stepcounter{partCounter}\\}

%
% Various Helper Commands
%

% Useful for algorithms
\newcommand{\alg}[1]{\textsc{\bfseries \footnotesize #1}}

% For derivatives
\newcommand{\deriv}[1]{\frac{\mathrm{d}}{\mathrm{d}x} (#1)}

% For partial derivatives
\newcommand{\pderiv}[2]{\frac{\partial}{\partial #1} (#2)}

% Integral dx
\newcommand{\dx}{\mathrm{d}x}

% Alias for the Solution section header
\newcommand{\solution}{\textbf{\large Solution}}

% Probability commands: Expectation, Variance, Covariance, Bias
\newcommand{\E}{\mathrm{E}}
\newcommand{\Var}{\mathrm{Var}}
\newcommand{\Cov}{\mathrm{Cov}}
\newcommand{\Bias}{\mathrm{Bias}}

%  proof-step macro:
\newcommand{\step}[2]{& #1 & & \text{#2} \\}

\begin{document}

\maketitle
\pagebreak

\begin{homeworkProblem}
	\paragraph{(i)}
	\begin{proof}

		By proof of contraction lets assume there does exist numbers x and y such that $x^2n = y^2$.

		By the Fundamental Theorem of Arithmetic, there are distinct primes $q_1,\dots,q_m$ and exponents $r_1,\dots,r_m\in\mathbb{N}$ with
		\[
			n = q_1^{r_1} q_2^{r_2} \cdots q_m^{r_m}.
		\]
		Since $n$ is not a square, at least one exponent $r_j$ is odd.


		Now, based on the same theorem, we can write,
		\[
			x = q_1^{a_1} q_2^{a_2} \cdots q_k^{a_k}, \qquad
			y = q_1^{b_1} q_2^{b_2} \cdots q_l^{b_l},
		\]

		Then, based on the rules of exponents we can write.
		\[
			x^2 = q_1^{2a_1} q_2^{2a_2} \cdots q_k^{2a_k}, \qquad
			y^2 = q_1^{2b_1} q_2^{2b_2} \cdots q_l^{2b_l}.
		\]


		We see that in both $x^2$ and $y^2$, all exponents are even.
		Multiplying $x^2 \cdot n$ gives
		\[
			x^{2}n = \left(q_1^{2a_1} q_2^{2a_2} \cdots q_k^{2a_k}\right)\left(p_1^{r_1} p_2^{r_2} \cdots p_m^{r_m}\right).
		\]

		The left-hand side has at least one odd exponent, while the right-hand side has only even exponents.
		And since we assumed

		\[
			x^{2}n = y^{2}.
		\]
		Contradiction and this cannot exist.
	\end{proof}

	\paragraph{(ii)}
	\begin{proof}
		By proof of contraction lets assume $\sqrt{n}$ is rational

		We can then write
		\[
			\sqrt{n} = \frac{y}{x},
		\]
		Where x and y are integers

		Squaring both sides results in,
		\[
			n = \frac{y^2}{x^2}
		\]
		And multiplying both sides by $x^2$
		\[
			x^2n = y^2
		\]

		This results in the same equation we had from (i), from before, we know: if $n$ is not a square, then no such natural numbers $x,y$ can exist.

		But our assumption gave exactly such a pair $(x,y)$. This is a contradiction.

		Therefore, $\sqrt{n}$ cannot be rational. Hence, if $n$ is not a square,
	\end{proof}

\end{homeworkProblem}

\begin{homeworkProblem}
	\begin{proof}
		Among the three consecutive odd integers $n, n+2,$ and $n+4$, exactly one must be divisible by $3$.
		To see this, reduce $n$ modulo $3$:
		\[
			n, \; n+2, \; n+4 \equiv n,\; n-1,\; n+1 \pmod{3}.
		\]
		These are three distinct residues modulo $3$, so one of them is congruent to $0 \pmod{3}$.

		If the number divisible by $3$ is greater than $3$, then it is composite and cannot be prime.
		Therefore, the only possibility is that the number divisible by $3$ is exactly $3$ itself.

		This forces $n = 3$, which yields the triple
		\[
			3, \; 5, \; 7,
		\]
		and indeed all three are prime.

		Hence, if $n, n+2, n+4$ are all prime, then $n = 3$.
	\end{proof}


\end{homeworkProblem}

\end{document}
