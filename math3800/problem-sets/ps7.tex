\documentclass{article}

\usepackage{fancyhdr}
\usepackage{extramarks}
\usepackage{amsmath}
\usepackage{amsthm}
\usepackage{amsfonts}
\usepackage{tikz}
\usepackage[plain]{algorithm}
\usepackage{algpseudocode}
\usepackage{amsmath,amssymb}

\usetikzlibrary{automata,positioning}

%
% Basic Document Settings
%

\topmargin=-0.45in
\evensidemargin=0in
\oddsidemargin=0in
\textwidth=6.5in
\textheight=9.0in
\headsep=0.25in

\linespread{1.1}

\pagestyle{fancy}
\lhead{\hmwkAuthorName}
\chead{\hmwkClass: \hmwkTitle}
\rhead{}
\lfoot{\lastxmark}
\cfoot{\thepage}

\renewcommand\headrulewidth{0.4pt}
\renewcommand\footrulewidth{0.4pt}

\setlength\parindent{0pt}

%
% Create Problem Sections
%

\newcommand{\enterProblemHeader}[1]{
    \nobreak\extramarks{}{Problem \arabic{#1} continued on next page\ldots}\nobreak{}
    \nobreak\extramarks{Problem \arabic{#1} (continued)}{Problem \arabic{#1} continued on next page\ldots}\nobreak{}
}

\newcommand{\exitProblemHeader}[1]{
    \nobreak\extramarks{Problem \arabic{#1} (continued)}{Problem \arabic{#1} continued on next page\ldots}\nobreak{}
    \stepcounter{#1}
    \nobreak\extramarks{Problem \arabic{#1}}{}\nobreak{}
}

\setcounter{secnumdepth}{0}
\newcounter{partCounter}
\newcounter{homeworkProblemCounter}
\setcounter{homeworkProblemCounter}{1}
\nobreak\extramarks{Problem \arabic{homeworkProblemCounter}}{}\nobreak{}



% Define theorem environment
\newtheorem*{theorem}{Theorem}

%
% Homework Problem Environment
%
% This environment takes an optional argument. When given, it will adjust the
% problem counter. This is useful for when the problems given for your
% assignment aren't sequential. See the last 3 problems of this template for an
% example.
%
\newenvironment{homeworkProblem}[1][-1]{
    \ifnum#1>0
        \setcounter{homeworkProblemCounter}{#1}
    \fi
    \section{Problem \arabic{homeworkProblemCounter}}
    \setcounter{partCounter}{1}
    \enterProblemHeader{homeworkProblemCounter}
}{
    \exitProblemHeader{homeworkProblemCounter}
}

%
% Homework Details
%   - Title
%   - Due date
%   - Class
%   - Section/Time
%   - Instructor
%   - Author
%


\newcommand{\hmwkTitle}{Problem set\ 7}
\newcommand{\hmwkDueDate}{October 20, 2025}
\newcommand{\hmwkClass}{Number Theory}
\newcommand{\hmwkClassTime}{Section 2}
\newcommand{\hmwkClassInstructor}{Dr. Eleanor McSpirit}
\newcommand{\hmwkAuthorName}{\textbf{Joseph Quinn}}

%
% Title Page
%

\title{
    \vspace{2in}
    \textmd{\textbf{\hmwkClass:\ \hmwkTitle}}\\
    \normalsize\vspace{0.1in}\small{\hmwkDueDate}\\
    \vspace{0.1in}\large{\textit{\hmwkClassInstructor\ \hmwkClassTime}}
    \vspace{3in}
}

\author{\hmwkAuthorName}
\date{}

\renewcommand{\part}[1]{\textbf{\large Part \Alph{partCounter}}\stepcounter{partCounter}\\}

%
% Various Helper Commands
%

% Useful for algorithms
\newcommand{\alg}[1]{\textsc{\bfseries \footnotesize #1}}

% For derivatives
\newcommand{\deriv}[1]{\frac{\mathrm{d}}{\mathrm{d}x} (#1)}

% For partial derivatives
\newcommand{\pderiv}[2]{\frac{\partial}{\partial #1} (#2)}

% Integral dx
\newcommand{\dx}{\mathrm{d}x}

% Alias for the Solution section header
\newcommand{\solution}{\textbf{\large Solution}}

% Probability commands: Expectation, Variance, Covariance, Bias
\newcommand{\E}{\mathrm{E}}
\newcommand{\Var}{\mathrm{Var}}
\newcommand{\Cov}{\mathrm{Cov}}
\newcommand{\Bias}{\mathrm{Bias}}

%  proof-step macro:
\newcommand{\step}[2]{& #1 & & \text{#2} \\}

\begin{document}

\maketitle
\pagebreak

\begin{homeworkProblem}

	\paragraph{(i)}

	\[
		n = pq = 5 \times 11 = 55
	\]
	\[
		\varphi(n) = (p - 1)(q - 1) = 4 \times 10 = 40.
	\]

	\paragraph{(ii)}

	We must have
	\[
		E D \equiv 1 \pmod{\varphi(n)}.
	\]
	Or,
	\[
		E \times 23 \equiv 1 \pmod{40}.
	\]
	Ill use the extended euclidian elgorithm to solve this,

	\[
		\begin{aligned}
			40 & = 1(23) + 17, \\
			23 & = 1(17) + 6,  \\
			17 & = 2(6) + 5,   \\
			6  & = 1(5) + 1,   \\
			5  & = 5(1) + 0.
		\end{aligned}
	\]

	Back-substituting gives:
	\[
		\begin{aligned}
			1 & = 6 - 1(5)              \\
			  & = 6 - 1(17 - 2(6))      \\
			  & = 3(6) - 1(17)          \\
			  & = 3(23 - 1(17)) - 1(17) \\
			  & = 3(23) - 4(17)         \\
			  & = 3(23) - 4(40 - 1(23)) \\
			  & = 7(23) - 4(40).
		\end{aligned}
	\]

	\[
		1 = 7(23) - 4(40).
	\]
	And we see that,
	\[
		E \equiv 7 \pmod{40}.
	\]
	So the public encryption exponent is $E = 7$.

	\paragraph{(iii)}

	The public key consists of the pair $(n, E)$:
	\[
		(n, E) = (55, 7).
	\]


	\paragraph{(iv)}
	\[
		W \equiv C^D \pmod{n} = 9^{23} \pmod{55}.
	\]

	To decrypt we will apply the Chinese Remainder Theorem,

	\[
		55 = 5 \times 11.
	\]
	Compute separately mod $5$ and $11$.

	Mod 5:
	\[
		9 \equiv 4 \pmod{5}, \quad 4^2 \equiv 1 \pmod{5}.
	\]
	Thus,
	\[
		4^{23} = (4^2)^{11} \cdot 4 \equiv 1^{11} \cdot 4 \equiv 4 \pmod{5}.
	\]
	So $W \equiv 4 \pmod{5}$.

	Mod 11:
	\[
		9 \equiv -2 \pmod{11}.
	\]
	By Fermat's Little Theorem, $(-2)^{10} \equiv 1 \pmod{11}$.
	Hence,
	\[
		(-2)^{23} = (-2)^{20} \cdot (-2)^3 \equiv 1^2 \cdot (-8) \equiv -8 \equiv 3 \pmod{11}.
	\]
	So $W \equiv 3 \pmod{11}$.

	CRT:
	\[
		\begin{cases}
			W \equiv 4 \pmod{5}, \\
			W \equiv 3 \pmod{11}.
		\end{cases}
	\]
	Let $W = 3 + 11k$. Substitute into the first congruence:
	\[
		3 + 11k \equiv 4 \pmod{5} \Rightarrow 3 + k \equiv 4 \pmod{5} \Rightarrow k \equiv 1 \pmod{5}.
	\]
	So $k = 1 + 5t$, and thus:
	\[
		W = 3 + 11(1 + 5t) = 14 + 55t.
	\]
	This gives us
	\[
		W \equiv 14 \pmod{55}.
	\]

	Decrypted message: $W = 14$.

\end{homeworkProblem}

\begin{homeworkProblem}
	\paragraph{(i)}

	Let $p$ be a prime.
	An integer $a$ is called a \emph{primitive root modulo $p$} if its powers generate all the nonzero residues modulo $p$.
	and if
	\[
		\operatorname{ord}_p(a) = p - 1,
	\]

	\medskip

	To check if $a$ is a primitive root modulo $p$:
	\begin{enumerate}
		\item Factor $p - 1$ into primes: $p - 1 = q_1^{e_1} q_2^{e_2} \cdots q_t^{e_t}$.
		\item Verify that for each distinct prime factor $q_i$ of $p - 1$,
		      \[
			      a^{(p-1)/q_i} \not\equiv 1 \pmod{p}.
		      \]
		      If this holds for all $q_i$, then $a$ is a primitive root modulo $p$.
	\end{enumerate}

	\paragraph{(i)}

	Here $p = 11$, so $\varphi(11) = 10 = 2 \times 5$.

	We must find $a$ such that $a^{10} \equiv 1 \pmod{11}$, but neither $a^{5} \equiv 1$ nor $a^{2} \equiv 1 \pmod{11}$.

	Lets try $a = 2$.

	\[
		\begin{aligned}
			2^1    & \equiv 2 \pmod{11},           \\
			2^2    & \equiv 4 \pmod{11},           \\
			2^3    & \equiv 8 \pmod{11},           \\
			2^4    & \equiv 16 \equiv 5 \pmod{11}, \\
			2^5    & \equiv 10 \pmod{11},          \\
			2^6    & \equiv 9 \pmod{11},           \\
			2^7    & \equiv 7 \pmod{11},           \\
			2^8    & \equiv 3 \pmod{11},           \\
			2^9    & \equiv 6 \pmod{11},           \\
			2^{10} & \equiv 1 \pmod{11}.
		\end{aligned}
	\]

	The powers of $2$ modulo $11$ give all nonzero residues $\{1, 2, 3, 4, 5, 6, 7, 8, 9, 10\}$.
	Since the smallest exponent $k$ such that $2^k \equiv 1$ is $k = 10$, we have
	\[
		\operatorname{ord}_{11}(2) = 10 = 11 - 1.
	\]
	Therefore, $2$ is a primitive root modulo $11$.

	\paragraph{(iii)}

	If we look at  $a = 3$.

	\[
		3^1 \equiv 3, \quad
		3^2 \equiv 9, \quad
		3^3 \equiv 27 \equiv 5, \quad
		3^4 \equiv 15 \equiv 4, \quad
		3^5 \equiv 12 \equiv 1 \pmod{11}.
	\]

	Here the smallest exponent giving $1$ is $k = 5$, so
	\[
		\operatorname{ord}_{11}(3) = 5 < 10.
	\]
	Thus $3$ does not generate all residues modulo $11$, and it is \emph{not} a primitive root.
\end{homeworkProblem}

\begin{homeworkProblem}

	By assumption, $b$ is an inverse of $a$ modulo $p$.
	That means
	\[
		a b \equiv 1 \pmod{p}.
	\]

	\medskip

	\textbf{($\Rightarrow$)} Suppose $a x \equiv c \pmod{p}$.

	Multiply both sides by $b$:
	\[
		b (a x) \equiv b c \pmod{p}.
	\]

	Since $b a \equiv 1 \pmod{p}$, we get
	\[
		x \equiv b c \pmod{p}.
	\]

	\medskip

	\textbf{($\Leftarrow$)} Conversely, suppose $x \equiv b c \pmod{p}$.

	Multiply both sides by $a$:
	\[
		a x \equiv a (b c) \pmod{p}.
	\]

	Because $a b \equiv 1 \pmod{p}$, we have
	\[
		a x \equiv c \pmod{p}.
	\]

	\medskip

	Thus the two congruences are equivalent:
	\[
		a x \equiv c \pmod{p}
		\quad \Longleftrightarrow \quad
		x \equiv b c \pmod{p}.
	\]
\end{homeworkProblem}

\begin{homeworkProblem}
	\paragraph{(i)}
	If $x \equiv c \pmod{mn}$, then $mn \mid (x - c)$.
	Since $m \mid mn$ and $n \mid mn$, it follows that
	\[
		x \equiv c \pmod{m} \quad \text{and} \quad x \equiv c \pmod{n}.
	\]
	Hence we may take $a = c$ and $b = c$.


	\paragraph{(ii)}
	\[
		x \equiv a \pmod{m}
		\quad \text{and} \quad
		x \equiv b \pmod{n}.
	\]
	Because $\gcd(m,n)=1$, there exist integers $u$ and $v$ satisfying the Bézout identity
	\[
		m u + n v = 1.
	\]
	Multiplying both sides by $(b - a)$ gives
	\[
		m u (b - a) + n v (b - a) = b - a.
	\]
	Now define
	\[
		x = a + m u (b - a).
	\]
	Then
	\[
		x \equiv a \pmod{m}
		\quad \text{(since $m \mid m u (b - a)$),}
	\]
	and
	\[
		x \equiv a + (1 - n v)(b - a) \equiv b \pmod{n}.
	\]
	Finally we get,
	\[
		x \equiv a + m u (b - a) \pmod{mn}
	\]
	is a simultaneous solution of the two given congruences.
\end{homeworkProblem}

\begin{homeworkProblem}
	\textbf{($\Rightarrow$)}
	Assume $a^x \equiv a^y \pmod{p}$.
	Multiplying both sides by the inverse of $a^y$ (which exists since $(a,p)=1$), we obtain
	\[
		a^{x - y} \equiv 1 \pmod{p}.
	\]
	By the definition of order, $k \mid (x - y)$, hence
	\[
		x \equiv y \pmod{k}.
	\]

	\medskip

	\textbf{($\Leftarrow$)}
	Conversely, assume $x \equiv y \pmod{k}$, so that $x = y + qk$ for some integer $q$.
	Then
	\[
		a^x = a^{y + qk} = a^y (a^k)^q \equiv a^y (1)^q \equiv a^y \pmod{p}.
	\]
	\medskip

	Thus,
	\[
		a^x \equiv a^y \pmod{p}
		\quad \Longleftrightarrow \quad
		x \equiv y \pmod{\operatorname{ord}_p(a)}.
	\]
\end{homeworkProblem}

\begin{homeworkProblem}

	We want to solve $109^{202} \bmod 100$, which gives us the last two digits of $109^{202}$.

	\medskip

	Since $109 \equiv 9 \pmod{100}$, we have
	\[
		109^{202} \equiv 9^{202} \pmod{100}.
	\]

	Because $\gcd(9,100)=1$, we can use Euler’s theorem:
	\[
		9^{\varphi(100)} \equiv 1 \pmod{100}.
	\]

	Using the given factorization $\varphi(100) = \varphi(4)\varphi(25)$, we compute
	\[
		\varphi(4) = 2, \quad \varphi(25) = 20, \quad \text{so} \quad \varphi(100) = 40.
	\]

	Thus
	\[
		9^{40} \equiv 1 \pmod{100}.
	\]

	We reduce the exponent modulo $40$:
	\[
		202 \equiv 2 \pmod{40}.
	\]
	Hence
	\[
		9^{202} \equiv 9^2 \equiv 81 \pmod{100}.
	\]

	Therefore, the last two digits of $109^{202}$ are 81.

\end{homeworkProblem}

\end{document}
