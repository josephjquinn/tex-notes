

\documentclass{article}

\usepackage{fancyhdr}
\usepackage{extramarks}
\usepackage{amsmath}
\usepackage{amsthm}
\usepackage{amsfonts}
\usepackage{tikz}
\usepackage[plain]{algorithm}
\usepackage{algpseudocode}
\usepackage{amsmath,amssymb}

\usetikzlibrary{automata,positioning}

%
% Basic Document Settings
%

\topmargin=-0.45in
\evensidemargin=0in
\oddsidemargin=0in
\textwidth=6.5in
\textheight=9.0in
\headsep=0.25in

\linespread{1.1}

\pagestyle{fancy}
\lhead{\hmwkAuthorName}
\chead{\hmwkClass: \hmwkTitle}
\rhead{}
\lfoot{\lastxmark}
\cfoot{\thepage}

\renewcommand\headrulewidth{0.4pt}
\renewcommand\footrulewidth{0.4pt}

\setlength\parindent{0pt}

%
% Create Problem Sections
%

\newcommand{\enterProblemHeader}[1]{
    \nobreak\extramarks{}{Problem \arabic{#1} continued on next page\ldots}\nobreak{}
    \nobreak\extramarks{Problem \arabic{#1} (continued)}{Problem \arabic{#1} continued on next page\ldots}\nobreak{}
}

\newcommand{\exitProblemHeader}[1]{
    \nobreak\extramarks{Problem \arabic{#1} (continued)}{Problem \arabic{#1} continued on next page\ldots}\nobreak{}
    \stepcounter{#1}
    \nobreak\extramarks{Problem \arabic{#1}}{}\nobreak{}
}

\setcounter{secnumdepth}{0}
\newcounter{partCounter}
\newcounter{homeworkProblemCounter}
\setcounter{homeworkProblemCounter}{1}
\nobreak\extramarks{Problem \arabic{homeworkProblemCounter}}{}\nobreak{}



% Define theorem environment
\newtheorem*{theorem}{Theorem}

%
% Homework Problem Environment
%
% This environment takes an optional argument. When given, it will adjust the
% problem counter. This is useful for when the problems given for your
% assignment aren't sequential. See the last 3 problems of this template for an
% example.
%
\newenvironment{homeworkProblem}[1][-1]{
    \ifnum#1>0
        \setcounter{homeworkProblemCounter}{#1}
    \fi
    \section{Problem \arabic{homeworkProblemCounter}}
    \setcounter{partCounter}{1}
    \enterProblemHeader{homeworkProblemCounter}
}{
    \exitProblemHeader{homeworkProblemCounter}
}

%
% Homework Details
%   - Title
%   - Due date
%   - Class
%   - Section/Time
%   - Instructor
%   - Author
%


\newcommand{\hmwkTitle}{Problem set\ 6}
\newcommand{\hmwkDueDate}{October 13, 2025}
\newcommand{\hmwkClass}{Number Theory}
\newcommand{\hmwkClassTime}{Section 2}
\newcommand{\hmwkClassInstructor}{Dr. Eleanor McSpirit}
\newcommand{\hmwkAuthorName}{\textbf{Joseph Quinn}}

%
% Title Page
%

\title{
    \vspace{2in}
    \textmd{\textbf{\hmwkClass:\ \hmwkTitle}}\\
    \normalsize\vspace{0.1in}\small{\hmwkDueDate}\\
    \vspace{0.1in}\large{\textit{\hmwkClassInstructor\ \hmwkClassTime}}
    \vspace{3in}
}

\author{\hmwkAuthorName}
\date{}

\renewcommand{\part}[1]{\textbf{\large Part \Alph{partCounter}}\stepcounter{partCounter}\\}

%
% Various Helper Commands
%

% Useful for algorithms
\newcommand{\alg}[1]{\textsc{\bfseries \footnotesize #1}}

% For derivatives
\newcommand{\deriv}[1]{\frac{\mathrm{d}}{\mathrm{d}x} (#1)}

% For partial derivatives
\newcommand{\pderiv}[2]{\frac{\partial}{\partial #1} (#2)}

% Integral dx
\newcommand{\dx}{\mathrm{d}x}

% Alias for the Solution section header
\newcommand{\solution}{\textbf{\large Solution}}

% Probability commands: Expectation, Variance, Covariance, Bias
\newcommand{\E}{\mathrm{E}}
\newcommand{\Var}{\mathrm{Var}}
\newcommand{\Cov}{\mathrm{Cov}}
\newcommand{\Bias}{\mathrm{Bias}}

%  proof-step macro:
\newcommand{\step}[2]{& #1 & & \text{#2} \\}

\begin{document}

\maketitle
\pagebreak

\begin{homeworkProblem}
	\paragraph*{(i)}

We see a part in the mods below:
\begin{align*}
12^1 &\equiv 12 \mod 15 \\
12^2 &= 144 \equiv 144 - 135 = 9 \mod 15 \\
12^3 &= 12 \cdot 9 = 108 \equiv 108 - 105 = 3 \mod 15 \\
12^4 &= 12 \cdot 3 = 36 \equiv 6 \mod 15 \\
12^5 &= 12 \cdot 6 = 72 \equiv 12 \mod 15
\end{align*}

The powers repeat every 4 terms:
\[
12,\ 9,\ 3,\ 6,\ 12,\ 9,\ \dots, \text{Cycle length is } 4.
\]

\[
49 \mod 4 = 1 \Rightarrow 12^{49} \equiv 12^1 \equiv 12 \mod 15
\]

\paragraph*{(ii)}

Since \( \gcd(13, 27) = 1 \), we can use Euler’s theorem. Since
\[
27 = 3^3 \Rightarrow \phi(27) = 27 - 9 = 18,
\]
we have:
\[
13^{18} \equiv 1 \mod 27
\]

Now we want to reduce the exponent modulo 48
\[
9112 \mod 18 = 4 \Rightarrow 13^{9112} \equiv 13^4 \mod 27
\]

And now reduce the full modulo
\begin{align*}
13^2 &= 169 \Rightarrow 169 \mod 27 = 169 - 162 = 7 \\
13^4 &= (13^2)^2 = 7^2 = 49 \Rightarrow 49 \mod 27 = 49 - 27 = 22
\end{align*}

\[
 13^{9112} \equiv 22 \mod 27
\]

\end{homeworkProblem}

\begin{homeworkProblem}
	\textbf{(i)}
	\[
		12^{49} \mod 15
	\]

	We compute the first few powers of 12 modulo 15:

	\begin{align*}
		12^1 & \equiv 12 \mod 15                                         \\
		12^2 & = 144 \equiv 144 \mod 15 = 9                              \\
		12^3 & = 12^2 \cdot 12 = 9 \cdot 12 = 108 \equiv 108 \mod 15 = 3 \\
		12^4 & = 12^3 \cdot 12 = 3 \cdot 12 = 36 \equiv 36 \mod 15 = 6   \\
		12^5 & = 12^4 \cdot 12 = 6 \cdot 12 = 72 \equiv 72 \mod 15 = 12
	\end{align*}

	We see that the mods start to repeat every 4th term

	\[
		12,\ 9,\ 3,\ 6,\ 12,\ 9,\ 3,\ 6,\ \dots
	\]

	Knowing this is a cycle of length 4 we need to see where 49 would fall in this.


	\[
		49 \mod 4 = 1
	\]

	So,
	\[
		12^{49} \equiv 12^1 \mod 15
	\]


	\[
		12^{49} \mod 15 = 12
	\]

	\textbf{(ii)}

	\[
		13^{9112} \bmod 27.
	\]


	Since $\gcd(13, 27) = 1$, Euler's Theorem tells us that
	\[
		13^{\phi(27)} \equiv 1 \pmod{27}.
	\]

	We compute $\phi(27)$ using the formula $\phi(p^k) = p^k - p^{k-1}$:

	\[
		\varphi(27) = \varphi(3^3) = 27 - 9 = 18.
	\]

	Thus,
	\[
		13^{18} \equiv 1 \pmod{27}.
	\]


	Now we want to reduce the exponent $9112$ modulo $18$:

	\[
		9112 \div 18 = 506 \text{ remainder } 4 \quad \Rightarrow \quad 9112 \equiv 4 \pmod{18}.
	\]

	Therefore,
	\[
		13^{9112} \equiv 13^4 \pmod{27}.
	\]

	\[
		13^2 = 169 \equiv 169 - 162 = 7 \pmod{27},
	\]
	\[
		13^4 = (13^2)^2 \equiv 7^2 = 49 \equiv 49 - 27 = 22 \pmod{27}.
	\]

	\[
		13^{9112} \bmod 27 = 22
	\]

\end{homeworkProblem}

\begin{homeworkProblem}

	Let $p$ be a prime number. Show that the natural numbers $1$ and $p - 1$ are their own inverses modulo $p$.

	\medskip

	We want to prove:
	\[
		1 \cdot 1 \equiv 1 \pmod{p}
		\quad \text{and} \quad
		(p - 1)^2 \equiv 1 \pmod{p}
	\]

	\textbf{(i)} For the number $1$:
	\[
		1 \cdot 1 \equiv 1 \pmod{p}.
	\]
	Since $1 \cdot 1 = 1$, and $1 \equiv 1 \pmod{p}$ for any modulus $p$, the result follows immediately. Therefore, $1$ is its own inverse modulo $p$.

	\medskip

	\textbf{(ii)} Now consider $p - 1$. We compute:
	\[
		(p - 1)^2 = p^2 - 2p + 1
	\]

	We know that $p \equiv 0 \pmod{p}$, as $p \mid p \quad $ so:

	\[
		(p - 1)^2 = p^2 - 2p + 1 \equiv 1 \pmod{p}
	\]

	We can rewrite this as:
	\[
		p(p-2) \equiv 0 \pmod{p}
	\]

	And we now see that $p \mid p(p-2)$.

	\medskip

	Therefore, $p - 1$ is its own inverse modulo $p$.

\end{homeworkProblem}

\begin{homeworkProblem}
	\textbf{(i)}

	\begin{itemize}
		\item d=1 $\quad \phi(1) = 1$
		\item d=3 $\quad \phi(3) = 2$
		\item d=5 $\quad \phi(5) = 4$
		\item d=15 $\quad \phi(5) = (3-1)(5-1) = 8$
	\end{itemize}
	\textbf{(ii)}

	\[
		\sum_{d \mid 15} \phi(d) = \phi(1) + \phi(3) + \phi(5) + \phi(15)
		= 1 + 2 + 4 + 8 = 15
	\]
\end{homeworkProblem}

\pagebreak

\begin{homeworkProblem}
	\textbf{(i)}
	Let $p$ be a prime. Then all positive integers $a < p$ satisfy $\gcd(a,p)=1$.\\
	There are exactly $p-1$ such integers. Therefore, $\phi(p)=p-1$.


	\textbf{(ii)}
	There are $pq - 1$ integers in this range. We subtract those divisible by $p$ and $q$:

	\begin{itemize}
		\item There are $q - 1$ multiples of $p$ less than $pq$.
		\item There are $p - 1$ multiples of $q$ less than $pq$.
	\end{itemize}

	So the number of integers not relatively prime to $pq$ is
	\[
		(p - 1) + (q - 1) = p + q - 2.
	\]

	Therefore,
	\[
		\phi(pq) = (pq - 1) - (p + q - 2) = pq - p - q + 1 = (p - 1)(q - 1).
	\]

	\textbf{(iii)}
Let $p$ be a prime and $k \in \mathbb{N}$. By definition, $\phi(p^k)$ is the number of positive integers less than or equal to $p^k$ that are coprime to $p^k$.

An integer $a$ is \emph{not} coprime to $p^k$ if and only if $p \mid a$. So we count how many integers between $1$ and $p^k$ are divisible by $p$.

These are the integers:
\[
p,\ 2p,\ 3p,\ \dots,\ p^k,
\]
which gives a total of $\frac{p^k}{p} = p^{k-1}$ such numbers.

Therefore, the number of integers \emph{coprime} to $p^k$ is:
\[
\phi(p^k) = p^k - p^{k-1}.
\]



\end{homeworkProblem}

\end{document}
