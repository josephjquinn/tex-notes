
\documentclass{article}

\usepackage{fancyhdr}
\usepackage{extramarks}
\usepackage{amsmath}
\usepackage{amsthm}
\usepackage{amsfonts}
\usepackage{tikz}
\usepackage[plain]{algorithm}
\usepackage{algpseudocode}
\usepackage{amsmath,amssymb}

\usetikzlibrary{automata,positioning}

%
% Basic Document Settings
%

\topmargin=-0.45in
\evensidemargin=0in
\oddsidemargin=0in
\textwidth=6.5in
\textheight=9.0in
\headsep=0.25in

\linespread{1.1}

\pagestyle{fancy}
\lhead{\hmwkAuthorName}
\chead{\hmwkClass: \hmwkTitle}
\rhead{}
\lfoot{\lastxmark}
\cfoot{\thepage}

\renewcommand\headrulewidth{0.4pt}
\renewcommand\footrulewidth{0.4pt}

\setlength\parindent{0pt}

%
% Create Problem Sections
%

\newcommand{\enterProblemHeader}[1]{
    \nobreak\extramarks{}{Problem \arabic{#1} continued on next page\ldots}\nobreak{}
    \nobreak\extramarks{Problem \arabic{#1} (continued)}{Problem \arabic{#1} continued on next page\ldots}\nobreak{}
}

\newcommand{\exitProblemHeader}[1]{
    \nobreak\extramarks{Problem \arabic{#1} (continued)}{Problem \arabic{#1} continued on next page\ldots}\nobreak{}
    \stepcounter{#1}
    \nobreak\extramarks{Problem \arabic{#1}}{}\nobreak{}
}

\setcounter{secnumdepth}{0}
\newcounter{partCounter}
\newcounter{homeworkProblemCounter}
\setcounter{homeworkProblemCounter}{1}
\nobreak\extramarks{Problem \arabic{homeworkProblemCounter}}{}\nobreak{}



% Define theorem environment
\newtheorem*{theorem}{Theorem}

%
% Homework Problem Environment
%
% This environment takes an optional argument. When given, it will adjust the
% problem counter. This is useful for when the problems given for your
% assignment aren't sequential. See the last 3 problems of this template for an
% example.
%
\newenvironment{homeworkProblem}[1][-1]{
    \ifnum#1>0
        \setcounter{homeworkProblemCounter}{#1}
    \fi
    \section{Problem \arabic{homeworkProblemCounter}}
    \setcounter{partCounter}{1}
    \enterProblemHeader{homeworkProblemCounter}
}{
    \exitProblemHeader{homeworkProblemCounter}
}

%
% Homework Details
%   - Title
%   - Due date
%   - Class
%   - Section/Time
%   - Instructor
%   - Author
%


\newcommand{\hmwkTitle}{Problem set\ 8}
\newcommand{\hmwkDueDate}{November 1, 2025}
\newcommand{\hmwkClass}{Number Theory}
\newcommand{\hmwkClassTime}{Section 2}
\newcommand{\hmwkClassInstructor}{Dr. Eleanor McSpirit}
\newcommand{\hmwkAuthorName}{\textbf{Joseph Quinn}}

%
% Title Page
%

\title{
    \vspace{2in}
    \textmd{\textbf{\hmwkClass:\ \hmwkTitle}}\\
    \normalsize\vspace{0.1in}\small{\hmwkDueDate}\\
    \vspace{0.1in}\large{\textit{\hmwkClassInstructor\ \hmwkClassTime}}
    \vspace{3in}
}

\author{\hmwkAuthorName}
\date{}

\renewcommand{\part}[1]{\textbf{\large Part \Alph{partCounter}}\stepcounter{partCounter}\\}

%
% Various Helper Commands
%

% Useful for algorithms
\newcommand{\alg}[1]{\textsc{\bfseries \footnotesize #1}}

% For derivatives
\newcommand{\deriv}[1]{\frac{\mathrm{d}}{\mathrm{d}x} (#1)}

% For partial derivatives
\newcommand{\pderiv}[2]{\frac{\partial}{\partial #1} (#2)}

% Integral dx
\newcommand{\dx}{\mathrm{d}x}

% Alias for the Solution section header
\newcommand{\solution}{\textbf{\large Solution}}

% Probability commands: Expectation, Variance, Covariance, Bias
\newcommand{\E}{\mathrm{E}}
\newcommand{\Var}{\mathrm{Var}}
\newcommand{\Cov}{\mathrm{Cov}}
\newcommand{\Bias}{\mathrm{Bias}}

%  proof-step macro:
\newcommand{\step}[2]{& #1 & & \text{#2} \\}

\begin{document}

\maketitle
\pagebreak

\begin{homeworkProblem}
	\begin{proof}
		By definition, $\operatorname{ord}_p(a^k)$ is the smallest positive integer $t$ such that
		\[
			(a^k)^t \equiv 1 \pmod{p}.
		\]
		This means
		\[
			a^{kt} \equiv 1 \pmod{p}.
		\]

		Since $\operatorname{ord}_p(a) = d$, we know that $a^d \equiv 1 \pmod{p}$
		and that $d$ is the smallest such positive integer.
		Hence, for $a^{kt} \equiv 1 \pmod{p}$ to hold, we must have
		\[
			d \mid k t.
		\]

		Let $g = \gcd(d,k)$. Then we can write $d = g d_1$ and $k = g k_1$
		where $\gcd(d_1, k_1) = 1$.
		Substituting into the divisibility condition gives
		\[
			g d_1 \mid g k_1 t \quad \Longrightarrow \quad d_1 \mid k_1 t.
		\]
		Because $\gcd(d_1, k_1) = 1$, it follows that $d_1 \mid t$.
		Thus, the smallest such $t$ is $t = d_1$.

		Therefore,
		\[
			\operatorname{ord}_p(a^k) = d_1 = \frac{d}{\gcd(d,k)}.
		\]
	\end{proof}
\end{homeworkProblem}
\begin{homeworkProblem}
	\textbf{(i)}
	Suppose $m,n \in \mathbb{N}$ with $m \mid n$. Write $n = m k$.
	If $(m,k) = 1$, then by the multiplicativity of Euler's $\phi$-function,
	\[
		\phi(n) = \phi(mk) = \phi(m)\phi(k),
	\]
	so $\phi(m) \mid \phi(n)$.

	If $(m,k) > 1$, write the prime-power decompositions
	\[
		m = p_1^{a_1}\cdots p_r^{a_r}, \qquad
		n = p_1^{b_1}\cdots p_r^{b_r}, \qquad a_i \le b_i.
	\]
	By Lemma~6.12 and the multiplicativity of $\phi$, we have
	$\phi(p_i^{a_i}) \mid \phi(p_i^{b_i})$ for each $i$, hence
	$\phi(m) \mid \phi(n)$.

	\medskip
	\textbf{(ii)}\quad
	The converse is false. For instance,
	\[
		\phi(3) = 2, \quad \phi(8) = 4,
	\]
	so $\varphi(3) \mid \phi(8)$, but $3 \nmid 8$.
\end{homeworkProblem}

\pagebreak

\begin{homeworkProblem}
	\textbf{4.}
	\[
		\begin{aligned}
			\phi(45)
			 & = \phi(3^2 \cdot 5) \\
			 & = \phi(3^2)\phi(5)  \\
			 & = (3^2 - 3)(5 - 1)  \\
			 & = 6 \times 4 = 24.
		\end{aligned}
	\]

	\textbf{5.}
	\[
		\begin{aligned}
			\quad \phi(98)
			 & = \phi(2 \cdot 7^2) \\
			 & = \phi(2)\phi(7^2)  \\
			 & = (2 - 1)(7^2 - 7)  \\
			 & = 1 \times 42 = 42.
		\end{aligned}
	\]

	\textbf{6.}
	\[
		\begin{aligned}
			\phi(5^6 11^4 17^{10})
			 & = \phi(5^6)\phi(11^4)\phi(17^{10})          \\
			 & = (5^6 - 5^5)(11^4 - 11^3)(17^{10} - 17^9)  \\
			 & = 12500 \times 13310 \times 17^9 \times 16.
		\end{aligned}
	\]
\end{homeworkProblem}
\begin{homeworkProblem}
	\begin{proof}


		Since $p$ is prime, the multiplicative group $(\mathbb{Z}/p\mathbb{Z})^\times$ is cyclic of order $p-1$.
		Let $g$ be a primitive root modulo $p$. Then every nonzero residue can be written as
		\[
			x \equiv g^t \pmod{p}
		\]
		for some integer $t$.

		Substituting into the given congruence gives
		\[
			(g^t)^k \equiv 1 \pmod{p} \quad \Rightarrow \quad g^{tk} \equiv 1 \pmod{p}.
		\]
		Since $g$ has order $p-1$, we must have
		\[
			p-1 \mid tk.
		\]
		This means
		\[
			tk \equiv 0 \pmod{p-1}.
		\]

		The number of distinct solutions $t$ modulo $p-1$ to this linear congruence is $\gcd(k, p-1)$.
		Each such $t$ yields a distinct $x = g^t \pmod{p}$.

		Hence, the congruence $x^k \equiv 1 \pmod{p}$ has exactly $\gcd(k, p-1)$ incongruent solutions.
	\end{proof}
\end{homeworkProblem}
\begin{homeworkProblem}
	For a prime $p$, the number of primitive roots modulo $p$ is $\phi(p - 1)$.

	\[
		p - 1 = 251 - 1 = 250 = 2 \times 5^3.
	\]
	\[
		\phi(250) = \phi(2)\phi(5^3) = (2 - 1)(5^3 - 5^2) = 1 \times 100 = 100.
	\]

	There are 100 primitive roots modulo 251.
\end{homeworkProblem}

\begin{homeworkProblem}

	We want to solve $x^{49} \equiv 100 \pmod{151}$.

	Since $151$ is prime, $\phi(151) = 150$. We first find the inverse of $49$ modulo $150$.

	\[
		150 = 3(49) + 3, \quad 49 = 16(3) + 1.
	\]
	Back-substituting:
	\[
		1 = 49 - 16(3) = 49 - 16(150 - 3(49)) = 49(49) - 16(150).
	\]
	Hence $49^{-1} \equiv 49 \pmod{150}$.

	Thus
	\[
		x \equiv 100^{49} \pmod{151}.
	\]

	Compute by repeated squaring:
	\[
		100^2 \equiv 34,\quad 100^4 \equiv 99,\quad 100^8 \equiv 137,\quad 100^{16} \equiv 45,\quad 100^{32} \equiv 62.
	\]
	Then
	\[
		100^{49} \equiv 100^{32+16+1} \equiv 62 \cdot 45 \cdot 100 \equiv 103 \pmod{151}.
	\]

	\[
		x \equiv 103 \pmod{151}.
	\]

\end{homeworkProblem}
\begin{homeworkProblem}

	\begin{proof}
		Assume $b \not\equiv 0 \pmod p$.
		Since $(\mathbb{Z}/p\mathbb{Z})^\times$ is cyclic of order $p-1$,
		let $g$ be a primitive root modulo $p$.
		Then every nonzero residue can be written as $x \equiv g^r$ and $b \equiv g^t$ for some integers $r,t$.

		The congruence $x^k \equiv b \pmod p$ becomes
		\[
			(g^r)^k \equiv g^t \pmod p \quad \Rightarrow \quad g^{kr} \equiv g^t \pmod p.
		\]
		Hence
		\[
			kr \equiv t \pmod{p-1}.
		\]
		This linear congruence in $r$ has a solution if and only if $\gcd(k,p-1)\mid t$,
		and in that case has exactly $\gcd(k,p-1)$ incongruent solutions modulo $p-1$.
		Each such $r$ yields a distinct $x = g^r \pmod p$.

		Therefore, the number of $k$th roots of $b$ modulo $p$ is $0$ or $\gcd(k,p-1)$,
		depending on whether $\gcd(k,p-1)$ divides the exponent $t$.
		If $b \equiv 0 \pmod p$, then $x^k \equiv 0 \pmod p$ has exactly one solution $x \equiv 0$.
	\end{proof}

\end{homeworkProblem}
\end{document}
