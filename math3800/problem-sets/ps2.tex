\documentclass{article}

\usepackage{fancyhdr}
\usepackage{extramarks}
\usepackage{amsmath}
\usepackage{amsthm}
\usepackage{amsfonts}
\usepackage{tikz}
\usepackage[plain]{algorithm}
\usepackage{algpseudocode}
\usepackage{amsmath,amssymb}

\usetikzlibrary{automata,positioning}

%
% Basic Document Settings
%

\topmargin=-0.45in
\evensidemargin=0in
\oddsidemargin=0in
\textwidth=6.5in
\textheight=9.0in
\headsep=0.25in

\linespread{1.1}

\pagestyle{fancy}
\lhead{\hmwkAuthorName}
\chead{\hmwkClass: \hmwkTitle}
\rhead{}
\lfoot{\lastxmark}
\cfoot{\thepage}

\renewcommand\headrulewidth{0.4pt}
\renewcommand\footrulewidth{0.4pt}

\setlength\parindent{0pt}

%
% Create Problem Sections
%

\newcommand{\enterProblemHeader}[1]{
    \nobreak\extramarks{}{Problem \arabic{#1} continued on next page\ldots}\nobreak{}
    \nobreak\extramarks{Problem \arabic{#1} (continued)}{Problem \arabic{#1} continued on next page\ldots}\nobreak{}
}

\newcommand{\exitProblemHeader}[1]{
    \nobreak\extramarks{Problem \arabic{#1} (continued)}{Problem \arabic{#1} continued on next page\ldots}\nobreak{}
    \stepcounter{#1}
    \nobreak\extramarks{Problem \arabic{#1}}{}\nobreak{}
}

\setcounter{secnumdepth}{0}
\newcounter{partCounter}
\newcounter{homeworkProblemCounter}
\setcounter{homeworkProblemCounter}{1}
\nobreak\extramarks{Problem \arabic{homeworkProblemCounter}}{}\nobreak{}



% Define theorem environment
\newtheorem*{theorem}{Theorem}

%
% Homework Problem Environment
%
% This environment takes an optional argument. When given, it will adjust the
% problem counter. This is useful for when the problems given for your
% assignment aren't sequential. See the last 3 problems of this template for an
% example.
%
\newenvironment{homeworkProblem}[1][-1]{
    \ifnum#1>0
        \setcounter{homeworkProblemCounter}{#1}
    \fi
    \section{Problem \arabic{homeworkProblemCounter}}
    \setcounter{partCounter}{1}
    \enterProblemHeader{homeworkProblemCounter}
}{
    \exitProblemHeader{homeworkProblemCounter}
}

%
% Homework Details
%   - Title
%   - Due date
%   - Class
%   - Section/Time
%   - Instructor
%   - Author
%


\newcommand{\hmwkTitle}{Problem set\ 2}
\newcommand{\hmwkDueDate}{September 1, 2025}
\newcommand{\hmwkClass}{Number Theory}
\newcommand{\hmwkClassTime}{Section 2}
\newcommand{\hmwkClassInstructor}{Dr. Eleanor McSpirit}
\newcommand{\hmwkAuthorName}{\textbf{Joseph Quinn}}

%
% Title Page
%

\title{
    \vspace{2in}
    \textmd{\textbf{\hmwkClass:\ \hmwkTitle}}\\
    \normalsize\vspace{0.1in}\small{\hmwkDueDate}\\
    \vspace{0.1in}\large{\textit{\hmwkClassInstructor\ \hmwkClassTime}}
    \vspace{3in}
}

\author{\hmwkAuthorName}
\date{}

\renewcommand{\part}[1]{\textbf{\large Part \Alph{partCounter}}\stepcounter{partCounter}\\}

%
% Various Helper Commands
%

% Useful for algorithms
\newcommand{\alg}[1]{\textsc{\bfseries \footnotesize #1}}

% For derivatives
\newcommand{\deriv}[1]{\frac{\mathrm{d}}{\mathrm{d}x} (#1)}

% For partial derivatives
\newcommand{\pderiv}[2]{\frac{\partial}{\partial #1} (#2)}

% Integral dx
\newcommand{\dx}{\mathrm{d}x}

% Alias for the Solution section header
\newcommand{\solution}{\textbf{\large Solution}}

% Probability commands: Expectation, Variance, Covariance, Bias
\newcommand{\E}{\mathrm{E}}
\newcommand{\Var}{\mathrm{Var}}
\newcommand{\Cov}{\mathrm{Cov}}
\newcommand{\Bias}{\mathrm{Bias}}

%  proof-step macro:
\newcommand{\step}[2]{& #1 & & \text{#2} \\}

\begin{document}

\maketitle

\pagebreak
\begin{homeworkProblem}

	\textbf{(i) a = 343, b = 49}

	To compute $(343, 49)$ using the Euclidean algorithm. We first write
	\[
		343 = 7 \cdot 49 + 0.
	\]
	By Theorem 1.33, we know that $(343, 49) = (49, 0)$. We know that $(49, 0) = 49$, since $49$ is the largest integer which divides itself and $49 \mid 0$ too. Therefore, $(343, 49) = 49$.

	\medskip

	\textbf{(ii) a = -469, b = 31}

	Let's compute $(-469, 31)$ using the Euclidean algorithm. We first write
	\[
		-469 = (-16) \cdot 31 + 27.
	\]
	By Theorem 1.33, we know that $(-469, 31) = (31, 27)$. We then have
	\[
		31 = 1 \cdot 27 + 4.
	\]
	By Theorem 1.33, we know that $(31, 27) = (27, 4)$. We then have
	\[
		27 = 6 \cdot 4 + 3.
	\]
	By Theorem 1.33, we know that $(27, 4) = (4, 3)$. We then have
	\[
		4 = 1 \cdot 3 + 1.
	\]
	By Theorem 1.33, we know that $(4, 3) = (3, 1)$. We then have
	\[
		3 = 3 \cdot 1 + 0.
	\]
	By Theorem 1.33, we know that $(3, 1) = (1, 0)$. We know that $(1, 0) = 1$, since $1$ is the largest integer which divides itself and $1 \mid 0$ too. Since $(-469, 31) = (31, 27) = (27, 4) = (4, 3) = (3, 1) = (1, 0) = 1$, we have computed $(-469, 31) = 1$.
\end{homeworkProblem}

\begin{homeworkProblem}
	\begin{proof}
		Suppose $p(x), g(x)$ are two polynomials with $g(x) \neq 0$.
		Assume there are two possible divisions:
		\[
			p(x) = q(x)g(x) + r(x), \quad \deg r(x) < \deg g(x),
		\]
		and
		\[
			p(x) = q'(x)g(x) + r'(x), \quad \deg r'(x) < \deg g(x).
		\]

		Subtracting gives
		\[
			0 = \big(q(x) - q'(x)\big) g(x) + \big(r(x) - r'(x)\big).
		\]

		If $q(x) \neq q'(x)$, then $(q(x)-q'(x))g(x)$ is a nonzero polynomial.
		Hence
		\[
			\deg\big((q(x)-q'(x))g(x)\big) = \deg(q(x)-q'(x)) + \deg g(x) \;\geq\; \deg g(x).
		\]
		But $\deg(r(x)-r'(x)) < \deg g(x)$ since both $r$ and $r'$ have degree $< \deg g(x)$.
		Thus the polynomial
		\[
			(q(x)-q'(x))g(x) + (r(x)-r'(x))
		\]
		would contain a term of degree at least $\deg g(x)$, which cannot be canceled by the remainder term of smaller degree.
		This contradiction shows $q(x) = q'(x)$.

		With $q(x) = q'(x)$, the equation reduces to
		\[
			0 = r(x) - r'(x),
		\]
		so $r(x) = r'(x)$.

		Therefore, the quotient and remainder in the division algorithm for polynomials are unique.
	\end{proof}
\end{homeworkProblem}
\pagebreak

\begin{homeworkProblem}
	\begin{proof}
		\textbf{($\Rightarrow$)} Assume $a \equiv b \pmod n$.
		By definition, $n \mid (a-b)$, so there exists $k \in \mathbb{Z}$ such that
		$a-b = nk$.
		Write $a = nq_1 + r_1$ and $b = nq_2 + r_2$ with $0 \le r_1, r_2 \le n-1$. Then

		\begin{align*}
			a - b
			 & = (nq_1 + r_1) - (nq_2 + r_2) \\
			 & = n(q_1 - q_2) + (r_1 - r_2).
		\end{align*}


		Since $n \mid (a-b)$, it follows that $n \mid (r_1 - r_2)$. But $r_1$ and $r_2$ are both between $0$ and $n-1$, so
		The only multiple of $n$ in this range is $0$. Hence $r_1 - r_2 = 0$, so $r_1 = r_2$.


		\medskip
		\textbf{($\Leftarrow$)} Assume $r_1 = r_2$.
		Then
		\begin{align*}
			a - b
			 & = (nq_1 + r_1) - (nq_2 + r_2) \\
			 & = n(q_1 - q_2) + (r_1 - r_2)  \\
			 & = n(q_1 - q_2)
		\end{align*}


		Thus $n \mid (a-b)$, which means $a \equiv b \pmod n$.

	\end{proof}
\end{homeworkProblem}

\begin{homeworkProblem}

	We want to describe the integer solutions to the congruence. $x \equiv 2 \pmod{5}$.
	\medskip

	We can rewrite the congruence as
	\[
		5 \mid (2-x)
	\]
	And by propery of divisibility


	\[
		5k = (2-x), k \in \mathbb{Z}
	\]
	\[
		x = 5k + 2
	\]
	With this we can see some examples of the solution set would be:

	\[
		\begin{aligned}
			k & = 0 \quad \Rightarrow \quad x = 2,   \\
			k & = 1 \quad \Rightarrow \quad x = 7,   \\
			k & = -1 \quad \Rightarrow \quad x = -3.
		\end{aligned}
	\]

	Thus, the solution set is the infinite arithmetic progression
	\[
		\{\ldots, -8, -3, 2, 7, 12, 17, \ldots\}.
	\]

\end{homeworkProblem}

\pagebreak
\begin{homeworkProblem}


	(i) Compute $\gcd(37,60)$.

	\[
		60 = 1\cdot 37 + 23
	\]
	\[
		37 = 1\cdot 23 + 14
	\]
	\[
		23 = 1\cdot 14 + 9
	\]
	\[
		14 = 1\cdot 9 + 5
	\]
	\[
		9 = 1\cdot 5 + 4
	\]
	\[
		5 = 1\cdot 4 + 1
	\]
	\[
		4 = 4\cdot 1 + 0
	\]
	So $\gcd(37,60)=1$.

	Linear combination
	\[
		1 = 5 - 1\cdot 4
	\]
	\[
		= 5 - (9 - 5) = 2\cdot 5 - 9
	\]
	\[
		= 2(14 - 9) - 9 = 2\cdot 14 - 3\cdot 9
	\]
	\[
		= 2\cdot 14 - 3(23 - 14) = 5\cdot 14 - 3\cdot 23
	\]
	\[
		= 5(37 - 23) - 3\cdot 23 = 5\cdot 37 - 8\cdot 23
	\]
	\[
		= 5\cdot 37 - 8(60 - 37) = 13\cdot 37 - 8\cdot 60
	\]

	Thus,

	\begin{align*}
		\gcd(37,60) & = 1                              \\
		            & = 37 \cdot (13) + 60 \cdot (-8).
	\end{align*}

	\[
		x = 13, \quad y = -8 \quad \text{such that } 37x + 60y = 1.
	\]

	(ii) Compute $\gcd(441,1155)$.

	\[
		1155 = 2\cdot 441 + 273
	\]
	\[
		441 = 1\cdot 273 + 168
	\]
	\[
		273 = 1\cdot 168 + 105
	\]
	\[
		168 = 1\cdot 105 + 63
	\]
	\[
		105 = 1\cdot 63 + 42
	\]
	\[
		63 = 1\cdot 42 + 21
	\]
	\[
		42 = 2\cdot 21 + 0
	\]
	So $\gcd(441,1155)=21$.

	Linear combination
	\[
		21 = 63 - 42
	\]
	\[
		= 63 - (105 - 63) = 2\cdot 63 - 105
	\]
	\[
		= 2(168 - 105) - 105 = 2\cdot 168 - 3\cdot 105
	\]
	\[
		= 2\cdot 168 - 3(273 - 168) = 5\cdot 168 - 3\cdot 273
	\]
	\[
		= 5(441 - 273) - 3\cdot 273 = 5\cdot 441 - 8\cdot 273
	\]
	\[
		= 5\cdot 441 - 8(1155 - 2\cdot 441) = 21\cdot 441 - 8\cdot 1155
	\]

	Thus,
	\begin{align*}
		\gcd(441,1155) & = 21                                \\
		               & = 21 \cdot 441 - 8 \cdot 1155       \\
		               & = 441 \cdot (21) + 1155 \cdot (-8),
	\end{align*}

	\[
		x = 21, \quad y = -8 \quad \text{such that } 441x + 1155y = 21.
	\]
\end{homeworkProblem}

\begin{homeworkProblem}

	\textbf{Exercise 1.56}

	The least common multiple between two integers is going to be the smallest integer mulitple that can be created with said integers.
	Denoted by lcm(a,b).

	We can also think of this as the smallest integer that can be divided by both numbers.

	\[
		\operatorname{lcm}(a,b) = \min \{\, n \in \mathbb{N} : a \mid n \ \text{and}\ b \mid n \,\}.
	\]

	Examples:
	\[
		\begin{aligned}
			\operatorname{lcm}(6,8)   & = 24, \\
			\operatorname{lcm}(21,6)  & = 42, \\
			\operatorname{lcm}(14,35) & = 70, \\
		\end{aligned}
	\]

	\begin{theorem}
		If a and b are natural numbers, then gcd(a,b) $\cdot$ lcm(a,b) = ab.
	\end{theorem}

	\begin{proof}
		Let $d = \gcd(a,b)$. Then we can write
		\[
			a = d \cdot a_1, \qquad b = d \cdot b_1
		\]
		with $\gcd(a_1,b_1)=1$.

		The least common multiple of $a$ and $b$ is then
		\[
			\operatorname{lcm}(a,b) = d \cdot a_1 b_1,
		\]
		because $a_1$ and $b_1$ are coprime, so their product is the smallest common multiple.

		Hence,
		\[
			\gcd(a,b)\cdot \operatorname{lcm}(a,b)
			= d \cdot (d \cdot a_1 b_1)
			= d^2 a_1 b_1.
		\]
		But
		\[
			ab = (d a_1)(d b_1) = d^2 a_1 b_1.
		\]
		Therefore,
		\[
			\gcd(a,b)\cdot \operatorname{lcm}(a,b) = ab.
		\]
	\end{proof}


\end{homeworkProblem}

\pagebreak
\begin{homeworkProblem}

	Show that $41$ divides $2^{20}-1$ by following these steps. Explain why each step is true.

	\begin{enumerate}
		\item

		      By the definition of congruence:
		      \begin{align*}
			      2^5 \equiv (-9)\pmod{41} \\
			      41 \mid 2^5 - (-9)       \\
			      41 \mid 41
		      \end{align*}

		\item Raising both sides to the 4th power gives
		      \[
			      (2^5)^4 \equiv (-9)^4 \pmod{41}.
		      \]
		      By Theorem 1.18 this is valid.
		      So
		      \[
			      2^{20} \equiv 9^4 \pmod{41}.
		      \]
		\item Note that:
		      \[
			      9^2 = 81.
		      \]
		      \[
			      81 + 1 = 82 = 2 \cdot 41,
		      \]
		      so
		      \[
			      41 \mid (9^2 + 1) \quad \iff \quad 9^2 \equiv -1 \pmod{41}.
		      \]

		      Again by Theorem 1.18 we can apply exponents,
		      \[
			      (9^2)^2 \equiv (-1)^2 \pmod{41},
		      \]
		      which means
		      \[
			      9^4 \equiv 1 \pmod{41}.
		      \]
		\item Substituting into step (2),
		      \[
			      2^{20} \equiv 9^4 \equiv 1 \pmod{41}.
		      \]
		      Thus
		      \[
			      2^{20} \equiv 1 \pmod{41} \\
		      \]
		      And by Theorem 1.12:
		      \[
			      2^{20}- 1 \equiv 0 \pmod{41} \\
		      \]
		      Leading back to our initial equation :
		      \[
			      41 \mid (2^{20}-1).
		      \]
	\end{enumerate}

\end{homeworkProblem}

\end{document}
