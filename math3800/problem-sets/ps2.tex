\documentclass{article}

\usepackage{fancyhdr}
\usepackage{extramarks}
\usepackage{amsmath}
\usepackage{amsthm}
\usepackage{amsfonts}
\usepackage{tikz}
\usepackage[plain]{algorithm}
\usepackage{algpseudocode}
\usepackage{amsmath,amssymb}

\usetikzlibrary{automata,positioning}

%
% Basic Document Settings
%

\topmargin=-0.45in
\evensidemargin=0in
\oddsidemargin=0in
\textwidth=6.5in
\textheight=9.0in
\headsep=0.25in

\linespread{1.1}

\pagestyle{fancy}
\lhead{\hmwkAuthorName}
\chead{\hmwkClass: \hmwkTitle}
\rhead{}
\lfoot{\lastxmark}
\cfoot{\thepage}

\renewcommand\headrulewidth{0.4pt}
\renewcommand\footrulewidth{0.4pt}

\setlength\parindent{0pt}

%
% Create Problem Sections
%

\newcommand{\enterProblemHeader}[1]{
    \nobreak\extramarks{}{Problem \arabic{#1} continued on next page\ldots}\nobreak{}
    \nobreak\extramarks{Problem \arabic{#1} (continued)}{Problem \arabic{#1} continued on next page\ldots}\nobreak{}
}

\newcommand{\exitProblemHeader}[1]{
    \nobreak\extramarks{Problem \arabic{#1} (continued)}{Problem \arabic{#1} continued on next page\ldots}\nobreak{}
    \stepcounter{#1}
    \nobreak\extramarks{Problem \arabic{#1}}{}\nobreak{}
}

\setcounter{secnumdepth}{0}
\newcounter{partCounter}
\newcounter{homeworkProblemCounter}
\setcounter{homeworkProblemCounter}{1}
\nobreak\extramarks{Problem \arabic{homeworkProblemCounter}}{}\nobreak{}



% Define theorem environment
\newtheorem*{theorem}{Theorem}

%
% Homework Problem Environment
%
% This environment takes an optional argument. When given, it will adjust the
% problem counter. This is useful for when the problems given for your
% assignment aren't sequential. See the last 3 problems of this template for an
% example.
%
\newenvironment{homeworkProblem}[1][-1]{
    \ifnum#1>0
        \setcounter{homeworkProblemCounter}{#1}
    \fi
    \section{Problem \arabic{homeworkProblemCounter}}
    \setcounter{partCounter}{1}
    \enterProblemHeader{homeworkProblemCounter}
}{
    \exitProblemHeader{homeworkProblemCounter}
}

%
% Homework Details
%   - Title
%   - Due date
%   - Class
%   - Section/Time
%   - Instructor
%   - Author
%


\newcommand{\hmwkTitle}{Problem set\ 2}
\newcommand{\hmwkDueDate}{September 1, 2025}
\newcommand{\hmwkClass}{Number Theory}
\newcommand{\hmwkClassTime}{Section 2}
\newcommand{\hmwkClassInstructor}{Dr. Eleanor McSpirit}
\newcommand{\hmwkAuthorName}{\textbf{Joseph Quinn}}

%
% Title Page
%

\title{
    \vspace{2in}
    \textmd{\textbf{\hmwkClass:\ \hmwkTitle}}\\
    \normalsize\vspace{0.1in}\small{\hmwkDueDate}\\
    \vspace{0.1in}\large{\textit{\hmwkClassInstructor\ \hmwkClassTime}}
    \vspace{3in}
}

\author{\hmwkAuthorName}
\date{}

\renewcommand{\part}[1]{\textbf{\large Part \Alph{partCounter}}\stepcounter{partCounter}\\}

%
% Various Helper Commands
%

% Useful for algorithms
\newcommand{\alg}[1]{\textsc{\bfseries \footnotesize #1}}

% For derivatives
\newcommand{\deriv}[1]{\frac{\mathrm{d}}{\mathrm{d}x} (#1)}

% For partial derivatives
\newcommand{\pderiv}[2]{\frac{\partial}{\partial #1} (#2)}

% Integral dx
\newcommand{\dx}{\mathrm{d}x}

% Alias for the Solution section header
\newcommand{\solution}{\textbf{\large Solution}}

% Probability commands: Expectation, Variance, Covariance, Bias
\newcommand{\E}{\mathrm{E}}
\newcommand{\Var}{\mathrm{Var}}
\newcommand{\Cov}{\mathrm{Cov}}
\newcommand{\Bias}{\mathrm{Bias}}

%  proof-step macro:
\newcommand{\step}[2]{& #1 & & \text{#2} \\}

\begin{document}

\maketitle

\pagebreak
\begin{homeworkProblem}

	\textbf{(i) a = 343, b = 49}

	To compute $(343, 49)$ using the Euclidean algorithm. We first write
	\[
		343 = 7 \cdot 49 + 0.
	\]
	By Theorem 1.33, we know that $(343, 49) = (49, 0)$. We know that $(49, 0) = 49$, since $49$ is the largest integer which divides itself and $49 \mid 0$ too. Therefore, $(343, 49) = 49$.

	\medskip

	\textbf{(ii) a = -469, b = 31}

	Let's compute $(-469, 31)$ using the Euclidean algorithm. We first write
	\[
		-469 = (-16) \cdot 31 + 27.
	\]
	By Theorem 1.33, we know that $(-469, 31) = (31, 27)$. We then have
	\[
		31 = 1 \cdot 27 + 4.
	\]
	By Theorem 1.33, we know that $(31, 27) = (27, 4)$. We then have
	\[
		27 = 6 \cdot 4 + 3.
	\]
	By Theorem 1.33, we know that $(27, 4) = (4, 3)$. We then have
	\[
		4 = 1 \cdot 3 + 1.
	\]
	By Theorem 1.33, we know that $(4, 3) = (3, 1)$. We then have
	\[
		3 = 3 \cdot 1 + 0.
	\]
	By Theorem 1.33, we know that $(3, 1) = (1, 0)$. We know that $(1, 0) = 1$, since $1$ is the largest integer which divides itself and $1 \mid 0$ too. Since $(-469, 31) = (31, 27) = (27, 4) = (4, 3) = (3, 1) = (1, 0) = 1$, we have computed $(-469, 31) = 1$.
\end{homeworkProblem}

\begin{homeworkProblem}
	prob 2
\end{homeworkProblem}
\pagebreak

\begin{homeworkProblem}
	\begin{proof}
		\textbf{($\Rightarrow$)} Assume $a \equiv b \pmod n$.
		By definition, $n \mid (a-b)$, so there exists $k \in \mathbb{Z}$ such that
		$a-b = nk$.
		Write $a = nq_1 + r_1$ and $b = nq_2 + r_2$ with $0 \le r_1, r_2 \le n-1$. Then

		\begin{align*}
			a - b
			 & = (nq_1 + r_1) - (nq_2 + r_2) \\
			 & = n(q_1 - q_2) + (r_1 - r_2).
		\end{align*}


		Since $n \mid (a-b)$, it follows that $n \mid (r_1 - r_2)$. But $r_1$ and $r_2$ are both between $0$ and $n-1$, so
		The only multiple of $n$ in this range is $0$. Hence $r_1 - r_2 = 0$, so $r_1 = r_2$.


		\medskip
		\textbf{($\Leftarrow$)} Assume $r_1 = r_2$.
		Then
		\begin{align*}
			a - b
			 & = (nq_1 + r_1) - (nq_2 + r_2) \\
			 & = n(q_1 - q_2) + (r_1 - r_2)  \\
			 & = n(q_1 - q_2)
		\end{align*}


		Thus $n \mid (a-b)$, which means $a \equiv b \pmod n$.

	\end{proof}
\end{homeworkProblem}

\begin{homeworkProblem}

	We want to describe the integer solutions to the congruence. $x \equiv 2 \pmod{5}$.
	\medskip

	We can rewrite the congruence as
	\[
		5 \mid (2-x)
	\]
	And by propery of divisibility


	\[
		5k = (2-x), k \in \mathbb{Z}
	\]
	\[
		x = 5k + 2
	\]
	With this we can see some examples of the solution set would be:

	\[
		\begin{aligned}
			k & = 0 \quad \Rightarrow \quad x = 2,   \\
			k & = 1 \quad \Rightarrow \quad x = 7,   \\
			k & = -1 \quad \Rightarrow \quad x = -3.
		\end{aligned}
	\]

	Thus, the solution set is the infinite arithmetic progression
	\[
		\{\ldots, -8, -3, 2, 7, 12, 17, \ldots\}.
	\]

\end{homeworkProblem}


\pagebreak
\begin{homeworkProblem}

	The least common multiple between two integers is going to be the smallest integer mulitple that can be created with said integers.
	Denoted by lcm(a,b).

	\begin{theorem}
		If a and b are natural numbers, then gcd(a,b) $\cdot$ lcm(a,b) = ab.
	\end{theorem}

	\begin{proof}

	\end{proof}


\end{homeworkProblem}

\end{document}
