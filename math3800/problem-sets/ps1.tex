

\documentclass{article}

\usepackage{fancyhdr}
\usepackage{extramarks}
\usepackage{amsmath}
\usepackage{amsthm}
\usepackage{amsfonts}
\usepackage{tikz}
\usepackage[plain]{algorithm}
\usepackage{algpseudocode}
\usepackage{amsmath,amssymb}

\usetikzlibrary{automata,positioning}

%
% Basic Document Settings
%

\topmargin=-0.45in
\evensidemargin=0in
\oddsidemargin=0in
\textwidth=6.5in
\textheight=9.0in
\headsep=0.25in

\linespread{1.1}

\pagestyle{fancy}
\lhead{\hmwkAuthorName}
\chead{\hmwkClass: \hmwkTitle}
\rhead{}
\lfoot{\lastxmark}
\cfoot{\thepage}

\renewcommand\headrulewidth{0.4pt}
\renewcommand\footrulewidth{0.4pt}

\setlength\parindent{0pt}

%
% Create Problem Sections
%

\newcommand{\enterProblemHeader}[1]{
    \nobreak\extramarks{}{Problem \arabic{#1} continued on next page\ldots}\nobreak{}
    \nobreak\extramarks{Problem \arabic{#1} (continued)}{Problem \arabic{#1} continued on next page\ldots}\nobreak{}
}

\newcommand{\exitProblemHeader}[1]{
    \nobreak\extramarks{Problem \arabic{#1} (continued)}{Problem \arabic{#1} continued on next page\ldots}\nobreak{}
    \stepcounter{#1}
    \nobreak\extramarks{Problem \arabic{#1}}{}\nobreak{}
}

\setcounter{secnumdepth}{0}
\newcounter{partCounter}
\newcounter{homeworkProblemCounter}
\setcounter{homeworkProblemCounter}{1}
\nobreak\extramarks{Problem \arabic{homeworkProblemCounter}}{}\nobreak{}



% Define theorem environment
\newtheorem*{theorem}{Theorem}

%
% Homework Problem Environment
%
% This environment takes an optional argument. When given, it will adjust the
% problem counter. This is useful for when the problems given for your
% assignment aren't sequential. See the last 3 problems of this template for an
% example.
%
\newenvironment{homeworkProblem}[1][-1]{
    \ifnum#1>0
        \setcounter{homeworkProblemCounter}{#1}
    \fi
    \section{Problem \arabic{homeworkProblemCounter}}
    \setcounter{partCounter}{1}
    \enterProblemHeader{homeworkProblemCounter}
}{
    \exitProblemHeader{homeworkProblemCounter}
}

%
% Homework Details
%   - Title
%   - Due date
%   - Class
%   - Section/Time
%   - Instructor
%   - Author
%


\newcommand{\hmwkTitle}{Problem set\ 1}
\newcommand{\hmwkDueDate}{August 27, 2025}
\newcommand{\hmwkClass}{Number Theory}
\newcommand{\hmwkClassTime}{Section 2}
\newcommand{\hmwkClassInstructor}{Dr. Eleanor McSpirit}
\newcommand{\hmwkAuthorName}{\textbf{Joseph Quinn}}

%
% Title Page
%

\title{
    \vspace{2in}
    \textmd{\textbf{\hmwkClass:\ \hmwkTitle}}\\
    \normalsize\vspace{0.1in}\small{\hmwkDueDate}\\
    \vspace{0.1in}\large{\textit{\hmwkClassInstructor\ \hmwkClassTime}}
    \vspace{3in}
}

\author{\hmwkAuthorName}
\date{}

\renewcommand{\part}[1]{\textbf{\large Part \Alph{partCounter}}\stepcounter{partCounter}\\}

%
% Various Helper Commands
%

% Useful for algorithms
\newcommand{\alg}[1]{\textsc{\bfseries \footnotesize #1}}

% For derivatives
\newcommand{\deriv}[1]{\frac{\mathrm{d}}{\mathrm{d}x} (#1)}

% For partial derivatives
\newcommand{\pderiv}[2]{\frac{\partial}{\partial #1} (#2)}

% Integral dx
\newcommand{\dx}{\mathrm{d}x}

% Alias for the Solution section header
\newcommand{\solution}{\textbf{\large Solution}}

% Probability commands: Expectation, Variance, Covariance, Bias
\newcommand{\E}{\mathrm{E}}
\newcommand{\Var}{\mathrm{Var}}
\newcommand{\Cov}{\mathrm{Cov}}
\newcommand{\Bias}{\mathrm{Bias}}

%  proof-step macro:
\newcommand{\step}[2]{& #1 & & \text{#2} \\}

\begin{document}

\maketitle

\pagebreak

\begin{homeworkProblem}

	\begin{theorem}
		Let $n$ be an integer. If $14 \mid n$, then $7 \mid n$.
	\end{theorem}

	\medskip

	\textbf{Assumptions:}
	\begin{itemize}
		\item $n$ is an integer.
		\item $14 \mid n$
	\end{itemize}

	\textbf{Conclusion:}
	\begin{itemize}
		\item $7 \mid n$
	\end{itemize}


	The proof uses the definition of divisibility. If $14$ divides $n$, then $n = 14k$ for some integer $k$.
	Since $14 = 7 \cdot 2$, we can write $n = 7(2k)$. The product $2k$ is still an integer, so call it $m$.
	That gives $n = 7m$, which shows $7$ divides $n$.


\end{homeworkProblem}

\begin{homeworkProblem}

	\begin{theorem}
		For every natural number $n$, $1 + 2 + 2^{2} + \cdots + 2^{n} = 2^{\,n+1}-1$.
	\end{theorem}

	\begin{proof}

		\mbox{}\\
		\textbf{Base case:} For $n=1$,

		\[
			1 + 2 = 2^{1+1} - 1 = 3,
		\]

		\medskip

		\textbf{Induction hypothesis:} Assume that
		\[
			1 + 2 + 2^{2} + \cdots + 2^{k} = 2^{k+1} -1.
		\]

		\medskip

		\textbf{Inductive step:} Now to show this holds for $2^{k+1}$:
		\begin{align*}
			1 + 2 + 2^{2} + \cdots + 2^{k} + 2^{k+1}
			 & = 2^{k+1} -1 + 2^{k+1} \\
			 & = 2 \cdot 2^{k+1} -1   \\
			 & = 2^{k+2} -1.
		\end{align*}

		Therefore, by the principle of mathematical induction,
		\[
			1 + 2 + 2^{2} + \cdots + 2^{n} = 2^{\,n+1}-1.
		\]
	\end{proof}
\end{homeworkProblem}

\pagebreak
\begin{homeworkProblem}

	\begin{theorem}
		The number $n^2 - n$, is even for every $n \in \mathbb{Z}$.

	\end{theorem}

	\begin{proof}

		\mbox{}\\
		A number is called \emph{even} if it can be written in the form $2m$ for some $m \in \mathbb{Z}$.


		\textbf{Base case:} For $n=1$,

		\[
			1^2 - 1 = 0
		\]


		Since $0 = 2 \cdot 0$, it is divisible by $2$, 0 is even.

		\medskip

		\textbf{Induction hypothesis:} Assume that for some $k \in \mathbb{Z}$,
		\[
			k^2 - k \ \text{is even}.
		\]

		\medskip

		\textbf{Inductive step:} Consider $n = k+1$.
		\begin{align*}
			(k+1)^2 - (k+1) & = k^2 + 2k + 1 - k - 1 \\
			                & = k^2 - k + 2k         \\
			                & = (k^2 - k) + (2k).
		\end{align*}

		By the induction hypothesis, $k^2 - k$ is even.
		And 2k is even by the definition of even.

		\medskip

		Therefore, $(k+1)^2 - (k+1)$ is even.

		\medskip
	\end{proof}

\end{homeworkProblem}

\begin{homeworkProblem}

	\begin{theorem}
		Let $a,b,c \in \mathbb{Z}$. If $a \mid b$ and $a \mid c$, then $a \mid (b-c)$.
	\end{theorem}

	\begin{proof}
		\begin{align*}
			\step{a \mid b \ \text{ and } \ a \mid c}{assumption}
			\step{\exists k\in \mathbb{Z}\ \text{ s.t. }\ 20=4k,\ c=al}{definition of divisibility}
			\step{b-c = ak - al}{substitution}
			\step{b-c = a(k-l)}{algebra}
			\step{m = (k-l)}{set}
			\step{b-c = a\cdot m}{algebra}
			\step{a \mid (b-c)}{definition of divisibility}
		\end{align*}
	\end{proof}
\end{homeworkProblem}

\begin{homeworkProblem}

	\textbf{Problem 2:}
  \\
	\textbf{Assumptions:}
	\begin{itemize}
		\item $n$ is a natural number.
	\end{itemize}
	\textbf{Conclusion:}
	\begin{itemize}
		\item $1 + 2 + 2^{2} + \cdots + 2^{n} = 2^{\,n+1}-1$.
	\end{itemize}
	\textbf{Summary of proof:}
	The proof is by induction. The base case holds for $n=1$. Assuming the formula is true for $n=k$, adding $2^{k+1}$ to both sides gives the formula for $n=k+1$. Therefore, the formula holds for all $n$.

	\medskip

	\textbf{Problem 3:}
  \\
	\textbf{Assumptions:}
	\begin{itemize}
		\item $n$ is an integer.
	\end{itemize}
	\textbf{Conclusion:}
	\begin{itemize}
		\item $n^2 - n$ is even.
	\end{itemize}
	\textbf{Summary of proof:}
	The proof is by induction. The base case $n=1$ gives $0$, which is even. Assuming $k^2-k$ is even, then $(k+1)^2-(k+1) = (k^2-k)+2k$, which is a sum of two even numbers. Thus, the expression is even for all integers $n$.

	\medskip

	\textbf{Problem 4:}
  \\
	\textbf{Assumptions:}
	\begin{itemize}
		\item $a,b,c$ are integers.
		\item $a \mid b$ and $a \mid c$.
	\end{itemize}
	\textbf{Conclusion:}
	\begin{itemize}
		\item $a \mid (b-c)$.
	\end{itemize}
	\textbf{Summary of proof:}
	Since $a \mid b$ and $a \mid c$, we can write $b=ak$ and $c=al$. Then $b-c=a(k-l)$. As $k-l$ is an integer, it follows that $a \mid (b-c)$.

\end{homeworkProblem}

\pagebreak


\begin{homeworkProblem}


  \textbf{Problem 2:}  

	\begin{theorem}
		For every natural number $n$, $1 + 2 + 2^{2} + \cdots + 2^{n} = 2^{\,n+1}-1$.
	\end{theorem}

	\begin{proof}

		\mbox{}\\
		\textbf{Base case:} For $n=1$,

		\[
			1 + 2 = 2^{1+1} - 1 = 3,
		\]

		\medskip

		\textbf{Induction hypothesis:} Assume that
		\[
			1 + 2 + 2^{2} + \cdots + 2^{k} = 2^{k+1} -1.
		\]

		\medskip

		\textbf{Inductive step:} Add $2^{k+1}$ to both sides:
		\begin{align*}
			1 + 2 + 2^{2} + \cdots + 2^{k} + 2^{k+1}
			 & = 2^{k+1} -1 + 2^{k+1} \\
			 & = 2 \cdot 2^{k+1} -1   \\
			 & = 2^{k+2} -1.
		\end{align*}

		Therefore, by the principle of mathematical induction,
		\[
			1 + 2 + 2^{2} + \cdots + 2^{n} = 2^{\,n+1}-1.
		\]
	\end{proof}

    \medskip

  \textbf{Problem 3:}  
	\begin{theorem}
		The number $n^2 - n$, is even for every $n \in \mathbb{Z}$.

	\end{theorem}

	\begin{proof}

		\mbox{}\\
		Definition of even: A number is called \emph{even} if it can be written in the form $2m$ for some $m \in \mathbb{Z}$.


		\textbf{Base case:} For $n=1$,

		\[
			1^2 - 1 = 0
		\]


		Since $0 = 2 \cdot 0$, it is divisible by $2$, 0 is even.

		\medskip

		\textbf{Induction hypothesis:} Assume that for some $k \in \mathbb{Z}$,
		\[
			k^2 - k \ \text{is even}.
		\]

		\medskip

		\textbf{Inductive step:} Consider $n = k+1$.
		\begin{align*}
			(k+1)^2 - (k+1) & = k^2 + 2k + 1 - k - 1 \\
			                & = k^2 - k + 2k         \\
			                & = (k^2 - k) + (2k).
		\end{align*}

		By the induction hypothesis, $k^2 - k$ is even.
		And 2k is even by the definition of even.

		\medskip

		Therefore, $(k+1)^2 - (k+1)$ is even.

		\medskip
	\end{proof}
    \medskip

  \textbf{Problem 4}  
	\begin{theorem}
		Let $a,b,c \in \mathbb{Z}$. If $a \mid b$ and $a \mid c$, then $a \mid (b-c)$.
	\end{theorem}

	\begin{proof}
		\begin{align*}
			\step{a \mid b \ \text{ and } \ a \mid c}{assumption}
			\step{\exists k\in \mathbb{Z}\ \text{ s.t. }\ 20=4k,\ c=al}{definition of divisibility}
			\step{b-c = ak - al}{substitution}
			\step{b-c = a(k-l)}{algebra}
			\step{m = (k-l)}{set}
			\step{b-c = a\cdot m}{algebra}
			\step{a \mid (b-c)}{definition of divisibility}
		\end{align*}
	\end{proof}

\end{homeworkProblem}

\pagebreak
\begin{homeworkProblem}

	\textbf{Claim.} Let $a=12$, $b=2$, $c=2$, and $n=4$. Then
	$c \not\equiv 0 \pmod{n}$,
	$ac \equiv bc \pmod{n}$,
	and $a \not\equiv b \pmod{n}$.

	\medskip


	\begin{proof}
		\[
			a=12,\quad b=2,\quad c=2,\quad n=4
		\]


		\textbf{(i) Show } $c \not\equiv 0 \pmod 4$.
		\begin{align*}
			\step{4 \mid (c - 0)}{definition of congruence}
			\step{4 \mid 2}{substitution, algebra}
			\step{\text{False, since } 4 \nmid 2}{contradiction}
			\step{c \not\equiv 0 \pmod 4}{definition of congruence}
		\end{align*}

		\textbf{(ii) Show } $ac \equiv bc \pmod 4$.
		\begin{align*}
			\step{4 \mid (ac - bc)}{definition of congruence}
			\step{4 \mid c(a - b)}{algebra}
			\step{4 \mid 2 \cdot(12-2)}{substitute $a,b,c$}
			\step{4 \mid 20}{arithmetic}
			\step{\exists k\in \mathbb{Z}\ \text{ s.t. }\ 20=4k,\ c=al}{definition of divisibility}
			\step{k=5}{True}
		\end{align*}
		\textbf{(iii) Show } $a \not\equiv b \pmod 4$.
		\begin{align*}
			\step{4 \mid (a - b)}{definition of congruence}
			\step{4 \mid (12 - 2)}{substitute a,b}
			\step{4 \mid 10}{arithmetic}
			\step{\text{False, since } 4 \nmid 10}{contradiction}
		\end{align*}
		All three statements hold, so this choice of $(a,b,c,n)$ is valid.
	\end{proof}
\end{homeworkProblem}

\end{document}
