
\documentclass{article}

\usepackage{fancyhdr}
\usepackage{extramarks}
\usepackage{amsmath}
\usepackage{amsthm}
\usepackage{amsfonts}
\usepackage{tikz}
\usepackage[plain]{algorithm}
\usepackage{algpseudocode}
\usepackage{amsmath,amssymb}

\usetikzlibrary{automata,positioning}

%
% Basic Document Settings
%

\topmargin=-0.45in
\evensidemargin=0in
\oddsidemargin=0in
\textwidth=6.5in
\textheight=9.0in
\headsep=0.25in

\linespread{1.1}

\pagestyle{fancy}
\lhead{\hmwkAuthorName}
\chead{\hmwkClass: \hmwkTitle}
\rhead{}
\lfoot{\lastxmark}
\cfoot{\thepage}

\renewcommand\headrulewidth{0.4pt}
\renewcommand\footrulewidth{0.4pt}

\setlength\parindent{0pt}

%
% Create Problem Sections
%

\newcommand{\enterProblemHeader}[1]{
    \nobreak\extramarks{}{Problem \arabic{#1} continued on next page\ldots}\nobreak{}
    \nobreak\extramarks{Problem \arabic{#1} (continued)}{Problem \arabic{#1} continued on next page\ldots}\nobreak{}
}

\newcommand{\exitProblemHeader}[1]{
    \nobreak\extramarks{Problem \arabic{#1} (continued)}{Problem \arabic{#1} continued on next page\ldots}\nobreak{}
    \stepcounter{#1}
    \nobreak\extramarks{Problem \arabic{#1}}{}\nobreak{}
}

\setcounter{secnumdepth}{0}
\newcounter{partCounter}
\newcounter{homeworkProblemCounter}
\setcounter{homeworkProblemCounter}{1}
\nobreak\extramarks{Problem \arabic{homeworkProblemCounter}}{}\nobreak{}



% Define theorem environment
\newtheorem*{theorem}{Theorem}

%
% Homework Problem Environment
%
% This environment takes an optional argument. When given, it will adjust the
% problem counter. This is useful for when the problems given for your
% assignment aren't sequential. See the last 3 problems of this template for an
% example.
%
\newenvironment{homeworkProblem}[1][-1]{
    \ifnum#1>0
        \setcounter{homeworkProblemCounter}{#1}
    \fi
    \section{Problem \arabic{homeworkProblemCounter}}
    \setcounter{partCounter}{1}
    \enterProblemHeader{homeworkProblemCounter}
}{
    \exitProblemHeader{homeworkProblemCounter}
}

%
% Homework Details
%   - Title
%   - Due date
%   - Class
%   - Section/Time
%   - Instructor
%   - Author
%


\newcommand{\hmwkTitle}{Problem set\ 5}
\newcommand{\hmwkDueDate}{October 3, 2025}
\newcommand{\hmwkClass}{Number Theory}
\newcommand{\hmwkClassTime}{Section 2}
\newcommand{\hmwkClassInstructor}{Dr. Eleanor McSpirit}
\newcommand{\hmwkAuthorName}{\textbf{Joseph Quinn}}

%
% Title Page
%

\title{
    \vspace{2in}
    \textmd{\textbf{\hmwkClass:\ \hmwkTitle}}\\
    \normalsize\vspace{0.1in}\small{\hmwkDueDate}\\
    \vspace{0.1in}\large{\textit{\hmwkClassInstructor\ \hmwkClassTime}}
    \vspace{3in}
}

\author{\hmwkAuthorName}
\date{}

\renewcommand{\part}[1]{\textbf{\large Part \Alph{partCounter}}\stepcounter{partCounter}\\}

%
% Various Helper Commands
%

% Useful for algorithms
\newcommand{\alg}[1]{\textsc{\bfseries \footnotesize #1}}

% For derivatives
\newcommand{\deriv}[1]{\frac{\mathrm{d}}{\mathrm{d}x} (#1)}

% For partial derivatives
\newcommand{\pderiv}[2]{\frac{\partial}{\partial #1} (#2)}

% Integral dx
\newcommand{\dx}{\mathrm{d}x}

% Alias for the Solution section header
\newcommand{\solution}{\textbf{\large Solution}}

% Probability commands: Expectation, Variance, Covariance, Bias
\newcommand{\E}{\mathrm{E}}
\newcommand{\Var}{\mathrm{Var}}
\newcommand{\Cov}{\mathrm{Cov}}
\newcommand{\Bias}{\mathrm{Bias}}

%  proof-step macro:
\newcommand{\step}[2]{& #1 & & \text{#2} \\}

\begin{document}

\maketitle
\pagebreak

\begin{homeworkProblem}
Since $2,3,$ and $5$ are pairwise coprime, the Chinese Remainder Theorem tells us
that there is a unique solution modulo $2 \cdot 3 \cdot 5 = 30$.

We first solve the system formed by the first two congruences. Since
\[
x \equiv 1 \pmod{2}
\quad\text{and}\quad
x \equiv 1 \pmod{3},
\]
we see that $x \equiv 1 \pmod{6}$. Thus any solution must be of the form
\[
x = 1 + 6k, \qquad k \in \mathbb{Z}.
\]
Now we impose the third congruence,
\[
1 + 6k \equiv 2 \pmod{5}
\quad\Longrightarrow\quad
6k \equiv 1 \pmod{5}.
\]
Since $6 \equiv 1 \pmod{5}$, this reduces to $k \equiv 1 \pmod{5}$. Hence
\[
k = 1 + 5t, \qquad t \in \mathbb{Z},
\]
and so
\[
x = 1 + 6(1 + 5t) = 7 + 30t.
\]
Therefore, the unique solution modulo $30$ is
\[
\boxed{x \equiv 7 \pmod{30}}.
\]
For example, $x = 7$ and $x = 37$ are both valid solutions.


\end{homeworkProblem}

\begin{homeworkProblem}
	\paragraph{(i)}
	a = {1,3,5,7} s.t. (a,8) = 1

	\paragraph{(ii)}
	\begin{enumerate}
		\item $ord_8(1)=1$
		\item $ord_8(3)=2$
		\item $ord_8(5)=2$
		\item $ord_8(7)=2$
	\end{enumerate}

\end{homeworkProblem}

\begin{homeworkProblem}
	(v1 $\Rightarrow$ v2)
	\begin{proof}
		\[
			a^{p-1} \equiv 1 \pmod{p}.
		\]

		Let $a$ be any integer.

		\begin{itemize}
			\item \textbf{Case 1:} Suppose $p \mid a$. Then $a \equiv 0 \pmod{p}$.
			      Raising both sides to the $p$th power gives
			      \[
				      a^p \equiv 0^p \equiv 0 \pmod{p}.
			      \]
			      Hence $a^p \equiv a \pmod{p}$.

			\item \textbf{Case 2:} Suppose $(a,p)=1$.
			      By assumption, $a^{p-1} \equiv 1 \pmod{p}$.
			      Multiplying both sides of this congruence by $a$, we can write
			      \[
				      a \cdot a^{p-1} \equiv a \cdot 1 \pmod{p}.
			      \]
			      Simplifying gives
			      \[
				      a^p \equiv a \pmod{p}.
			      \]
		\end{itemize}

		So in both cases Theorem 4.16 holds.

		\medskip

		(v2 $\Rightarrow$ v1)

		\[
			a^p \equiv a \pmod{p}.
		\]

		Let $(a,p)=1$. Then subtracting $a$ from both sides gives
		\[
			a^p - a \equiv 0 \pmod{p}.
		\]
		This means
		\[
			p \mid (a^p - a).
		\]

		We can rewrite the rhs as
		\[
			p \mid a(a^{p-1} - 1).
		\]

		By Euclid's Lemma:
		\[
			p \mid (a^{p-1} - 1).
		\]

		This is equivalent to
		\[
			a^{p-1} \equiv 1 \pmod{p},
		\]
		which is Theorem 4.15.
	\end{proof}
\end{homeworkProblem}

\begin{homeworkProblem}
Compute $3444^{3233} \pmod{17}$.

First note that by the Division Algorithm we have that $3444 \equiv 10 \pmod{17}$, so
\[
3444^{3233} \equiv 10^{3233} \pmod{17}.
\]
Now since $(10,17) = 1$ we may apply Fermat's Little Theorem. We compute $17 - 1 = 16$ and that $3233 \equiv 1 \pmod{16}$. Therefore, we have that
\[
3444^{3233} \equiv 10^{3233} \equiv 10^1 \equiv 10 \pmod{17}.
\]

\bigskip

Compute $123^{456} \pmod{23}$.

First note that $123 \equiv 8 \pmod{23}$, so
\[
123^{456} \equiv 8^{456} \pmod{23}.
\]
Now since $(8,23) = 1$ we may apply Fermat's Little Theorem. We compute $23 - 1 = 22$ and that $456 \equiv 16 \pmod{22}$. Therefore, we have that
\[
123^{456} \equiv 8^{456} \equiv 8^{16} \pmod{23}.
\]
Next we compute $8^{16}$ by repeated squaring:
\[
8^2 \equiv 64 \equiv 18,\qquad 18^2 \equiv 324 \equiv 2,\qquad 2^2 \equiv 4,\qquad 4^2 \equiv 16 \pmod{23}.
\]
Therefore
\[
123^{456} \equiv 16 \pmod{23}.
\]

\end{homeworkProblem}

\begin{homeworkProblem}
    Let $p$ and $q$ be distinct prime numbers. Prove that $p^{q-1} + q^{p-1} \equiv 1 \pmod{pq}$.
\begin{proof}

    Because $p$ divides any positive power of $p$. We can write
    \[
        p^{\,q-1} \equiv 0 \pmod{p},
    \]

    Also, since $p$ and $q$ are distinct primes, we have $(q,p)=1$.  
    By Fermat's Little Theorem v1 we see
    \[
        q^{\,p-1} \equiv 1 \pmod{p}.
    \]
    Therefore, by Theorem 1.12,
    \[
        p^{\,q-1} + q^{\,p-1} \equiv 0 + 1 \equiv 1 \pmod{p}.
    \]

    Now doing the same for mod q,
    \[
        q^{\,p-1} \equiv 0 \pmod{q}.
    \]
    Since $(p,q)=1$, we again apply Fermat's Little Theorem:
    \[
        p^{\,q-1} \equiv 1 \pmod{q}.
    \]
    Therefore,
    \[
        p^{\,q-1} + q^{\,p-1} \equiv 1 + 0 \equiv 1 \pmod{q}.
    \]


    Since $p^{\,q-1} + q^{\,p-1} \equiv 1 \pmod{p}$ and  
    $p^{\,q-1} + q^{\,p-1} \equiv 1 \pmod{q}$, and since $p$ and $q$ are coprime,
    it follows that
    \[
        p^{\,q-1} + q^{\,p-1} \equiv 1 \pmod{pq}.
    \]
\end{proof}
\end{homeworkProblem}

\begin{homeworkProblem}

\begin{proof}
Recall that
\[
\binom{p}{i} = \frac{p!}{i!(p-i)!}.
\]
We write
\[
p! = p \cdot (p-1)(p-2)\cdots 1,
\]
so
\[
\binom{p}{i} = p \cdot \frac{(p-1)(p-2)\cdots 1}{i!(p-i)!}.
\]
Since $i!$ and $(p-i)!$ contain only factors strictly less than $p$, neither of them is divisible by $p$ (because $p$ is prime). Therefore the denominator is not divisible by $p$, and thus $\binom{p}{i}$ is an integer multiple of $p$.
Hence $p \mid \binom{p}{i}$.
\end{proof}

\begin{theorem}[Fermat's Little Theorem, Version II]
If $p$ is a prime and $a$ is an integer, then $a^p \equiv a \pmod{p}$.
\end{theorem}

\begin{proof}
We proceed by induction on $a$.

For $a = 0$, we have $0^p \equiv 0 \pmod{p}$. For $a = 1$, we have $1^p = 1 \equiv 1 \pmod{p}$. Thus the statement holds for $a = 0$ and $a = 1$.

Assume the statement holds for some integer $a$, i.e.\ $a^p \equiv a \pmod{p}$. Consider $a+1$. By the Binomial Theorem,
\[
(a+1)^p = \sum_{i=0}^{p} \binom{p}{i} a^{p-i}.
\]
Expanding,
\[
(a+1)^p = a^p + \binom{p}{1} a^{p-1} + \binom{p}{2} a^{p-2} + \cdots + \binom{p}{p-1} a + 1.
\]
By Lemma 4.25, each $\binom{p}{i}$ with $1 \le i \le p-1$ is divisible by $p$. Therefore all middle terms vanish modulo $p$ and we obtain
\[
(a+1)^p \equiv a^p + 1 \pmod{p}.
\]
Applying the induction hypothesis $a^p \equiv a$, we conclude
\[
(a+1)^p \equiv a + 1 \pmod{p}.
\]
Thus the statement holds for $a+1$. By induction, it follows that $a^p \equiv a \pmod{p}$ for all integers $a \ge 0$.

For negative $a$, the result follows from the fact that congruences are compatible with additive inverses.
\end{proof}

\end{homeworkProblem}

\end{document}
