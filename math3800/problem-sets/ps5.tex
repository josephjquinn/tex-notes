
\documentclass{article}

\usepackage{fancyhdr}
\usepackage{extramarks}
\usepackage{amsmath}
\usepackage{amsthm}
\usepackage{amsfonts}
\usepackage{tikz}
\usepackage[plain]{algorithm}
\usepackage{algpseudocode}
\usepackage{amsmath,amssymb}

\usetikzlibrary{automata,positioning}

%
% Basic Document Settings
%

\topmargin=-0.45in
\evensidemargin=0in
\oddsidemargin=0in
\textwidth=6.5in
\textheight=9.0in
\headsep=0.25in

\linespread{1.1}

\pagestyle{fancy}
\lhead{\hmwkAuthorName}
\chead{\hmwkClass: \hmwkTitle}
\rhead{}
\lfoot{\lastxmark}
\cfoot{\thepage}

\renewcommand\headrulewidth{0.4pt}
\renewcommand\footrulewidth{0.4pt}

\setlength\parindent{0pt}

%
% Create Problem Sections
%

\newcommand{\enterProblemHeader}[1]{
    \nobreak\extramarks{}{Problem \arabic{#1} continued on next page\ldots}\nobreak{}
    \nobreak\extramarks{Problem \arabic{#1} (continued)}{Problem \arabic{#1} continued on next page\ldots}\nobreak{}
}

\newcommand{\exitProblemHeader}[1]{
    \nobreak\extramarks{Problem \arabic{#1} (continued)}{Problem \arabic{#1} continued on next page\ldots}\nobreak{}
    \stepcounter{#1}
    \nobreak\extramarks{Problem \arabic{#1}}{}\nobreak{}
}

\setcounter{secnumdepth}{0}
\newcounter{partCounter}
\newcounter{homeworkProblemCounter}
\setcounter{homeworkProblemCounter}{1}
\nobreak\extramarks{Problem \arabic{homeworkProblemCounter}}{}\nobreak{}



% Define theorem environment
\newtheorem*{theorem}{Theorem}

%
% Homework Problem Environment
%
% This environment takes an optional argument. When given, it will adjust the
% problem counter. This is useful for when the problems given for your
% assignment aren't sequential. See the last 3 problems of this template for an
% example.
%
\newenvironment{homeworkProblem}[1][-1]{
    \ifnum#1>0
        \setcounter{homeworkProblemCounter}{#1}
    \fi
    \section{Problem \arabic{homeworkProblemCounter}}
    \setcounter{partCounter}{1}
    \enterProblemHeader{homeworkProblemCounter}
}{
    \exitProblemHeader{homeworkProblemCounter}
}

%
% Homework Details
%   - Title
%   - Due date
%   - Class
%   - Section/Time
%   - Instructor
%   - Author
%


\newcommand{\hmwkTitle}{Problem set\ 5}
\newcommand{\hmwkDueDate}{October 3, 2025}
\newcommand{\hmwkClass}{Number Theory}
\newcommand{\hmwkClassTime}{Section 2}
\newcommand{\hmwkClassInstructor}{Dr. Eleanor McSpirit}
\newcommand{\hmwkAuthorName}{\textbf{Joseph Quinn}}

%
% Title Page
%

\title{
    \vspace{2in}
    \textmd{\textbf{\hmwkClass:\ \hmwkTitle}}\\
    \normalsize\vspace{0.1in}\small{\hmwkDueDate}\\
    \vspace{0.1in}\large{\textit{\hmwkClassInstructor\ \hmwkClassTime}}
    \vspace{3in}
}

\author{\hmwkAuthorName}
\date{}

\renewcommand{\part}[1]{\textbf{\large Part \Alph{partCounter}}\stepcounter{partCounter}\\}

%
% Various Helper Commands
%

% Useful for algorithms
\newcommand{\alg}[1]{\textsc{\bfseries \footnotesize #1}}

% For derivatives
\newcommand{\deriv}[1]{\frac{\mathrm{d}}{\mathrm{d}x} (#1)}

% For partial derivatives
\newcommand{\pderiv}[2]{\frac{\partial}{\partial #1} (#2)}

% Integral dx
\newcommand{\dx}{\mathrm{d}x}

% Alias for the Solution section header
\newcommand{\solution}{\textbf{\large Solution}}

% Probability commands: Expectation, Variance, Covariance, Bias
\newcommand{\E}{\mathrm{E}}
\newcommand{\Var}{\mathrm{Var}}
\newcommand{\Cov}{\mathrm{Cov}}
\newcommand{\Bias}{\mathrm{Bias}}

%  proof-step macro:
\newcommand{\step}[2]{& #1 & & \text{#2} \\}

\begin{document}

\maketitle
\pagebreak
\begin{homeworkProblem}
	\paragraph{(i)}
	a = {1,3,5,7} s.t. (a,8) = 1

	\paragraph{(ii)}
	\begin{enumerate}
		\item $ord_8(1)=1$
		\item $ord_8(3)=2$
		\item $ord_8(5)=2$
		\item $ord_8(7)=2$
	\end{enumerate}

\end{homeworkProblem}

\begin{homeworkProblem}
	(v1 $\Rightarrow$ v2)
	\begin{proof}
		\[
			a^{p-1} \equiv 1 \pmod{p}.
		\]

		Let $a$ be any integer.

		\begin{itemize}
			\item \textbf{Case 1:} Suppose $p \mid a$. Then $a \equiv 0 \pmod{p}$.
			      Raising both sides to the $p$th power gives
			      \[
				      a^p \equiv 0^p \equiv 0 \pmod{p}.
			      \]
			      Hence $a^p \equiv a \pmod{p}$.

			\item \textbf{Case 2:} Suppose $(a,p)=1$.
			      By assumption, $a^{p-1} \equiv 1 \pmod{p}$.
			      Multiplying both sides of this congruence by $a$, we can write
			      \[
				      a \cdot a^{p-1} \equiv a \cdot 1 \pmod{p}.
			      \]
			      Simplifying gives
			      \[
				      a^p \equiv a \pmod{p}.
			      \]
		\end{itemize}

		So in both cases Theorem 4.16 holds.

		\medskip

		(v2 $\Rightarrow$ v1)

		\[
			a^p \equiv a \pmod{p}.
		\]

		Let $(a,p)=1$. Then subtracting $a$ from both sides gives
		\[
			a^p - a \equiv 0 \pmod{p}.
		\]
		This means
		\[
			p \mid (a^p - a).
		\]

		We can rewrite the rhs as
		\[
			p \mid a(a^{p-1} - 1).
		\]

		By Euclid's Lemma:
		\[
			p \mid (a^{p-1} - 1).
		\]

		This is equivalent to
		\[
			a^{p-1} \equiv 1 \pmod{p},
		\]
		which is Theorem 4.15.
	\end{proof}
\end{homeworkProblem}

\begin{homeworkProblem}

	\begin{enumerate}
		\item $512^{372} \bmod 13$

		      \begin{align*}
			      512     & \equiv 5 \pmod{13}                                                   &  & (\text{since } 512 = 13 \cdot 39 + 5) \\
			      5^{12}  & \equiv 1 \pmod{13}                                                   &  & (\text{by Fermat's Little Theorem})   \\
			      372     & = 12 \cdot 31 + 0                                                                                               \\
			      5^{372} & = (5^{12})^{31} \cdot 5^{0} \equiv 1^{31} \cdot 1 \equiv 1 \pmod{13}
		      \end{align*}

		      \[
			      \boxed{512^{372} \equiv 1 \pmod{13}}
		      \]

		\item $3444^{3233} \bmod 17$

		      \begin{align*}
			      3444      & \equiv 10 \pmod{17}                                                        &  & (\text{since } 3444 = 17 \cdot 202 + 10) \\
			      10^{16}   & \equiv 1 \pmod{17}                                                         &  & (\text{by Fermat's Little Theorem})      \\
			      3233      & = 16 \cdot 202 + 1                                                                                                       \\
			      10^{3233} & = (10^{16})^{202} \cdot 10^{1} \equiv 1^{202} \cdot 10 \equiv 10 \pmod{17}
		      \end{align*}

		      \[
			      \boxed{3444^{3233} \equiv 10 \pmod{17}}
		      \]

		\item $123456 \bmod 23$

		      \begin{align*}
			      123456 & = 23 \cdot 5367 + 15 \\
			      123456 & \equiv 15 \pmod{23}
		      \end{align*}

		      \[
			      \boxed{123456 \equiv 15 \pmod{23}}
		      \]
	\end{enumerate}

\end{homeworkProblem}

\end{document}
