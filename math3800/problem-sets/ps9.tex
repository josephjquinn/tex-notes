
\documentclass{article}

\usepackage{fancyhdr}
\usepackage{extramarks}
\usepackage{amsmath}
\usepackage{amsthm}
\usepackage{amsfonts}
\usepackage{tikz}
\usepackage[plain]{algorithm}
\usepackage{algpseudocode}
\usepackage{amsmath,amssymb}

\usetikzlibrary{automata,positioning}

%
% Basic Document Settings
%

\topmargin=-0.45in
\evensidemargin=0in
\oddsidemargin=0in
\textwidth=6.5in
\textheight=9.0in
\headsep=0.25in

\linespread{1.1}

\pagestyle{fancy}
\lhead{\hmwkAuthorName}
\chead{\hmwkClass: \hmwkTitle}
\rhead{}
\lfoot{\lastxmark}
\cfoot{\thepage}

\renewcommand\headrulewidth{0.4pt}
\renewcommand\footrulewidth{0.4pt}

\setlength\parindent{0pt}

%
% Create Problem Sections
%

\newcommand{\enterProblemHeader}[1]{
    \nobreak\extramarks{}{Problem \arabic{#1} continued on next page\ldots}\nobreak{}
    \nobreak\extramarks{Problem \arabic{#1} (continued)}{Problem \arabic{#1} continued on next page\ldots}\nobreak{}
}

\newcommand{\exitProblemHeader}[1]{
    \nobreak\extramarks{Problem \arabic{#1} (continued)}{Problem \arabic{#1} continued on next page\ldots}\nobreak{}
    \stepcounter{#1}
    \nobreak\extramarks{Problem \arabic{#1}}{}\nobreak{}
}

\setcounter{secnumdepth}{0}
\newcounter{partCounter}
\newcounter{homeworkProblemCounter}
\setcounter{homeworkProblemCounter}{1}
\nobreak\extramarks{Problem \arabic{homeworkProblemCounter}}{}\nobreak{}



% Define theorem environment
\newtheorem*{theorem}{Theorem}

%
% Homework Problem Environment
%
% This environment takes an optional argument. When given, it will adjust the
% problem counter. This is useful for when the problems given for your
% assignment aren't sequential. See the last 3 problems of this template for an
% example.
%
\newenvironment{homeworkProblem}[1][-1]{
    \ifnum#1>0
        \setcounter{homeworkProblemCounter}{#1}
    \fi
    \section{Problem \arabic{homeworkProblemCounter}}
    \setcounter{partCounter}{1}
    \enterProblemHeader{homeworkProblemCounter}
}{
    \exitProblemHeader{homeworkProblemCounter}
}

%
% Homework Details
%   - Title
%   - Due date
%   - Class
%   - Section/Time
%   - Instructor
%   - Author
%


\newcommand{\hmwkTitle}{Problem set\ 9}
\newcommand{\hmwkDueDate}{November 8, 2025}
\newcommand{\hmwkClass}{Number Theory}
\newcommand{\hmwkClassTime}{Section 2}
\newcommand{\hmwkClassInstructor}{Dr. Eleanor McSpirit}
\newcommand{\hmwkAuthorName}{\textbf{Joseph Quinn}}

%
% Title Page
%

\title{
    \vspace{2in}
    \textmd{\textbf{\hmwkClass:\ \hmwkTitle}}\\
    \normalsize\vspace{0.1in}\small{\hmwkDueDate}\\
    \vspace{0.1in}\large{\textit{\hmwkClassInstructor\ \hmwkClassTime}}
    \vspace{3in}
}

\author{\hmwkAuthorName}
\date{}

\renewcommand{\part}[1]{\textbf{\large Part \Alph{partCounter}}\stepcounter{partCounter}\\}

%
% Various Helper Commands
%

% Useful for algorithms
\newcommand{\alg}[1]{\textsc{\bfseries \footnotesize #1}}

% For derivatives
\newcommand{\deriv}[1]{\frac{\mathrm{d}}{\mathrm{d}x} (#1)}

% For partial derivatives
\newcommand{\pderiv}[2]{\frac{\partial}{\partial #1} (#2)}

% Integral dx
\newcommand{\dx}{\mathrm{d}x}

% Alias for the Solution section header
\newcommand{\solution}{\textbf{\large Solution}}

% Probability commands: Expectation, Variance, Covariance, Bias
\newcommand{\E}{\mathrm{E}}
\newcommand{\Var}{\mathrm{Var}}
\newcommand{\Cov}{\mathrm{Cov}}
\newcommand{\Bias}{\mathrm{Bias}}

%  proof-step macro:
\newcommand{\step}[2]{& #1 & & \text{#2} \\}

\begin{document}

\maketitle
\pagebreak

\begin{homeworkProblem}
	We can factor the congruence:
	\[
		x^2 - 1 \equiv 0 \pmod{p} \quad \Longleftrightarrow \quad (x - 1)(x + 1) \equiv 0 \pmod{p}.
	\]
	Because we are mod a prime we can separate this into:
	\[
		x \equiv 1 \pmod{p} \quad \text{or} \quad x \equiv -1 \pmod{p}.
	\]
	Thus the two incongruent solutions are
	\[
		x \equiv 1,\, -1 \pmod{p}
	\]
	No others exist, since a quadratic equation modulo a prime has at most two distinct roots.
\end{homeworkProblem}

\begin{homeworkProblem}
	\[
		\begin{array}{c|cccccccc}
			x           & 0 & 1 & 2 & 3 & 4 & 5 & 6 & 7 \\ \hline
			x^2 \bmod 8 & 0 & 1 & 4 & 1 & 0 & 1 & 4 & 1
		\end{array}
	\]
	Hence,
	\[
		x^2 \equiv 1 \pmod{8}
		\quad \text{for } x \equiv 1,\, 3,\, 5,\, 7 \pmod{8}.
	\]
	Therefore, the four incongruent solutions are
	\[
		x \equiv 1,\, 3,\, 5,\, 7 \pmod{8}.
	\]

	This result does not contradict Lagrange’s Theorem, because that theorem applies only when the modulus is prime.
\end{homeworkProblem}

\begin{homeworkProblem}

	\textbf{(i) Mod 7:}

	\[
		\begin{array}{c|cccccc}
			x           & 1 & 2 & 3 & 4 & 5 & 6 \\ \hline
			x^2 \bmod 7 & 1 & 4 & 2 & 2 & 4 & 1
		\end{array}
	\]
	The distinct quadratic residues are
	\[
		1,\, 2,\, 4,
	\]
	and the quadratic nonresidues are
	\[
		3,\, 5,\, 6.
	\]

	\textbf{(ii) Mod 13:}

	\[
		\begin{array}{c|cccccc}
			x            & 1 & 2 & 3 & 4 & 5  & 6  \\ \hline
			x^2 \bmod 13 & 1 & 4 & 9 & 3 & 12 & 10
		\end{array}
	\]
	The distinct quadratic residues are
	\[
		1,\, 3,\, 4,\, 9,\, 10,\, 12,
	\]
	and the quadratic nonresidues are
	\[
		2,\, 5,\, 6,\, 7,\, 8,\, 11.
	\]

\end{homeworkProblem}

\begin{homeworkProblem}
	\textbf{(i)}

	Since $-5 \equiv 2 \pmod{7}$, we can rewrite the congruence as:
	\[
		x^2 + 3x + 2 \equiv 0 \pmod{7}.
	\]
	Then we can factor the rhs
	\[
		(x + 1)(x + 2) \equiv 0 \pmod{7}.
	\]

	And because 7 is prime this becomes two separate congruences:
	\[
		x \equiv -1 \pmod7
	\]
	and
	\[
		x \equiv -2 \pmod7
	\]

	Thus, the incongruent solutions are
	\[
		x \equiv 5,\,6 \pmod{7}.
	\]

	\textbf{(ii)}
	\[
		3x^2 + 2x - 4 \equiv 0 \pmod{13}.
	\]
	We first find the inverse of $3$ modulo $13$. We need $3x \equiv 1 \pmod{13}$. Off the bat I can see that $3 \times 9 = 27$. And we can write:

	$3^{-1} \equiv 9 \pmod{13}$.

	Multiply the entire congruence by $9$:
	\[
		9(3x^2 + 2x - 4) \equiv 0 \pmod{13},
	\]
	which simplifies to
	\[
		x^2 + 18x - 36 \equiv 0 \pmod{13}.
	\]
	Reduce coefficients modulo $13$:
	\[
		18 \equiv 5, \quad -36 \equiv 3,
	\]
	so
	\[
		x^2 + 5x + 3 \equiv 0 \pmod{13}.
	\]

	Now we want to complete the square
	\[
		x^2 + 5x \equiv -3 \pmod{13}.
	\]
	We need to divide $5$ by $2$ modulo $13$, so we first find $2^{-1} \pmod{13}$.

	We solve $2x\equiv 1 \pmod{13}$:
	\[
		2 \times 7 = 14 \equiv 1 \pmod{13},
	\]
	so $2^{-1} \equiv 7 \pmod{13}$.

	Then
	\[
		\frac{5}{2} \equiv 5 \times 7 \equiv 35 \equiv 9 \pmod{13}.
	\]
	and taking the square we get
	\[
		\left(\frac{5}{2}\right)^2 \equiv 9^2 \equiv 81 \equiv 3 \pmod{13}.
	\]

	Adding this to both sides and simplifying we get:
	\[
		(x + 9)^2 \equiv 0 \pmod{13}.
	\]
	Therefore,
	\[
		x + 9 \equiv 0 \pmod{13} \quad \Rightarrow \quad x \equiv 4 \pmod{13}.
	\]
\end{homeworkProblem}

\begin{homeworkProblem}

	\textbf{(i)}

	\[
		2^{(11-1)/2} = 2^5 = 32 \equiv 10 \equiv -1 \pmod{11}.
	\]
	So $\displaystyle \left(\frac{2}{11}\right) = -1$.

	\textbf{(ii)}
	\[
		3^{(7-1)/2} = 3^3 = 27 \equiv 6 \equiv -1 \pmod{7}.
	\]
	So $\displaystyle \left(\frac{3}{7}\right) = -1$.

	\textbf{(iii)}
	\[
		7^{(3-1)/2} = 7^1 \equiv 1 \pmod{3}.
	\]
	So $\displaystyle \left(\frac{7}{3}\right) = 1$.

\end{homeworkProblem}

\begin{homeworkProblem}
\textbf{(i)}
\[
	\left(\frac{-1}{p}\right) =
	\begin{cases}
		1  & \text{if } p \equiv 1 \pmod{4}, \\
		-1 & \text{if } p \equiv 3 \pmod{4}.
	\end{cases}
\]
Hence $x^2 \equiv -1 \pmod{p}$ has a solution iff $p \equiv 1 \pmod{4}$.

\end{document}
