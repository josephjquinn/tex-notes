\documentclass{article}

\usepackage{amsmath}
\usepackage{amsthm}
\usepackage{amsfonts}
\usepackage{amssymb}
\usepackage{array}


\topmargin=-0.45in
\evensidemargin=0in
\oddsidemargin=0in
\textwidth=6.5in
\textheight=9.0in
\headsep=0.25in

\linespread{1.1}
\setlength\parindent{0pt}

\newtheorem{theorem}{Theorem}
\newtheorem*{theorem*}{Theorem}
\theoremstyle{definition}
\newtheorem{definition}{Definition}
\newtheorem*{definition*}{Definition}

% Command for manual theorem numbering
\newcommand{\manualnum}[1]{\setcounter{theorem}{\numexpr#1-1\relax}}

% Environment for theorems with manual numbers
\newenvironment{manualtheorem}[1]{%
  \renewcommand{\thetheorem}{#1}%
  \theorem%
}{%
  \endtheorem%
}

\begin{document}
\setcounter{section}{-1}


\section{Introduction}
\[
	\mathbb{N} = \{1,2,3,4,\dots\}
\]
\[
	\mathbb{N}_0 = \{0,1,2,3,4,\dots\}
\]
\[
	\mathbb{Z} = \{\dots,-3,-2,-1,0,1,2,3,\dots\}
\]

\begin{itemize}
	\item $\mathbb{N}$: only positive whole numbers (and maybe $0$).
	\item $\mathbb{Z}$: all whole numbers, positive, negative, and $0$.
\end{itemize}
\subsection{Proof Strategies}

\begin{tabular}{|>{\bfseries}m{3cm}|m{10cm}|}
	\hline
	Strategy       & Core Idea                                                                                                                                                             \\
	\hline
	Direct         & Start from given assumptions and proceed step by step until the desired statement emerges.                                                                            \\
	\hline
	Contrapositive & Instead of proving ``If $P$ is true, then $Q$ is true,'' prove the equivalent statement ``If $Q$ is false, then $P$ is false.'' Often the negative form is easier.    \\
	\hline
	Contradiction  & Instead of proving ``If $P$ is true, then $Q$ is true,'' assume ``$P$ is true and $Q$ is false'' and use the assumptions to reach an impossibility (such as $0 = 1$). \\
	\hline
	Induction      & Prove an initial case; then show that whenever the statement holds for one (or all earlier) integer(s), it holds for the next.                                        \\
	\hline
\end{tabular}


\section{Divide and Conquer}

\subsection{Divisibility}

\begin{definition*}
	Suppose $a$ and $d$ are integers. Then we say $d$ divides $a$, denoted $d \mid a$, if (and only if) there is an integer $k$ such that $a = kd$. We may also say that $d$ is a divisor or factor of $a$, and that $a$ is a multiple of $d$.
\end{definition*}



\begin{theorem*}
	Let $n$ be an integer. If $14 \mid n$, then $7 \mid n$.
\end{theorem*}

\begin{manualtheorem}{1.1}
	Let $a, b, c \in \mathbb{Z}$. If $a \mid b$ and $a \mid c$, then $a \mid (b + c)$.
\end{manualtheorem}

\begin{manualtheorem}{1.3}
	Let $a, b, c \in \mathbb{Z}$. If $a \mid b$ and $a \mid c$, then $a \mid bc$.
\end{manualtheorem}

\begin{manualtheorem}{1.32}
	Let $a, n, b, r, k \in \mathbb{Z}$. If $a = nb + r$ and $k \mid a$ and $k \mid b$, then $k \mid r$.
\end{manualtheorem}


\subsection{Congruency}
\begin{definition*}
	Suppose $a$, $b$ and $n$ are integers, with $n > 0$. We say that $a$ and $b$ are congruent modulo $n$ if $n \mid (a - b)$. We denote this relationship as
	\(a \equiv b \pmod{n}\)
\end{definition*}



\begin{theorem*}
	Let $k \in \mathbb{Z}$. If $k \equiv 5 \pmod{2}$, then $k \equiv 3 \pmod{2}$.
\end{theorem*}

\begin{manualtheorem}{1.9}
	Let $a, n \in \mathbb{Z}$ with $n > 0$. Then $a \equiv a \pmod{n}$.
\end{manualtheorem}

\begin{manualtheorem}{1.10}
	Let $a, b, n \in \mathbb{Z}$ with $n > 0$. If $a \equiv b \pmod{n}$, then $b \equiv a \pmod{n}$.
\end{manualtheorem}

\begin{manualtheorem}{1.11}
	Let $a, b, c, n \in \mathbb{Z}$ with $n > 0$. If $a \equiv b \pmod{n}$ and $b \equiv c \pmod{n}$, then $a \equiv c \pmod{n}$.
\end{manualtheorem}

\begin{manualtheorem}{1.12}
	Let $a, b, c, d, n \in \mathbb{Z}$ with $n > 0$. If $a \equiv b \pmod{n}$ and $c \equiv d \pmod{n}$, then $a + c \equiv b + d \pmod{n}$.
\end{manualtheorem}

\begin{manualtheorem}{1.14}
	Let $a, b, c, d, n \in \mathbb{Z}$ with $n > 0$. If $a \equiv b \pmod{n}$ and $c \equiv d \pmod{n}$, then $ac \equiv bd \pmod{n}$.
\end{manualtheorem}

\begin{manualtheorem}{1.18}
	Let $a, b, k, n \in \mathbb{Z}$ with $n > 0$ and $k > 0$. If $a \equiv b \pmod{n}$, then $a^k \equiv b^k \pmod{n}$.
\end{manualtheorem}

\begin{manualtheorem}{1.21}
	Let a natural number $n$ be expressed in base 10 as $n = a_ka_{k-1}\ldots a_1a_0$. If $m = a_k + a_{k-1} + \cdots + a_1 + a_0$, then $n \equiv m \pmod{9}$.
\end{manualtheorem}

\begin{manualtheorem}{1.45}
	Let $a, b, c, n \in \mathbb{Z}$ with $n>0$. If $ac \equiv bc \pmod{n}$ and $(c,n)=1$, then $a \equiv b \pmod{n}$.
\end{manualtheorem}


\subsection{Number Properties}
\begin{theorem*}
	For every natural number $n$, $1 + 2 + 2^{2} + \cdots + 2^{n} = 2^{\,n+1}-1$.
\end{theorem*}

\begin{manualtheorem}{A.1}
	For all $n \in \mathbb{N}$, we have $1 + 2 + \cdots + n = \frac{n(n+1)}{2}$.
\end{manualtheorem}

\begin{theorem*}
	The number $n^2 - n$ is even for every $n \in \mathbb{Z}$.
\end{theorem*}

\begin{theorem*}
	Let $S$ be any nonempty set of natural numbers. Then $S$ has a smallest element.
\end{theorem*}

\begin{manualtheorem}{1.27}
	For every natural number n there is a natural number k such that 11k
	differs from n by less than 11. \\Note that the same proof works when 11 is replaced by any natural number.
\end{manualtheorem}

\subsection{The Division Algorithm}
Let $n, m \in \mathbb{N}$ with $n > 0$.
\begin{theorem*}
	\textbf{Existence}
	There exist integers $q$ (quotient) and $r$ (remainder) such that
	\[
		m = nq + r, \qquad 0 \leq r \leq n - 1.
	\]
\end{theorem*}

\begin{theorem*}
	\textbf{Uniqueness}
	If $q, q'$ and $r, r'$ are integers such that
	\[
		m = nq + r = nq' + r', \qquad 0 \leq r, r' \leq n - 1,
	\]
	then $q = q'$ and $r = r'$.
\end{theorem*}


\subsection{GCD and Linear Diophantine Equations}

\begin{definition*}
	A common divisor of integers $a$ and $b$ is an integer $d$ such that $d \mid a$ and $d \mid b$.
\end{definition*}

\begin{definition*}
	The greatest common divisor of two integers $a$ and $b$, not both $0$, is the largest integer $d$ such that $d \mid a$ and $d \mid b$. The greatest common divisor of two integers $a$ and $b$ is denoted $\gcd(a, b)$ or more briefly as just $(a, b)$.
\end{definition*}

\begin{definition*}
	If $\gcd(a, b) = 1$, then $a$ and $b$ are said to be relatively prime.
\end{definition*}
\begin{manualtheorem}{1.32}
	Let $a, n, b, r,$ and $k$ be integers. If $a = nb + r$ and $k \mid a$ and $k \mid b$,
	then $k \mid r$.
\end{manualtheorem}

\begin{manualtheorem}{1.33}
	Let $a, b, n_{1}, r_{1}$ be integers with $a$ and $b$ not both $0$.
	If $a = n_{1}b + r_{1}$, then $(a, b) = (b, r_{1})$.
\end{manualtheorem}

\begin{theorem*}
	The Euclidean algorithm terminates after finitely many steps and outputs
	(a, b), the greatest common divisor of a and b.
\end{theorem*}

\begin{theorem}
	The Extended Euclidean Algorithm terminates after finitely many steps and
	produces integers $x$ and $y$ such that
	\[
		\gcd(a,b) = ax + by.
	\]
\end{theorem}


\begin{manualtheorem}{1.38}
	If $(a,b)=1$, then there exist integers $x,y$ such that $ax+by=1$.
\end{manualtheorem}

\begin{manualtheorem}{1.39}
	If there exist integers $x,y$ such that $ax+by=1$, then $(a,b)=1$.
\end{manualtheorem}

\begin{manualtheorem}{1.40}
	(\textbf{B\'ezout's Identity}) For any integers $a,b$, not both $0$, there exist integers $x,y$ such that $ax+by=(a,b)$.
\end{manualtheorem}

\begin{manualtheorem}{1.41}
	(\textbf{Euclid's Lemma}) If $a \mid bc$ and $(a,b)=1$, then $a \mid c$.
\end{manualtheorem}

\begin{manualtheorem}{1.42}
	If $a \mid n$, $b \mid n$, and $(a,b)=1$, then $ab \mid n$.
\end{manualtheorem}

\begin{manualtheorem}{1.43}
	If $(a,n)=1$ and $(b,n)=1$, then $(ab,n)=1$.
\end{manualtheorem}

\begin{manualtheorem}{1.48}
	Let $a,b,c \in \mathbb{Z}$ with $a$ and $b$ not both $0$. The linear Diophantine equation $ax+by=c$ has an integer solution $(x,y)$ if and only if $(a,b)\mid c$.
\end{manualtheorem}

\begin{manualtheorem}{1.53}
	Let $a,b,c \in \mathbb{Z}$ with $a$ and $b$ not both $0$. Suppose $(x_0,y_0)$ is one integer solution of $ax+by=c$. Then all integer solutions are
	\[
		x = x_0 + \frac{b}{(a,b)}\,k,\qquad
		y = y_0 - \frac{a}{(a,b)}\,k,\qquad k \in \mathbb{Z}.
	\]
	Conversely, every integer solution arises uniquely in this way from some $k \in \mathbb{Z}$.
\end{manualtheorem}

\begin{manualtheorem}{1.55}
	For $a,b \in \mathbb{Z}$ not both $0$ and $k \in \mathbb{N}$, \quad $\gcd(ka,kb)=k\cdot \gcd(a,b)$.
\end{manualtheorem}



\pagebreak

\section{Primes}
\begin{definition*}
	A natural number $p > 1$ is $prime$ if and only if $p$ cannot be written as
	the product of natural numbers less than $p$.
\end{definition*}

\begin{definition*}
	A natural number $n$ is $composite$ if and only if $n$ is not $prime$. That is,
	$n$ is a product of natural numbers less than $n$.
\end{definition*}

\begin{manualtheorem}{2.1}
	If $n$ is a natural number greater than 1, then there exists a prime $p$
	such that $p \mid n$.
\end{manualtheorem}

\begin{manualtheorem}{2.3}
	A natural number $n > 1$ is prime if and only if for all primes $p \leq \sqrt{n}$,
	p does not divide n.
\end{manualtheorem}

\begin{manualtheorem}{2.7}
	\textbf{Fundamental Theorem of Arithmetic (Existence Part).}
	Every natural number greater than $1$ is either a prime number or can be expressed as a finite product of prime numbers.
	That is, for every natural number $n > 1$, there exist distinct primes $p_1, p_2, \ldots, p_m$ and natural numbers $r_1, r_2, \ldots, r_m$ such that

	\[
		n = p_1^{r_1} p_2^{r_2} \cdots p_m^{r_m}.
	\]
\end{manualtheorem}

\begin{manualtheorem}{2.9}
	\textbf{Fundamental Theorem of Arithmetic (Uniqueness Part).}
	Let $n$ be a natural number. Let $\{p_{1}, p_{2}, \ldots, p_{m}\}$ and $\{q_{1}, q_{2}, \ldots, q_{s}\}$ be sets of primes with $p_{i} \neq p_{j}$ if $i \neq j$ and $q_{i} \neq q_{j}$ if $i \neq j$. Let $\{r_{1}, r_{2}, \ldots, r_{m}\}$ and $\{t_{1}, t_{2}, \ldots, t_{s}\}$ be sets of natural numbers such that

	\[
		n = p_{1}^{r_{1}} p_{2}^{r_{2}} \cdots p_{m}^{r_{m}}
		= q_{1}^{t_{1}} q_{2}^{t_{2}} \cdots q_{s}^{t_{s}}.
	\]
\end{manualtheorem}

\begin{manualtheorem}{2.19}
	Show that there do not exist natural numbers $m$ and $n$ such that
	\[
		2m^2 = n^2.
	\]
\end{manualtheorem}

\begin{manualtheorem}{2.27}
	Let $p$ be a prime and let $a, b \in \mathbb{Z}$. If $p \mid ab$, then $p \mid a$ or $p \mid b$.
\end{manualtheorem}

\begin{manualtheorem}{2.32}
	For all natural numbers n, (n, n+1) = 1.
\end{manualtheorem}

\begin{manualtheorem}{2.33}
	Let k be a natural number. There exists a natural number n (which will be larger than k) such that no natural number less than k and greater than 1 divides n.
\end{manualtheorem}
\begin{manualtheorem}{2.34}
	Let k be a natural number. Then there exists a prime larger than
	k.
\end{manualtheorem}
\begin{manualtheorem}{2.35}

	\textbf{(Infinitude of Primes).} There are infinitely many prime numbers.
\end{manualtheorem}

\begin{manualtheorem}{2.46}
	There exist arbitrarily long strings of consecutive composite numbers. That is, for any natural number n there is a string of n consecutive composite
	numbers.
\end{manualtheorem}

\begin{manualtheorem}
	As n approaches infinity, the proportion of natural numbers up to n which are prime, $\pi (n)/n$,
	approaches $1 / ln(n)$, that is,
	\[
		\lim_{n \to \infty} \frac{\pi(n)}{n / \ln(n)} = 1.
	\]
\end{manualtheorem}

\section{A Modular World}

\begin{manualtheorem}{3.8}
	Suppose \(f(x) = a_n x^n + a_{n-1} x^{n-1} + \cdots + a_1 x + a_0\)
	is a polynomial of degree $n > 0$ with integer coefficients.
	Let $a, b, m \in \mathbb{Z}$ with $m > 0$.
	If \(a \equiv b \pmod{m},\)
	then \(f(a) \equiv f(b) \pmod{m}.\)
\end{manualtheorem}

\begin{manualtheorem}{3.14}
	Given any integer $a$ and any natural number $n$, there exists a unique integer \\ $t
		\in \{0,1,2,\ldots,n-1\}$ such that $a \equiv t \pmod{n}$.
\end{manualtheorem}

\begin{definition*}
	Let $n$ be a natural number. The set $\{0,1,2,\ldots,n-1\}$ is
	called the canonical complete residue system modulo $n$.
\end{definition*}

\begin{definition*}
	Let $n$ be a natural number. A set $\{a_{1}, a_{2}, \ldots, a_{k}\}$ of integers is called a
	\emph{complete residue system modulo $n$} if every integer is congruent modulo $n$ to exactly
	one element of the set.
\end{definition*}


\begin{manualtheorem}{3.17}
	Let $n$ be a natural number. Any set $\{a_1, a_2, \ldots, a_n\}$ of $n$ integers
	for which no two are congruent modulo $n$ is a complete residue system modulo $n$.
\end{manualtheorem}

\begin{manualtheorem}{3.19}
	Let $a, b, c \in \mathbb{Z}$ with $b > 0$. The congruence
	\[
		ax \equiv c \pmod{b}
	\]
	has a solution if and only if there exist integers $x$ and $y$ such that
	\[
		ax + by = c.
	\]
\end{manualtheorem}

\begin{manualtheorem}{3.24}
	Let $a, b, n \in \mathbb{Z}$ with $n > 0$. Then
	\begin{enumerate}
		\item The congruence $ax \equiv b \pmod{n}$ is solvable in integers if and only if $\gcd(a,n) \mid b$.
		\item If $x_{0}$ is a solution to the congruence $ax \equiv b \pmod{n}$, then all solutions are given by
		      \[
			      x \equiv x_{0} + \frac{n}{\gcd(a,n)} \cdot m \pmod{n}, \quad m \in \mathbb{Z}.
		      \]
		\item If $ax \equiv b \pmod{n}$ has a solution, then there are exactly $\gcd(a,n)$ incongruent solutions modulo $n$.
	\end{enumerate}

\end{manualtheorem}

\begin{manualtheorem}{3.27}
	Let $a, b, m, n \in \mathbb{Z}$ with $m > 0$ and $n > 0$. Then the system
	\[
		\begin{cases}
			x \equiv a \pmod{n}, \\
			x \equiv b \pmod{m}
		\end{cases}
	\]
	has a solution if and only if $\gcd(n,m) \mid (a - b)$.
\end{manualtheorem}

\begin{manualtheorem}{3.28}
	Let $a, b, m, n \in \mathbb{Z}$ with $m > 0$ and $n > 0$, and suppose $\gcd(m,n) = 1$.
	Then the system
	\[
		\begin{cases}
			x \equiv a \pmod{n}, \\
			x \equiv b \pmod{m}
		\end{cases}
	\]
	has a unique solution modulo $nm$.
\end{manualtheorem}

\begin{manualtheorem}{3.29}[Chinese Remainder Theorem]
	Let $L \in \mathbb{N}$. Suppose $a_{1}, a_{2}, \ldots, a_{L}$ are integers and \\
	$n_{1}, n_{2}, \ldots, n_{L}$ are pairwise relatively prime natural numbers
	(that is, $\gcd(n_{i}, n_{j}) = 1$ for $i \neq j$, $1 \leq i, j \leq L$).
	Then the system of $L$ congruences
	\[
		\begin{cases}
			x \equiv a_{1} \pmod{n_{1}}, \\
			x \equiv a_{2} \pmod{n_{2}}, \\
			\ \ \vdots                   \\
			x \equiv a_{L} \pmod{n_{L}}
		\end{cases}
	\]
	has a unique solution modulo $n_{1} n_{2} \cdots n_{L}$.
\end{manualtheorem}

\textbf{How to solve:}
\[
	\begin{cases}
		x \equiv a \pmod{m} \\
		x \equiv b \pmod{n}
	\end{cases}
\]

\begin{enumerate}
	\item Identify the values:
	      \[
		      x \equiv a \pmod{m}, \quad x \equiv b \pmod{n}
	      \]

	\item Confirm that \( m \) and \( n \) are coprime: \( \gcd(m, n) = 1 \). If true, a unique solution exists modulo \( mn \).

	\item Use the Extended Euclidean Algorithm to find integers \( u \) and \( v \) such that:
	      \[
		      mu + nv = 1
	      \]

	\item Multiply both sides of the equation by \( a - b \):
	      \[
		      m u (a - b) + n v (a - b) = a - b
	      \]

	\item Rearranging, define:
	      \[
		      x_0 = a - m \cdot u \cdot (a - b)
	      \]
	      Then \( x_0 \equiv a \pmod{m} \) and \( x_0 \equiv b \pmod{n} \).

	\item The general solution is:
	      \[
		      x \equiv x_0 \pmod{mn}
		      \quad \text{or} \quad
		      x = x_0 + mn \cdot k, \quad k \in \mathbb{Z}
	      \]
\end{enumerate}

\section{Fermat’s Little Theorem and Euler’s Theorem}

\begin{definition*}
	Let $a$ and $n$ be natural numbers with $\gcd(a,n) = 1$. The smallest natural number $k$ such that $a^{k} \equiv 1 \pmod{n}$ is called the \emph{order} of $a$ modulo $n$ and is denoted $\operatorname{ord}_{n}(a)$.
\end{definition*}

\begin{manualtheorem}{4.2}
	Let $a$ and $n$ be natural numbers with $\gcd(a,n) = 1$. Then $\gcd(a^{j}, n) = 1$ for any natural number $j$.
\end{manualtheorem}

\begin{manualtheorem}{4.4}
	Let $a$ and $n$ be natural numbers. Then there exist natural numbers $i$ and $j$ with $i \ne j$ such that $a^{i} \equiv a^{j} \pmod{n}$.
\end{manualtheorem}

\begin{manualtheorem}{4.6}
	Let $a$ and $n$ be natural numbers with $\gcd(a,n) = 1$. Then there exists a natural number $k$ such that $a^{k} \equiv 1 \pmod{n}$.
\end{manualtheorem}

\begin{manualtheorem}{4.8}
	Let $a$ and $n$ be natural numbers with $\gcd(a,n) = 1$, and let
	$k = \operatorname{ord}_n(a)$. Then the numbers
	$a^1, a^2, \ldots, a^k$ are pairwise incongruent modulo $n$; that is,
	\[
		a^i \not\equiv a^j \pmod{n} \quad \text{whenever } i \neq j.
	\]
\end{manualtheorem}

\subsection{Fermat’s Little Theorem}

\begin{manualtheorem}{4.9}
	Let $a$ and $n$ be natural numbers with $\gcd(a,n) = 1$ and let $k = \operatorname{ord}_{n}(a)$. For any natural number $m$, $a^{m}$ is congruent modulo $n$ to one of the numbers $a^{1}, a^{2}, \ldots, a^{k}$.

\end{manualtheorem}

\begin{manualtheorem}{4.10}
	Let $a$ and $n$ be natural numbers with $\gcd(a,n) = 1$, let $k = \operatorname{ord}_{n}(a)$, and let $m$ be a natural number. Then $a^{m} \equiv 1 \pmod{n}$ if and only if $k \mid m$.
\end{manualtheorem}

\begin{manualtheorem}{4.11}
	Let $a$ and $n$ be natural numbers with $\gcd(a,n) = 1$. Then $\operatorname{ord}_{n}(a) < n$.
\end{manualtheorem}

\begin{manualtheorem}{4.13}
	Let $p$ be a prime and let $a$ be an integer not divisible by $p$; that is, $\gcd(a,p) = 1$. Then the set $\{a, 2a, 3a, \ldots, (p-1)a\}$ is a complete residue system modulo $p$.

\end{manualtheorem}

\begin{manualtheorem}{4.14}
	Let $p$ be a prime and let $a$ be an integer not divisible by $p$. Then
	\[
		a \cdot 2a \cdot 3a \cdots (p-1)a \equiv 1 \cdot 2 \cdot 3 \cdots (p-1) \pmod{p}.
	\]
\end{manualtheorem}

\begin{manualtheorem}{4.15}[Fermat's Little Theorem, Version I]
	If $p$ is a prime and $a$ is an integer such that $\gcd(a, p) = 1$, then
	\[
		a^{p-1} \equiv 1 \pmod{p}.
	\]
\end{manualtheorem}

\begin{manualtheorem}{4.16}[Fermat's Little Theorem, Version II]
	If $p$ is a prime and $a$ is any integer, then
	\[
		a^{p} \equiv a \pmod{p}.
	\]
\end{manualtheorem}

\begin{manualtheorem}{4.18}
	Let $p$ be a prime and $a$ be an integer. If $\gcd(a, p) = 1$, then
	\[
		\operatorname{ord}_p(a) \mid (p - 1).
	\]
\end{manualtheorem}

\subsection{Euler’s Theorem}

\begin{definition*}
	For a natural number $n$, the Euler $\phi$-function, $\phi(n)$, is equal to the number of natural numbers less than or equal to $n$ that are relatively prime to $n$.
	(Note that $\varphi(1) = 1$.)
\end{definition*}

\begin{manualtheorem}{4.31}
	Let $a, n \in \mathbb{N}$ with $\gcd(a, n) = 1$, and let $x_1, x_2, \ldots, x_{\varphi(n)}$ be the distinct natural numbers less than or equal to $n$ that are relatively prime to $n$.

	Let $i$ and $j$ be natural numbers less than or equal to $\varphi(n)$. Then
	\[
		a x_i \equiv a x_j \pmod{n} \quad \text{implies} \quad i = j.
	\]
\end{manualtheorem}

\begin{manualtheorem}{4.32}[Euler's Theorem]
	If $a$ and $n$ are integers with $n > 0$ and $\gcd(a, n) = 1$, then
	\[
		a^{\varphi(n)} \equiv 1 \pmod{n}.
	\]
\end{manualtheorem}

As long as we can compute $\phi{(n)}$, Euler’s Theorem can be used just like Fermat’s
Little Theorem for computing powers of numbers modulo $n$.

\subsection{Inverses and Wilson’s Theorem}

\begin{definition*}
	Let $p$ be a prime and let $a$ and $b$ be integers such that
	\[
		ab \equiv 1 \pmod{p}.
	\]
	Then $a$ and $b$ are said to be \emph{inverses modulo $p$}.
\end{definition*}

\begin{manualtheorem}{4.36}
	Let $p$ be a prime and let $a$ be an integer such that $1 \leq a < p$. Then there exists a unique natural number $b$ less than $p$ such that
	\[
		ab \equiv 1 \pmod{p}.
	\]
\end{manualtheorem}

\begin{manualtheorem}{4.38}
	Let $p$ be a prime and let $a$ and $b$ be integers such that $1 < a, b < p - 1$ and
	\[
		ab \equiv 1 \pmod{p}.
	\]
	Then $a \neq b$.
\end{manualtheorem}

\begin{manualtheorem}{4.40}
	If $p$ is a prime larger than $3$, then
	\[
		2 \cdot 3 \cdot 4 \cdots (p - 2) \equiv 1 \pmod{p}.
	\]
\end{manualtheorem}

\begin{manualtheorem}{4.41}[Wilson's Theorem]
	If $p$ is a prime, then
	\[
		(p - 1)! \equiv -1 \pmod{p}.
	\]
\end{manualtheorem}

\begin{manualtheorem}{4.42}[Converse of Wilson's Theorem]
	If $n$ is a natural number such that
	\[
		(n - 1)! \equiv -1 \pmod{n},
	\]
	then $n$ is prime.
\end{manualtheorem}

\section{Public Key Cryptography}

\begin{manualtheorem}{4.21}
	Let $n$ and $m$ be natural numbers that are relatively prime, and let $a$ be an integer. If
	\[
		x \equiv a \pmod{n} \quad \text{and} \quad x \equiv a \pmod{m},
	\]
	then
	\[
		x \equiv a \pmod{nm}.
	\]
\end{manualtheorem}

\begin{manualtheorem}{5.1}
	If $p$ and $q$ are distinct primes and $W$ is a natural number with $\gcd(W, pq) = 1$, then
	\[
		W^{(p - 1)(q - 1)} \equiv 1 \pmod{pq}.
	\]
\end{manualtheorem}

\begin{manualtheorem}{5.2}
	Let $p$ and $q$ be distinct primes, $k$ a natural number, and $W$ a natural number less than $pq$. Then
	\[
		W^{1 + k(p - 1)(q - 1)} \equiv W \pmod{pq}.
	\]
\end{manualtheorem}

\begin{manualtheorem}{5.3}
	Let $p$ and $q$ be distinct primes, $E$ a natural number relatively prime to $(p - 1)(q - 1)$. Then there exist natural numbers $D$ and $y$ such that
	\[
		ED = 1 + y(p - 1)(q - 1).
	\]
\end{manualtheorem}

\begin{manualtheorem}{5.4}
	Let $p$ and $q$ be distinct primes, $W$ a natural number less than $pq$, and $E$, $D$, and $y$ be natural numbers such that
	\[
		ED = 1 + y(p - 1)(q - 1).
	\]
	Then
	\[
		W^{ED} \equiv W \pmod{pq}.
	\]
\end{manualtheorem}

What have we proven now? We have shown that raising $W$ to a certain power ($ED$) and reducing modulo $pq$ recovers $W$.

Note that
\[
	W^{ED} = (W^{E})^{D}.
\]

\textbf{RSA Public Key Coding System}

\medskip

The number $W$ is the message Bob wants to send to Alice without anyone else gaining access to what the message says.

\begin{enumerate}
	\item \textbf{Step 1: Key Generation.}

	      Alice does the following:
	      \begin{enumerate}
		      \item Choose two distinct primes $p$ and $q$ and compute the modulus $n = pq$.
		      \item Compute $\phi(n) = (p - 1)(q - 1)$.
		      \item Pick an encoding exponent $E$ with $1 < E < \phi(n)$ and $\gcd(E, \phi(n)) = 1$.
		      \item Find the decoding exponent $D$ by solving
		            \[
			            ED \equiv 1 \pmod{\phi(n)} \quad (1 < D < \phi(n)).
		            \]
	      \end{enumerate}

	      The pair $(n, E)$ is called the \textbf{public key} and the pair $(n, D)$ is called the \textbf{private key}.
	      Alice publishes the public key for anyone to see but keeps her private key secret.

	\item \textbf{Step 2: Encryption.}

	      Now Bob wants to send Alice a message.
	      To encrypt $W$, he uses Alice’s public key to compute
	      \[
		      C := W^{E} \pmod{n}.
	      \]
	      This ciphertext $C$ is sent to Alice.
	      Without knowledge of $n$'s prime decomposition, it is computationally infeasible for anyone to recover $W$ by brute force.
	      Bob does not need to know any private information to send his encrypted message.

	\item \textbf{Step 3: Decryption.}

	      Alice receives $C$ and computes
	      \[
		      C^{D} \pmod{n},
	      \]
	      which equals
	      \[
		      (W^{E})^{D} \equiv W \pmod{n}.
	      \]
	      Thus, she recovers the original message $W$.
\end{enumerate}

\section{Polynomial Congruences and Primitive Roots}

\begin{definition}
	Recall that $r$ is a \emph{root} of the polynomial
	\[
		f(x) = a_n x^n + a_{n-1} x^{n-1} + \cdots + a_0
	\]
	if $f(r) = 0$.
\end{definition}

\begin{manualtheorem}{6.1}
	Let
	\[
		f(x) = a_n x^n + a_{n-1} x^{n-1} + \cdots + a_0
	\]
	be a polynomial of degree $n > 0$ with integer coefficients and assume $a_n \neq 0$.

	Then an integer $r$ is a root of $f(x)$ if and only if there exists a polynomial $g(x)$ of degree $n - 1$ with integer coefficients such that
	\[
		f(x) = (x - r)g(x).
	\]
\end{manualtheorem}

\end{document}

