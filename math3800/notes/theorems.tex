\documentclass{article}

\usepackage{amsmath}
\usepackage{amsthm}
\usepackage{amsfonts}
\usepackage{amssymb}
\usepackage{array}


\topmargin=-0.45in
\evensidemargin=0in
\oddsidemargin=0in
\textwidth=6.5in
\textheight=9.0in
\headsep=0.25in

\linespread{1.1}
\setlength\parindent{0pt}

\newtheorem{theorem}{Theorem}
\newtheorem*{theorem*}{Theorem}
\theoremstyle{definition}
\newtheorem{definition}{Definition}
\newtheorem*{definition*}{Definition}

% Command for manual theorem numbering
\newcommand{\manualnum}[1]{\setcounter{theorem}{\numexpr#1-1\relax}}

% Environment for theorems with manual numbers
\newenvironment{manualtheorem}[1]{%
  \renewcommand{\thetheorem}{#1}%
  \theorem%
}{%
  \endtheorem%
}

\begin{document}
\setcounter{section}{-1}


\section{Introduction}
\[
	\mathbb{N} = \{1,2,3,4,\dots\}
\]
\[
	\mathbb{N}_0 = \{0,1,2,3,4,\dots\}
\]
\[
	\mathbb{Z} = \{\dots,-3,-2,-1,0,1,2,3,\dots\}
\]

\begin{itemize}
	\item $\mathbb{N}$: only positive whole numbers (and maybe $0$).
	\item $\mathbb{Z}$: all whole numbers, positive, negative, and $0$.
\end{itemize}
\subsection{Proof Strategies}

\begin{tabular}{|>{\bfseries}m{3cm}|m{10cm}|}
	\hline
	Strategy                       & Core Idea                                                                                                                                                             \\
	\hline
	Direct                         & Start from given assumptions and proceed step by step until the desired statement emerges.                                                                            \\
	\hline
	Contrapositive                 & Instead of proving ``If $P$ is true, then $Q$ is true,'' prove the equivalent statement ``If $Q$ is false, then $P$ is false.'' Often the negative form is easier.    \\
	\hline
	Contradiction                  & Instead of proving ``If $P$ is true, then $Q$ is true,'' assume ``$P$ is true and $Q$ is false'' and use the assumptions to reach an impossibility (such as $0 = 1$). \\
	\hline
	Induction (ordinary or strong) & Prove an initial case; then show that whenever the statement holds for one (or all earlier) integer(s), it holds for the next.                                        \\
	\hline
\end{tabular}


\section{Divide and Conquer}

\subsection{Divisibility}

\begin{definition*}
	Suppose $a$ and $d$ are integers. Then we say $d$ divides $a$, denoted $d \mid a$, if (and only if) there is an integer $k$ such that $a = kd$. We may also say that $d$ is a divisor or factor of $a$, and that $a$ is a multiple of $d$.
\end{definition*}



\begin{theorem*}
	Let $n$ be an integer. If $14 \mid n$, then $7 \mid n$.
\end{theorem*}

\begin{manualtheorem}{1.1}
	Let $a, b, c \in \mathbb{Z}$. If $a \mid b$ and $a \mid c$, then $a \mid (b + c)$.
\end{manualtheorem}

\begin{manualtheorem}{1.3}
	Let $a, b, c \in \mathbb{Z}$. If $a \mid b$ and $a \mid c$, then $a \mid bc$.
\end{manualtheorem}

\begin{manualtheorem}{1.32}
	Let $a, n, b, r, k \in \mathbb{Z}$. If $a = nb + r$ and $k \mid a$ and $k \mid b$, then $k \mid r$.
\end{manualtheorem}


\subsection{Congruency}
\begin{definition*}
	Suppose $a$, $b$ and $n$ are integers, with $n > 0$. We say that $a$ and $b$ are congruent modulo $n$ if $n \mid (a - b)$. We denote this relationship as
	\(a \equiv b \pmod{n}\)
\end{definition*}



\begin{theorem*}
	Let $k \in \mathbb{Z}$. If $k \equiv 5 \pmod{2}$, then $k \equiv 3 \pmod{2}$.
\end{theorem*}

\begin{manualtheorem}{1.9}
	Let $a, n \in \mathbb{Z}$ with $n > 0$. Then $a \equiv a \pmod{n}$.
\end{manualtheorem}

\begin{manualtheorem}{1.10}
	Let $a, b, n \in \mathbb{Z}$ with $n > 0$. If $a \equiv b \pmod{n}$, then $b \equiv a \pmod{n}$.
\end{manualtheorem}

\begin{manualtheorem}{1.11}
	Let $a, b, c, n \in \mathbb{Z}$ with $n > 0$. If $a \equiv b \pmod{n}$ and $b \equiv c \pmod{n}$, then $a \equiv c \pmod{n}$.
\end{manualtheorem}

\begin{manualtheorem}{1.12}
	Let $a, b, c, d, n \in \mathbb{Z}$ with $n > 0$. If $a \equiv b \pmod{n}$ and $c \equiv d \pmod{n}$, then $a + c \equiv b + d \pmod{n}$.
\end{manualtheorem}

\begin{manualtheorem}{1.14}
	Let $a, b, c, d, n \in \mathbb{Z}$ with $n > 0$. If $a \equiv b \pmod{n}$ and $c \equiv d \pmod{n}$, then $ac \equiv bd \pmod{n}$.
\end{manualtheorem}

\begin{manualtheorem}{1.18}
	Let $a, b, k, n \in \mathbb{Z}$ with $n > 0$ and $k > 0$. If $a \equiv b \pmod{n}$, then $a^k \equiv b^k \pmod{n}$.
\end{manualtheorem}

\begin{manualtheorem}{1.21}
	Let a natural number $n$ be expressed in base 10 as $n = a_ka_{k-1}\ldots a_1a_0$. If $m = a_k + a_{k-1} + \cdots + a_1 + a_0$, then $n \equiv m \pmod{9}$.
\end{manualtheorem}

\begin{manualtheorem}{1.45}
	Let $a, b, c, n \in \mathbb{Z}$ with $n>0$. If $ac \equiv bc \pmod{n}$ and $(c,n)=1$, then $a \equiv b \pmod{n}$.
\end{manualtheorem}


\subsection{Number Properties}
\begin{theorem*}
	For every natural number $n$, $1 + 2 + 2^{2} + \cdots + 2^{n} = 2^{\,n+1}-1$.
\end{theorem*}

\begin{manualtheorem}{A.1}
	For all $n \in \mathbb{N}$, we have $1 + 2 + \cdots + n = \frac{n(n+1)}{2}$.
\end{manualtheorem}

\begin{theorem*}
	The number $n^2 - n$ is even for every $n \in \mathbb{Z}$.
\end{theorem*}

\begin{theorem*}
	Let $S$ be any nonempty set of natural numbers. Then $S$ has a smallest element.
\end{theorem*}

\begin{manualtheorem}{1.27}
For every natural number n there is a natural number k such that 11k
differs from n by less than 11. \\Note that the same proof works when 11 is replaced by any natural number.
\end{manualtheorem}

\subsection{The Division Algorithm}
Let $n, m \in \mathbb{N}$ with $n > 0$.
\begin{theorem*}
	\textbf{Existence}
	There exist integers $q$ (quotient) and $r$ (remainder) such that
	\[
		m = nq + r, \qquad 0 \leq r \leq n - 1.
	\]
\end{theorem*}

\begin{theorem*}
	\textbf{Uniqueness}
	If $q, q'$ and $r, r'$ are integers such that
	\[
		m = nq + r = nq' + r', \qquad 0 \leq r, r' \leq n - 1,
	\]
	then $q = q'$ and $r = r'$.
\end{theorem*}


\subsection{GCD and Linear Diophantine Equations}

\begin{definition*}
	A common divisor of integers $a$ and $b$ is an integer $d$ such that $d \mid a$ and $d \mid b$.
\end{definition*}

\begin{definition*}
	The greatest common divisor of two integers $a$ and $b$, not both $0$, is the largest integer $d$ such that $d \mid a$ and $d \mid b$. The greatest common divisor of two integers $a$ and $b$ is denoted $\gcd(a, b)$ or more briefly as just $(a, b)$.
\end{definition*}

\begin{definition*}
	If $\gcd(a, b) = 1$, then $a$ and $b$ are said to be relatively prime.
\end{definition*}
\begin{manualtheorem}{1.32}
Let $a, n, b, r,$ and $k$ be integers. If $a = nb + r$ and $k \mid a$ and $k \mid b$,
then $k \mid r$.
\end{manualtheorem}

\begin{manualtheorem}{1.33}
	Let $a, b, n_{1}, r_{1}$ be integers with $a$ and $b$ not both $0$.  
If $a = n_{1}b + r_{1}$, then $(a, b) = (b, r_{1})$.
\end{manualtheorem}

\begin{theorem*}
	The Euclidean algorithm terminates after finitely many steps and outputs
(a, b), the greatest common divisor of a and b.
\end{theorem*}

\begin{theorem}
The Extended Euclidean Algorithm terminates after finitely many steps and 
produces integers $x$ and $y$ such that
\[
\gcd(a,b) = ax + by.
\]
\end{theorem}


\begin{manualtheorem}{1.38}
	If $(a,b)=1$, then there exist integers $x,y$ such that $ax+by=1$.
\end{manualtheorem}

\begin{manualtheorem}{1.39}
	If there exist integers $x,y$ such that $ax+by=1$, then $(a,b)=1$.
\end{manualtheorem}

\begin{manualtheorem}{1.40}
	(\textbf{B\'ezout's Identity}) For any integers $a,b$, not both $0$, there exist integers $x,y$ such that $ax+by=(a,b)$.
\end{manualtheorem}

\begin{manualtheorem}{1.41}
	(\textbf{Euclid's Lemma}) If $a \mid bc$ and $(a,b)=1$, then $a \mid c$.
\end{manualtheorem}

\begin{manualtheorem}{1.42}
	If $a \mid n$, $b \mid n$, and $(a,b)=1$, then $ab \mid n$.
\end{manualtheorem}

\begin{manualtheorem}{1.43}
	If $(a,n)=1$ and $(b,n)=1$, then $(ab,n)=1$.
\end{manualtheorem}

\begin{manualtheorem}{1.48}
	Let $a,b,c \in \mathbb{Z}$ with $a$ and $b$ not both $0$. The linear Diophantine equation $ax+by=c$ has an integer solution $(x,y)$ if and only if $(a,b)\mid c$.
\end{manualtheorem}

\begin{manualtheorem}{1.53}
	Let $a,b,c \in \mathbb{Z}$ with $a$ and $b$ not both $0$. Suppose $(x_0,y_0)$ is one integer solution of $ax+by=c$. Then all integer solutions are
	\[
		x = x_0 + \frac{b}{(a,b)}\,k,\qquad
		y = y_0 - \frac{a}{(a,b)}\,k,\qquad k \in \mathbb{Z}.
	\]
	Conversely, every integer solution arises uniquely in this way from some $k \in \mathbb{Z}$.
\end{manualtheorem}

\begin{manualtheorem}{1.55}
	For $a,b \in \mathbb{Z}$ not both $0$ and $k \in \mathbb{N}$, \quad $\gcd(ka,kb)=k\cdot \gcd(a,b)$.
\end{manualtheorem}



\pagebreak

\section{Primes}
\begin{definition*}
	A natural number $p > 1$ is $prime$ if and only if $p$ cannot be written as
	the product of natural numbers less than $p$.
\end{definition*}

\begin{definition*}
	A natural number $n$ is $composite$ if and only if $n$ is not $prime$. That is,
	$n$ is a product of natural numbers less than $n$.
\end{definition*}

\begin{manualtheorem}{2.1}
	If $n$ is a natural number greater than 1, then there exists a prime $p$
	such that $p \mid n$.
\end{manualtheorem}

\begin{manualtheorem}{2.3}
	A natural number $n > 1$ is prime if and only if for all primes $p \leq \sqrt{n}$,
	p does not divide n.
\end{manualtheorem}

\begin{manualtheorem}{2.7}
	\textbf{Fundamental Theorem of Arithmetic (Existence Part).}
	Every natural number greater than $1$ is either a prime number or can be expressed as a finite product of prime numbers.
	That is, for every natural number $n > 1$, there exist distinct primes $p_1, p_2, \ldots, p_m$ and natural numbers $r_1, r_2, \ldots, r_m$ such that

	\[
		n = p_1^{r_1} p_2^{r_2} \cdots p_m^{r_m}.
	\]
\end{manualtheorem}

\begin{manualtheorem}{2.9}
	\textbf{Fundamental Theorem of Arithmetic (Uniqueness Part).}
	Let $n$ be a natural number. Let $\{p_{1}, p_{2}, \ldots, p_{m}\}$ and $\{q_{1}, q_{2}, \ldots, q_{s}\}$ be sets of primes with $p_{i} \neq p_{j}$ if $i \neq j$ and $q_{i} \neq q_{j}$ if $i \neq j$. Let $\{r_{1}, r_{2}, \ldots, r_{m}\}$ and $\{t_{1}, t_{2}, \ldots, t_{s}\}$ be sets of natural numbers such that
	
	\[
		n = p_{1}^{r_{1}} p_{2}^{r_{2}} \cdots p_{m}^{r_{m}}
		  = q_{1}^{t_{1}} q_{2}^{t_{2}} \cdots q_{s}^{t_{s}}.
	\]
\end{manualtheorem}

\begin{manualtheorem}{2.19}
Show that there do not exist natural numbers $m$ and $n$ such that
\[
2m^2 = n^2.
\]
\end{manualtheorem}

\begin{manualtheorem}{2.27}
Let $p$ be a prime and let $a, b \in \mathbb{Z}$. If $p \mid ab$, then $p \mid a$ or $p \mid b$.
\end{manualtheorem}

\begin{manualtheorem}{2.32}
	For all natural numbers n, (n, n+1) = 1.
\end{manualtheorem}

\begin{manualtheorem}{2.33}
	Let k be a natural number. There exists a natural number n (which will be larger than k) such that no natural number less than k and greater than 1 divides n.
\end{manualtheorem}
\begin{manualtheorem}{2.34}
	Let k be a natural number. Then there exists a prime larger than
	k.
\end{manualtheorem}
\begin{manualtheorem}{2.35}

	\textbf{(Infinitude of Primes).} There are infinitely many prime numbers.
\end{manualtheorem}

\begin{manualtheorem}{2.46}
	There exist arbitrarily long strings of consecutive composite numbers. That is, for any natural number n there is a string of n consecutive composite
	numbers.
\end{manualtheorem}

\begin{manualtheorem}
	As n approaches infinity, the proportion of natural numbers up to n which are prime, $\pi (n)/n$,
	approaches $1 / ln(n)$, that is,
	\[
		\lim_{n \to \infty} \frac{\pi(n)}{n / \ln(n)} = 1.
	\]
\end{manualtheorem}

\section{A Modular World}

\begin{manualtheorem}{3.8}
	Suppose \(f(x) = a_n x^n + a_{n-1} x^{n-1} + \cdots + a_1 x + a_0\)
	is a polynomial of degree $n > 0$ with integer coefficients.
	Let $a, b, m \in \mathbb{Z}$ with $m > 0$.
	If \(a \equiv b \pmod{m},\)
	then \(f(a) \equiv f(b) \pmod{m}.\)
\end{manualtheorem}

\begin{manualtheorem}{3.14}
	Given any integer $a$ and any natural number $n$, there exists a unique integer \\ $t
		\in \{0,1,2,\ldots,n-1\}$ such that $a \equiv t \pmod{n}$.
\end{manualtheorem}

\begin{definition*}
	Let $n$ be a natural number. The set $\{0,1,2,\ldots,n-1\}$ is
	called the canonical complete residue system modulo $n$.
\end{definition*}

\begin{definition*}
Let $n$ be a natural number. A set $\{a_{1}, a_{2}, \ldots, a_{k}\}$ of integers is called a 
\emph{complete residue system modulo $n$} if every integer is congruent modulo $n$ to exactly
one element of the set.
\end{definition*}


\begin{manualtheorem}{3.17}
	Let $n$ be a natural number. Any set $\{a_1, a_2, \ldots, a_n\}$ of $n$ integers
	for which no two are congruent modulo $n$ is a complete residue system modulo $n$.
\end{manualtheorem}

\begin{manualtheorem}{3.19}
	Let $a, b, c \in \mathbb{Z}$ with $b > 0$. The congruence
	\[
		ax \equiv c \pmod{b}
	\]
	has a solution if and only if there exist integers $x$ and $y$ such that
	\[
		ax + by = c.
	\]
\end{manualtheorem}

\begin{manualtheorem}{3.24}
Let $a, b, n \in \mathbb{Z}$ with $n > 0$. Then:
\begin{enumerate}
    \item The congruence
    \(
        ax \equiv b \pmod{n}
    \)
    is solvable in integers if and only if $\gcd(a,n) \mid b$.
    
    \item If $x_0$ is a solution to the congruence $ax \equiv b \pmod{n}$, then all solutions are given by
    \[
        x \equiv x_0 + \frac{n}{\gcd(a,n)} \cdot m \pmod{n}, \quad m \in \mathbb{Z}.
    \]
    
    \item If $ax \equiv b \pmod{n}$ has a solution, then there are exactly $\gcd(a,n)$ solutions in the canonical complete residue system modulo $n$.
\end{enumerate}
\end{manualtheorem}


\end{document}

