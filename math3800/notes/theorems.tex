\documentclass{article}

\usepackage{amsmath}
\usepackage{amsthm}
\usepackage{amsfonts}
\usepackage{amssymb}

\topmargin=-0.45in
\evensidemargin=0in
\oddsidemargin=0in
\textwidth=6.5in
\textheight=9.0in
\headsep=0.25in

\linespread{1.1}
\setlength\parindent{0pt}

\newtheorem{theorem}{Theorem}
\newtheorem*{theorem*}{Theorem}
\theoremstyle{definition}
\newtheorem{definition}{Definition}
\newtheorem*{definition*}{Definition}

% Command for manual theorem numbering
\newcommand{\manualnum}[1]{\setcounter{theorem}{\numexpr#1-1\relax}}

% Environment for theorems with manual numbers
\newenvironment{manualtheorem}[1]{%
  \renewcommand{\thetheorem}{#1}%
  \theorem%
}{%
  \endtheorem%
}

\begin{document}

\section{Definitions}

\begin{definition*}
The natural numbers are the positive integers $\{1, 2, 3, 4, \ldots\}$. We denote the natural numbers by $\mathbb{N}$ or $\mathbb{Z}_{>0}$. If we wish to include $0$, we write $\mathbb{N}_0$ or $\mathbb{Z}_{\geq 0}$.
\end{definition*}

\begin{definition*}
Suppose $a$ and $d$ are integers. Then we say $d$ divides $a$, denoted $d \mid a$, if (and only if) there is an integer $k$ such that $a = kd$. We may also say that $d$ is a divisor or factor of $a$, and that $a$ is a multiple of $d$.
\end{definition*}

\begin{definition*}
Suppose $a$, $b$ and $n$ are integers, with $n > 0$. We say that $a$ and $b$ are congruent modulo $n$ if $n \mid (a - b)$. We denote this relationship as
\[a \equiv b \pmod{n}\]
and read these symbols as ``$a$ is congruent to $b$ modulo $n$''.
\end{definition*}

\begin{definition*}
A common divisor of integers $a$ and $b$ is an integer $d$ such that $d \mid a$ and $d \mid b$.
\end{definition*}

\begin{definition*}
The greatest common divisor of two integers $a$ and $b$, not both $0$, is the largest integer $d$ such that $d \mid a$ and $d \mid b$. The greatest common divisor of two integers $a$ and $b$ is denoted $\gcd(a, b)$ or more briefly as just $(a, b)$.
\end{definition*}

\begin{definition*}
If $\gcd(a, b) = 1$, then $a$ and $b$ are said to be relatively prime.
\end{definition*}

\section{Divisibility Theorems}

\begin{theorem*}
Let $n$ be an integer. If $14 \mid n$, then $7 \mid n$.
\end{theorem*}

\begin{manualtheorem}{1.1}
Let $a, b, c \in \mathbb{Z}$. If $a \mid b$ and $a \mid c$, then $a \mid (b + c)$.
\end{manualtheorem}

\begin{manualtheorem}{1.3}
Let $a, b, c \in \mathbb{Z}$. If $a \mid b$ and $a \mid c$, then $a \mid bc$.
\end{manualtheorem}

\begin{manualtheorem}{1.32}
Let $a, n, b, r, k \in \mathbb{Z}$. If $a = nb + r$ and $k \mid a$ and $k \mid b$, then $k \mid r$.
\end{manualtheorem}

\begin{manualtheorem}{1.33}
Let $a, b, n_1, r_1 \in \mathbb{Z}$ with $a$ and $b$ not both $0$. If $a = n_1b + r_1$, then $\gcd(a, b) = \gcd(b, r_1)$.
\end{manualtheorem}

\section{Congruence Theorems}

\begin{theorem*}
Let $k \in \mathbb{Z}$. If $k \equiv 5 \pmod{2}$, then $k \equiv 3 \pmod{2}$.
\end{theorem*}

\begin{manualtheorem}{1.9}
Let $a, n \in \mathbb{Z}$ with $n > 0$. Then $a \equiv a \pmod{n}$.
\end{manualtheorem}

\begin{manualtheorem}{1.10}
Let $a, b, n \in \mathbb{Z}$ with $n > 0$. If $a \equiv b \pmod{n}$, then $b \equiv a \pmod{n}$.
\end{manualtheorem}

\begin{manualtheorem}{1.11}
Let $a, b, c, n \in \mathbb{Z}$ with $n > 0$. If $a \equiv b \pmod{n}$ and $b \equiv c \pmod{n}$, then $a \equiv c \pmod{n}$.
\end{manualtheorem}

\begin{manualtheorem}{1.12}
Let $a, b, c, d, n \in \mathbb{Z}$ with $n > 0$. If $a \equiv b \pmod{n}$ and $c \equiv d \pmod{n}$, then $a + c \equiv b + d \pmod{n}$.
\end{manualtheorem}

\begin{manualtheorem}{1.14}
Let $a, b, c, d, n \in \mathbb{Z}$ with $n > 0$. If $a \equiv b \pmod{n}$ and $c \equiv d \pmod{n}$, then $ac \equiv bd \pmod{n}$.
\end{manualtheorem}

\begin{manualtheorem}{1.18}
Let $a, b, k, n \in \mathbb{Z}$ with $n > 0$ and $k > 0$. If $a \equiv b \pmod{n}$, then $a^k \equiv b^k \pmod{n}$.
\end{manualtheorem}

\begin{manualtheorem}{1.21}
Let a natural number $n$ be expressed in base 10 as $n = a_ka_{k-1}\ldots a_1a_0$. If $m = a_k + a_{k-1} + \cdots + a_1 + a_0$, then $n \equiv m \pmod{9}$.
\end{manualtheorem}

\section{Summation Theorems}

\begin{theorem*}
For every natural number $n$, $1 + 2 + 2^{2} + \cdots + 2^{n} = 2^{\,n+1}-1$.
\end{theorem*}

\begin{manualtheorem}{A.1}
For all $n \in \mathbb{N}$, we have $1 + 2 + \cdots + n = \frac{n(n+1)}{2}$.
\end{manualtheorem}

\section{Number Properties}

\begin{theorem*}
The number $n^2 - n$, is even for every $n \in \mathbb{Z}$.
\end{theorem*}

\section{Fundamental Theorems}

\begin{theorem*}
Let $S$ be any nonempty set of natural numbers. Then $S$ has a smallest element.
\end{theorem*}

\begin{manualtheorem}{1.26}
Let $n$ and $m$ be natural numbers. Then there exist integers $q$ and $r$ such that $m = nq + r$ and $0 \leq r \leq n - 1$.
\end{manualtheorem}

\begin{manualtheorem}{1.27}
If $q, q'$ and $r, r'$ are integers that satisfy $m = nq + r = nq' + r'$ with $0 \leq r, r' \leq n - 1$, then $q = q'$ and $r = r'$.
\end{manualtheorem}

\begin{theorem*}
The Euclidean algorithm terminates after finitely many steps and outputs $\gcd(a, b)$, the greatest common divisor of $a$ and $b$.
\end{theorem*}

\end{document}
